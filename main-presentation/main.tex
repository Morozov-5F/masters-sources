%!TEX TS-program = xelatex
%!TEX encoding = UTF-8 Unicode
\documentclass[russian,hyperref={unicode}]{beamer}
% Other packages (for math and grahics)
\usepackage{amsmath, amssymb, graphicx, csquotes}
\graphicspath{{./images/}}
% Exclude backup slides from numbering
\usepackage{appendixnumberbeamer}
% Theme selection
\usetheme[sectionpage=none, numbering=fraction]{metropolis}
% Locale packages
\usepackage{polyglossia}
\setmainlanguage{russian}
\setkeys{russian}{babelshorthands=true}
\usepackage[backend=biber,
            sorting=nyt,
            bibstyle=gost-authoryear,
            language=auto,
            babel=other,
            doi=false,
            eprint=false,
            isbn=false,
            dashed=false,
            maxbibnames=4,
            ]{biblatex}
\bibliography{../text/bibliography.bib}

\title{Алгоритмы определения пространственной ориентации подвижных объектов в задачах радионавигации}
\institute
{
  Воронежский Государственный Университет \\
  Факультет Компьютерных Наук \\
  Кафедра Цифровых Технологий
}
\author
{
  Выполнил: студент 2 курса маг. Морозов Е.~Ю. \\
  Руководитель: к.~ф.-м.~н.~Минин Л.~А.
}
\date{6 июня 2019 г.}

\begin{document}
  \frame{\titlepage}

  \section{Введение}
  \begin{frame}{Введение}
    В настоящее время большое распространение получили глобальные навигационные спутниковые системы.

    Основная задача таких систем "--- определение координат.

    Угловые характеристики определяются другими системами (например, VOZ/DME или РСБН) или вспомогательными навигационными приборами.

    Навигационные системы постоянно совершенствуются, но пока нет единой системы, позволяющей с достаточной точностью определять одновременно координаты и угловую ориентацию подвижных объектов.~\nocite{WMMU:2019:IIS, WMMU:2019:RLNC, WMMU:2019:ANTENNAS, WMM:2018}
  \end{frame}

  % \begin{frame}{Введение}
  %   Профессор А.~Д.~Виноградов из Военно-воздушной академии им.~Н.~Е.~Жуковского и Ю.~А.~Гагарина предложил новый метод 
  % \end{frame}

  \section{Постановка задачи}
  \begin{frame}{Постановка задачи}
      Пусть несколько радиоориентиров размещены в точках пространства с известными координатами.

      Подвижный объект оснащен бортовой пеленгаторной антенной, способной для каждого РО определить азимут и угол места в связанной системе координат.

      Необходимо определить пространственную ориентацию подвижного объекта,используя минимальное число РО.
  \end{frame}

  \begin{frame}{Постановка задачи}
    Всего необходимо определить шесть параметров "--- три координаты и три угла Эйлера, определяющих угловую ориентацию подвижного объекта.

    БПА подвижного объекта измеряет два параметра для каждого из радиоориентиров "--- азимут и угол места.

    Таким образом, минимальное число РО равно трем.
  \end{frame}

  \begin{frame}{Схема решения}
    Можно выделить три этапа решения задачи:
    \begin{enumerate}
        \item Нахождение совокупности расстояний от подвижного объекта до радиоориентиров;
        \item Определение координат подвижного объекта;
        \item Нахождение матрицы вращения и связанных с нею углов Эйлера, определяющих угловую ориентацию подвижного объекта.
    \end{enumerate}
    Задачи второго и третьего этапов являются стандартными для радионавигации подвижных объектов.
  \end{frame}

  \begin{frame}{Схема решения. Первый этап}
    В задаче есть две системы координат:
    \begin{itemize}
      \item \textit{Базовая} или \textit{земная} "--- в ней известны координаты РО и в ней требуется определить координаты объекта;
      \item \textit{Связанная} с подвижным объектом "--- может быть повернута относительно базовой, в ней проводятся измерения направлений на РО.
    \end{itemize}

    По измеренным направлениям можно определить косинусы углов между направлениями.

    Фактически, решается задача нахождения длины боковых ребер пирамиды, при известных длинах ребер основания и косинусов плоских углов при вершине.
  \end{frame}

  \begin{frame}{Схема решения. Первый этап}
    \begin{figure}
      \begin{center}
        \includegraphics[width=.8\textheight]{tetrahedron}

        \caption{Схема размещения в пространстве трех радиоориентиров $M_1$, $M_2$ и $M_3$ и подвижного объекта $M_0$}
        \label{fig:tetrahedron}
      \end{center}
    \end{figure}
  \end{frame}

  \begin{frame}{Схема решения. Первый этап}
    Система уравнений для нахождения расстояний (теорема косинусов):
    \begin{equation} \label{eq:system}
      \begin{cases}
        \ell_1^2 + \ell_2^2 - 2 \ell_1 \ell_2 \cos\alpha_{12} = d_{12}^2 \\
        \ell_1^2 + \ell_3^2 - 2 \ell_1 \ell_3 \cos\alpha_{13} = d_{13}^2 \\
        \ell_2^2 + \ell_3^2 - 2 \ell_2 \ell_3 \cos\alpha_{23} = d_{23}^2
      \end{cases}
    \end{equation}
    Нелинейная, однородная система второго порядка.
  \end{frame}

  \section{Особенности решения}
  \begin{frame}{Особенности решения. Первый этап}
    Проведенные расчеты, произведенные с помощью Wolfram Mathematica, показали, что система~\eqref{eq:system} имеет от одного до четырех решений. Отсюда возникают три проблемы:
    \begin{enumerate}
      \item Как численно узнать количество решений? \label{q:1}
      \item Как выбрать <<правильное>> решение? \label{q:2}
      \item Как оценить погрешность нахождения решения при заданных входных данных? \label{q:3}
    \end{enumerate}

  \end{frame}

  \begin{frame}{Особенности решения. Первый этап}
    Проблемы~\ref{q:1} и~\ref{q:2} были решены птуем выбора детерминированного подхода "--- число уравнений равно трем.

    Увеличение количества РО усложняет задачу, посколько из-за погрешностей измерений система~\eqref{eq:system} становится несовместной.

    В таком случае требуется составлять минимизирующий функционал ошибок, сложность которого возрастает из-за нелинейности.
  \end{frame}

  \begin{frame}{Особенности решения. Первый этап}
    Систему~(\ref{eq:system}) можно решать методом Ньютона. Это обусловлено следующим:
    \begin{itemize}
      \item Имеется хорошее начальное приближение;
      \item Быстрая сходимость к решению (5-10 итераций);
      \item В реализации присутствуют только элементарные арифметические действия "--- обратная матрица выписывается явно.
    \end{itemize}

    С.~Н.~Ушаковым была установлена структура и условия динамика появления паразитных решений; это же было подтверждено с помощью компьютерных расчетов.
  \end{frame}

  \section{Заключение}

  \begin{frame}{Вспомогательные задачи}
    В данной работе также рассматриваются другие варианты локальных навигационных систем, в частности, когда в системе присутствует несколько подвижных объектов, или когда подвижный объект оснащен высотомером.

    Также была рассмотрена вспомогательная задача "--- влияние на результаты измерений БПА подстилающей поверхности.
  \end{frame}

  \begin{frame}{Дальнейшие планы}
    Дальнейшие планы заключаются в следующем:
    \begin{itemize}
      \item Произвести оценку погрешностей работы рассмотренной системы;
      \item Сотрудничество с Институтом проблем управления имени В.~А.~Трапезникова;
      \item Подача заявки на патент совместно с А.~Д.~Виноградовым.
      \item Исследование других перспективных конфигураций локальных навигационных систем.
    \end{itemize}
  \end{frame}

  \begin{frame}[allowframebreaks]{Список используемых источников}
    % \bibliographystyle{amsalpha}
    \printbibliography
  \end{frame}

  \frame{\titlepage}
\end{document}
