% XeLaTeX can use any Mac OS X font. See the setromanfont command below.
% Input to XeLaTeX is full Unicode, so Unicode characters can be typed directly into the source.
% The next lines tell TeXShop to typeset with xelatex, and to open and save the source with Unicode encoding.
%!TEX TS-program = xelatex
%!TEX encoding = UTF-8 Unicode
\documentclass[russian,a4paper,14pt]{extarticle}
\usepackage{geometry}
\geometry{a4paper}                   % ... or a4paper or a5paper or ...
\setlength{\parindent}{15mm}
%\renewcommand{\baselinestretch}{1.5}
\usepackage{amsmath, amssymb, mathtools, amsthm}
\usepackage{csquotes}

\usepackage{tocloft}
\usepackage{ifxetex}

\ifxetex
  \usepackage{preamble_xelatex}
\else
  \usepackage{preamble_latex}
\fi

% Images setup
\usepackage{graphicx}
\usepackage{subfig}
\graphicspath{{./images/}}
\renewcommand{\thefigure}{\arabic{section}.\arabic{figure}}
\renewcommand{\thesubfigure}{\asbuk{subfigure}}
\captionsetup[subfigure]{labelformat=empty}
\captionsetup[figure]{labelfont=bf}


% Typesetting
\usepackage{microtype} % Microtypography
\setlength{\emergencystretch}{3em}

% Inline fractions
\usepackage{xfrac, nicefrac}
\newcommand*\rfrac[2]{{}^{#1}\!/_{#2}}

% Package with SI units
\usepackage{siunitx}
\DeclareSIUnit\deciBells{дБ}

\usepackage[titletoc, title]{appendix}
\renewcommand{\appendixtocname}{Приложения}
\renewcommand{\appendixname}{Приложение}

% Setup sections
\newcommand{\sectionbreak}{\clearpage} % Start every section from new page
\usepackage{titlesec}
\usepackage{indentfirst}
\usepackage{setspace}
\onehalfspacing

\titleformat{\section}[block]{\normalsize\bfseries\centering}{\thesection}{1ex}{}
\titleformat{\subsection}[block]{\normalsize\bfseries}{\thesubsection}{1ex}{}

% Prepare listings for Mathematica
\usepackage{listings}
\lstdefinestyle{wolfram_mathematica}{
  language=[5.2]Mathematica,
  morekeywords={Solve, Evaluate, Function, Print, Flatten, Line, Point, Purple, Blue, LightOrange, Orange, Red, Magenta, ComposeFaces, PlotFigure, PlotPointsColored, AssignColorsToPoints, ExtractApex, SolveForThree, SolveForFour4, SolveForFour6, },
  breaklines=true,
  extendedchars=\true,
  inputencoding=utf8,
  tabsize=2,
  escapeinside={\%}{\%},
  basicstyle=\small\ttfamily,
  keywordstyle=\textbf,
  texcl=true,
}
\lstdefinestyle{custom_cpp}{
  language=C++,
  % morekeywords={},
  breaklines=true,
  extendedchars=\true,
  inputencoding=utf8,
  tabsize=2,
  escapeinside={\%}{\%},
  basicstyle=\small\ttfamily,
  keywordstyle=\textbf,
  texcl=true,
}

\usepackage{hyphenat}
\hyphenation{ази-му-таль-но}

\usepackage[backend=biber,
            bibstyle=gost-numeric,
            language=auto,
            babel=other,
            sorting=ntvy,
            doi=false,
            eprint=false,
            isbn=false,
            dashed=false,
            maxbibnames=4
            ]{biblatex}
\addbibresource{./bibliography.bib}
\setcounter{tocdepth}{2}
\addto\captionsrussian{\def\refname{Список использованных источников}}
\renewcommand{\thepage}{{\normalsize\textrm{\upshape\arabic{page}}}}
\usepackage{csquotes}
\usepackage{subfiles}
\numberwithin{equation}{section}

\makeatletter
\renewcommand\@biblabel[1]{#1}
\makeatother

\usepackage[lastpage,user]{zref}
\usepackage{totcount}
\newtotcounter{citnum} %From the package documentation
\def\oldbibitem{} \let\oldbibitem=\bibitem
\def\bibitem{\stepcounter{citnum}\oldbibitem}

\renewcommand{\cftsecleader}{\cftdotfill{\cftdotsep}}
\cftsetindents{section}{0em}{2em}
\cftsetindents{subsection}{0em}{2em}

\renewcommand\cfttoctitlefont{\hfill\normalsize\bfseries}
\renewcommand\cftaftertoctitle{\hfill\mbox{}}

\usepackage{array}
\newcolumntype{P}[1]{>{\centering\arraybackslash}p{#1}}

\newtotcounter{totalappendices}

\begin{document}

\newgeometry{left=3cm, right=1cm, top=2cm, bottom=2cm}

\begin{titlepage}
\subfile{sections/titlepage}
\end{titlepage}
\newgeometry{left=2cm, right=1cm, top=2cm, bottom=2cm}
%% ЗАДАНИЕ НА ВЫПОЛНЕНИЕ
\newpage\thispagestyle{empty}
\addtocounter{page}{1}
\subfile{sections/useless_page}

%% РЕФЕРАТ
\newgeometry{top=15mm, left=30mm, right=15mm, bottom=20mm}
\subfile{sections/abstract}
\tableofcontents
%\maketitle

%% ВВЕДЕНИЕ
\addcontentsline{toc}{section}{Введение}
\section*{\centering Введение}
\subfile{sections/introduction}

%%% ТЕОРЕТИЧЕСКАЯ ЧАСТЬ
\section{Обзор существующих алгоритмов}
\subfile{sections/first_chapter}

%%% ПОСТАНОВКА ЗАДАЧИ
\section{Постановка задачи}
\subfile{sections/second_chapter}

%%% ЭКСПЕРИМЕНТАЛЬНАЯ ЧАСТЬ
\section{Локальные навигационные системы}
\subfile{sections/third_chapter}

%%% Обсуждение результатов
\section{Обсуждение результатов}
\subfile{sections/fourth_chapter}

%%% ЗАКЛЮЧЕНИЕ
\section*{\centering Заключение}
\nocite{*}
\addcontentsline{toc}{section}{Заключение}
\subfile{sections/conclusion}

%%% СПИСОК ЛИТЕРАТУРЫ
%TEST:
\newpage
\addcontentsline{toc}{section}{Список использованных источников}
\printbibliography
% \bibliography{bibliography}
% \bibliographystyle{ugost2003}

%%% ПРИЛОЖЕНИЕ
\begin{appendices}
\renewcommand{\thesection}{\Asbuk{section}}
\titleformat{\section}[display]
    {\normalfont\bfseries}{\addtocounter{totalappendices}{1}\appendixname\enspace\centering\thesection}{.5em}{}
\addtocontents{toc}{\protect\setcounter{tocdepth}{1}}

\section{Определения систем координат}
\subfile{sections/appendix_one}

\section{Листинг программы для моделирования простейшей ЛНС}\label{sec:math_code}
\subfile{sections/appendix_two}

\end{appendices}

\end{document}