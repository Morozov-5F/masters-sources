% !TEX root = ../main.tex
\documentclass[../main.tex]{subfiles}
\begin{document}
Принципиальная возможность реализации предложенных в работе алгоритмов обусловлена тем, что есть возможность проводить измерения направлений от нескольких источников практически одновременно с достаточной точностью. Привлечение дополнительных источников призвано, вообще говоря, повысить точность. Но, в нашем случае, рассмотрение переопределенных несовместных нелинейных систем резко усложняет задачу. Покажем, что стандартные минимизационные методы, такие как метод наименьших квадратов, не приводят к упрощению нахождения искомых параметров. Выпишем минимизирующий функционал ошибок для $N$ радиоориентиров ($N \geq 4$):
\begin{equation*}
    \Phi\left(\ell_1, \cdots, \ell_N \right) = \frac{1}{2}\sum_{1 \leq j < i \leq N} \left(\ell_i^2 + \ell_j^2 - 2 \ell_i \ell_j \cos\alpha_{ij} - d_{ij}^2\right)^2
\end{equation*}
Приравняем к нулю частные производные по $\ell_1, \cdots, \ell_N$:
\begin{equation}\label{eq:nf:diff}
      \frac{\partial\Phi}{\partial\ell_k} = \sum_{j \ne k} \left(\ell_k^2 + \ell_j^2 - 2 \ell_k \ell_j \cos\alpha_{kj} - d_{kj}^2\right)\left(2 \ell_k - 2 \ell_j \cos\alpha_{kj}\right) = 0,
\end{equation}
где $k = 1, 2, \cdots, N$. Отсюда, нужно решить систему из $N$ кубических уравнений вида~\eqref{eq:nf:diff} относительно $N$ переменных. Для четырех источников получаем систему из четырех однородных кубических уравнений относительно четырех неизвестных. При этом, система уравнений для трех источников квадратична и имеет всего три уравнения. Поэтому было принято решение ограничиться малым числом источников.

Покажем, что небольшие погрешности измерений не приводят к несовместности системы~\eqref{eq:tetrahedron:system}. Это обстоятельство оказывается существенным, поскольку мы остаемся в рамках детерминированной модели. Следовательно, можно обойтись без функционала ошибок, построение и изучение свойств которого в случае нелинейных систем является предметом серьезных дополнительных исследований. Для этого воспользуемся результатами, изложенными в пункте~\ref{sec:tetrahedron:math}. На поверхности закрытого тора один из плоских углов при вершине пирамиды оказывается постоянным. Если двигаться по линии пересечения двух таких торов, то два плоских угла при вершине пирамиды остаются постоянными, а третий меняется. Следовательно, мы можем поочередно изменить на небольшую величину все три угла. Это дает нам возможность построить пирамиду, соответствующую данным с погрешностями. Следовательно, система уравнений~\eqref{eq:tetrahedron:system} остается совместной, и не требуется строить функционал ошибок.

Когда источников больше трех и никакие три из них не лежат на одной прямой, предлагается следующая схема расчета: выбирается несколько троек, для каждой из них решается система~\eqref{eq:tetrahedron:system}, полученные решения согласуются с помощью метода наименьших квадратов или простым усреднением. Построение процедуры отсечения паразитных решений за счет большего числа источников требует дополнительных исследований. Более продуктивным выглядит подход, когда из имеющихся источников в каждый момент времени выбирается тройка, наиболее удобная для расчета. Критерии оптимальности выбора тройки следует формировать из соображений устойчивости проводимых расчетов.

Если можно расположить три или четыре источника так, что содержащая их плоскость наклонена относительно горизонта, то конфигурация допустимой области изменяется, уходя не вертикально вверх, а под углом к поверхности Земли. Но здесь приходится дополнительно учитывать то обстоятельство, что углы тангажа, близкие к вертикальным, хуже регистрируются измерительной аппаратурой. Поэтому несколько сужаются диапазоны угловых отклонений подвижного объекта, при которых возможно устойчивое определение параметров этих отклонений.

В целом, можно сделать следующее заключение. Решение задачи определения расстояний от подвижного объекта до радиоориентиров имеет сложную структуру. Аналитические методы решения, в принципе, возможны, но приводят к уравнениям высоких степеней с возможным вырождением порядка уравнения. Это означает неустойчивость прямых методов решения. По-видимому, возможно применение методов регуляризации, но это требует математического обоснования.

В работе предлагается устойчивый численный метод решения, позволяющий в случае регулярного мониторинга подвижного объекта отсекать лишние решения. Предлагаемая методика позволяет для заданной конфигурации радиоориентиров определять области устойчивости, т.е. рекомендуемые зоны передвижения подвижных объектов. Можно численно решать и обратную задачу: по заданным траекториям передвижения определять оптимальную конфигурацию радиоориентиров.

\end{document}