% !TEX root = ../main.tex
\documentclass[../main.tex]{subfiles}
\begin{document}
В этой главе рассмотрены варианты локальных угломерных навигационных систем, которые позволяют определять координаты и угловую ориентацию подвижных объектов.

%
% TODO: Привести в соответствие с первой статьей
%
%
\subsection{Простейшая локальная навигационная система}
Пусть три радиоориентира расположены в точках $M_1\left(x_1, y_1, z_1\right)$, $M_2\left(x_2, y_2, z_2\right)$ и $M_3\left(x_3, y_3, z_3\right)$ с заданными координатами и находятся не на одной прямой, а ФЦ БПА, размещенной на подвижном объекте находится в точке $M_0\left(x, y, z\right)$. В таком случае, эти четыре точки образуют в пространстве треугольную пирамиду, схематическое представление которой приведено на рисунке \ref{fig:tetrahedron:pic1}. Обозначим через $\ell_i$ длину бокового ребра $M_0M_i$, через $d_{ij}$ "--- длину ребра основания $M_iM_j$, а через $\alpha_{ij}$ "--- плоский угол $\angle M_iM_0M_j$ при вершине пирамиды. Пространственное положение подвижного объекта и радиоориентиров также будем определять с помощью радиус-векторов $\mathbf{r}_0 = \left(x, y, z\right)$ и $\mathbf{r}_i = \left(x_i, y_i, z_i\right)$ соответственно.

\begin{figure}[htbp]
    \centering

    \fbox{\includegraphics[width=0.6\columnwidth]{3/tetrahedron/pic1}}

    \caption{Схема размещения в пространстве трех радиоориентиров и фазового центра БПА}
    \label{fig:tetrahedron:pic1}
\end{figure}

Допустим, что в результате азимутально-угломестного радиопеленгования радиоориентиров с использованием БПА определены три пары азимутов $\alpha_{i}$ и углов места $\varepsilon_{i}$. Тогда в связанной системе координат БПА можно определить три единичных вектора $\mathbf{s}_{\text{св}i}$ направлений на каждый из радиоориентиров следующим образом:
\begin{equation}\label{eq:vecs}
    \mathbf{s}_{\text{св}i} = \left(\cos\alpha_i \cos\varepsilon_i, \sin\alpha_i\cos\varepsilon_i, \sin\varepsilon_i\right).
\end{equation}
С учетом (\ref{eq:vecs}), косинусы плоских углов $\alpha_{ij}$ равны
\begin{equation}
    \cos\alpha_{ij} = \left(\mathbf{s}_{\text{св}i}, \mathbf{s}_{\text{св}j}\right) =
    \cos\varepsilon_i \cos\varepsilon_j \cos\left(\alpha_i - \alpha_j\right) + \sin\varepsilon_i \sin\varepsilon_j
\end{equation}

С учетом вышеупомянутых значений, система уравнений для нахождения неизвестных значений длин $\ell_1$, $\ell_2$
и $\ell_3$ имеет следующий вид:
\begin{equation} \label{eq:system}
    \begin{cases}
    \ell_1^2 + \ell_2^2 - 2 \ell_1 \ell_2 \cos\alpha_{12} = d_{12}^2 \\
    \ell_1^2 + \ell_2^2 - 2 \ell_1 \ell_2 \cos\alpha_{12} = d_{12}^2 \\
    \ell_1^2 + \ell_2^2 - 2 \ell_1 \ell_2 \cos\alpha_{12} = d_{12}^2
    \end{cases}
\end{equation}

Как показали вычислительные эксперименты, система уравнений (\ref{eq:system}) относительно искомых значений $\ell_1$, $\ell_2$ и $\ell_3$ может иметь от одного до четырех решений в каждой из областей пространства, находящихся симметрично относительно плоскости расположения трех радиоориентиров. Структура этих решений и правила отбора истинного решения будут описаны далее. Предположим, что величины $\ell_1$, $\ell_2$ и $\ell_3$ найдены. В таком случае, для неизвестных $x$, $y$ и $z$ точки $M_0\left(x, y, z\right)$ расположения ФЦ БПА получим следующую систему из трех уравнений
\begin{equation}\label{eq:system_coordinates}
    \begin{cases}
        \left(x_1 - x\right)^2 + \left(y_1 - y\right)^2 + \left(z_1 - z\right)^2 = \ell_1^2 \\
        \left(x_2 - x\right)^2 + \left(y_2 - y\right)^2 + \left(z_2 - z\right)^2 = \ell_2^2 \\
        \left(x_3 - x\right)^2 + \left(y_3 - y\right)^2 + \left(z_3 - z\right)^2 = \ell_3^2
    \end{cases}
\end{equation}
Для решения системы уравнений (\ref{eq:system_coordinates}), вычтем из второго и третьего уравнений первое и перенесем члены, связанные с $z$ в правые части уравнений, в результате чего получим линейную относительно $x$ и $y$ систему уравнений:
\begin{equation}\label{eq:system_linear}
    \begin{cases}
        2x \left(x_1 - x_2\right) + 2 y \left(y_1 - y_2\right) =\ell_2^2 - \ell_1^2 + x_1^2 - x_2^2 + y_1^2 - y_2^2 + z_1^2 - z_2^2 - 2z\left(z_1 - z_2\right) \\
        2x \left(x_1 - x_3\right) + 2 y \left(y_1 - y_3\right) =\ell_3^2 - \ell_1^2 + x_1^2 - x_3^2 + y_1^2 - y_3^2 + z_1^2 - z_3^2 - 2z\left(z_1 - z_3\right)
    \end{cases}
\end{equation}
Выразив, используя правило Крамера, $x$ и $y$ через $z$ из системы (\ref{eq:system_linear}) и подставив полученные выражения в (\ref{eq:system_coordinates}), получим квадратное уравнение относительно $z$. Следует заметить, что определитель системы (\ref{eq:system_linear}) не равен нулю, так как точки $M_1$, $M_2$ и $M_3$ не лежат на одной прямой.

После нахождения координат точки $M_0$, можно определить три единичных вектора $\mathbf{s}_{\text{н}i}$ направлений на $i$-е радиоориентиры в соответствии с соотношением
\begin{equation*}
    \mathbf{s}_{\text{н}i} = \frac{\mathbf{r}_0 - \mathbf{r}_1}{|\mathbf{r}_0 - \mathbf{r}_1|}.
\end{equation*}

Определим квадратную матрицу $\mathbf{S}_{\text{н}}$ из векторов $\mathbf{s}_{\text{н}1}$, $\mathbf{s}_{\text{н}2}$ и $\mathbf{s}_{\text{н}3}$ в соответствии с
\begin{equation*}
    \mathbf{S}_{\text{н}} = \left(\mathbf{s}_{\text{н}1}^\text{т}, \mathbf{s}_{\text{н}2}^\text{т}, \mathbf{s}_{\text{н}3}^\text{т}\right).
\end{equation*}
Также определим квадратную матрицу $\mathbf{S}_{\text{св}}$ из векторов (\ref{eq:vecs}):
\begin{equation*}
    \mathbf{S}_{\text{св}} = \left(\mathbf{s}_{\text{св}1}^\text{т}, \mathbf{s}_{\text{св}2}^\text{т}, \mathbf{s}_{\text{св}3}^\text{т}\right).
\end{equation*}
В таком случае, квадратные матрицы $\mathbf{S}_{\text{н}}$ и $\mathbf{S}_{\text{св}}$ связаны соотношением
\begin{equation}\label{eq:matrix}
    \mathbf{S}_{\text{н}} = \mathbf{\xi}_{\text{св}}^{\text{н}} \times \mathbf{S}_{\text{св}},
\end{equation}
где $\mathbf{\xi}_{\text{св}}^{\text{н}}$ "--- квадратная матрица вращения при переходе от связной системы координат к нормальной земной системе координат. Таким образом, из (\ref{eq:matrix}) получим явное выражение для матрицы $\mathbf{\xi}_{\text{св}}^{\text{н}}$:
\begin{equation*}
    \mathbf{\xi}_{\text{св}}^{\text{н}} = \mathbf{S}_{\text{н}} \times \mathbf{S}_{\text{св}}^{-1},
\end{equation*}
где $\mathbf{S}_{\text{св}}^{-1}$ "--- обратная к $\mathbf{S}_{\text{св}}$ матрица. Углы курса $\psi$, крена $\mu$ и тангажа $\vartheta$ находятся из матрицы $\mathbf{\xi}_{\text{св}}^{\text{н}}$ стандартным образом.

\subsubsection{Математические о}
Несмотря на внешнюю простоту системы \eqref{eq:system}, она имеет массу особенностей, описание всех ее решений для разных случаев расположения источников и воздушного объекта требует обширного математического исследования. С прикладной точки зрения более практичным выглядит следующий подход: использовать приближенный метод решения, а при его разработке учитывать структуру возможных решений, полученную с помощью качественных рассуждений. Без понимания такой структуры гораздо сложнее строить эффективные методы, позволяющие контролировать погрешности промежуточных вычислений.

Систему \eqref{eq:system} можно свести, вообще говоря, к уравнению четвёртой степени относительно одной переменной. Прежде чем показать, как это делается, опишем некоторые свойства решений исходной системы.

Во-первых, если $\left(\ell_1, \ell_2, \ell_3\right)$ --- решение, то $\left(-\ell_1, -\ell_2, -\ell_3\right)$ "--- тоже решение.  Кроме того, хотя решение с отрицательными значениями ребер не физично, оно  оказывается связанным с аналогичной задачей с теми же точками $M_1$, $M_2$, $M_3$, но с набором углов, дающих те же значения по модулю косинусов. Например, если $\left(-\ell_1, \ell_2, \ell_3\right)$ -- одно из решений для набора углов $\left(\alpha_{12}, \alpha_{13}, \alpha_{23}\right)$, то $\left(\ell_1, \ell_2, \ell_3\right)$ -- для $\left(\pi-\alpha_{12}, \pi-\alpha_{13}, \alpha_{23}\right)$.

Во-вторых, если считать все возможные решения, в том числе и с нулевыми и отрицательными рёбрами, то их получится не больше восьми. С учетом описанной выше симметрии число положительных решений не больше четырёх.

В-третьих, каждое конкретное решение $\left(\ell_1, \ell_2, \ell_3\right)$ непрерывно зависит от набора углов $\left(\alpha_{12}, \alpha_{13}, \alpha_{23}\right)$.

\begin{figure}[hbp]
  \centering

  \includegraphics[width=0.8\textwidth]{3/tetrahedron/pic2}

  \caption{Внешний вид <<закрытого>> тора, образованного вращением окружности вокруг ее хорды.}
  \label{fig:tetrahedron:pic2}
\end{figure}

В-четвёртых, систему \eqref{eq:system} можно интерпретировать следующим образом. В каждой из боковых гранях нам известна сторона и угол лежащий напротив неё. Следовательно, нам известен радиус описанной окружности около боковой грани. Обозначим длину этой стороны через $d$, а радиус окружности описанной около грани через $R_d$. Искомая точка может располагаться в  зависимости от угла (острый или тупой)  на меньшей или большей дуге этой окружности соответственно. Эта окружность не единственная и образует семейство. Его можно описать как множество точек равноудаленных на расстояние $R_d$ от окружности с центром в середине стороны и радиусом равным $\sqrt{R_d^2 - \rfrac{d^2}{4}}$. Эта конструкция похожа на тор, но её отличие в том, что  радиус окружности меньше расстояния, на которое равноудалены точки. Это <<тор>> без дырки, <<тор>> с внешними границами, оказавшимися внутри (рис. \ref{fig:tetrahedron:pic2}).

В-пятых, исходя из геометрии задачи  можно утверждать, что если каждый угол боковой грани больше соответствующего ему угла треугольника основания (углы должны опираться на одну и ту же сторону), решение единственно и его проекция на плоскость $M_1 M_2 M_3$ находится внутри треугольника $M_1 M_2 M_3$. В тот момент, когда плоский угол при вершине пирамиды совпадает с соответствующем ему углом треугольника основания, возникает решение с нулевым ребром.  Далее, при уменьшении плоского угла (например, при вертикальном подъеме воздушного объекта) в силу непрерывности оно переходит в решение со всеми положительными боковыми ребрами.

Трудности и особенности, относящиеся к получению аналитического решения,  будут продемонстрированы на основе частного случая, когда в основании находится равносторонний треугольник. Заметим, что используемые при этом приемы в целом применимы и для общего случая. Система \eqref{eq:system} примет следующий вид:

\begin{equation}
  \begin{cases}
    \ell_1^2+\ell_2^2-2 \ell_1 \ell_2 \cos \alpha_{12} = a^2 \\
    \ell_1^2+\ell_3^2-2 \ell_1 \ell_3 \cos \alpha_{13} = a^2 \\
    \ell_2^2+\ell_3^2-2 \ell_2 \ell_3 \cos \alpha_{23} = a^2
  \end{cases}
  \label{ell_1}
\end{equation}
Если в этой системе сделать замену $\ell_2= t \ell_1$, $\ell_3= s \ell_1$ (случай $\ell_1=0$ следует рассмотреть отдельно), то перейдём к системе
\begin {equation}
  \begin{cases}
    \ell_1^2(1-2 t \cos \alpha_{12} + t^2) = a^2 \\
    \ell_1^2(1-2 s \cos \alpha_{13} + s^2) = a^2 \\
    \ell_1^2(s^2-2 s t \cos \alpha_{23} + t^2) = a^2
  \end{cases}
  \label {ell_2}
\end{equation}
Теперь разделим первое уравнение на второе. Также из суммы первого и второго вычтем третье и результат поделим на первое. Получим,
\begin{equation}
\left\{
\begin {matrix}
t^2-2 t \cos \alpha_{12}  =  s^2-2 s \cos \alpha_{13}  \\
2 s (t \cos \alpha_{23} - \cos \alpha_{13} ) = t^2-1.
\end {matrix}\right.
\label{ell_3}
\end{equation}
Из второго уравнения выразим $s$ и подставим в первое (случай $t=\pm 1$ требует отдельного рассмотрения):

\begin {equation}
\begin {array}{l}
(1-4 \cos^2 \alpha_{23}) t^4 + 4 \cos \alpha_{23} (\cos \alpha_{23} + 2 \cos \alpha_{12} \cos \alpha_{23} ) t^3 - \\
- 2 (1 + 8 \cos \alpha_{12} \cos \alpha_{13} \cos \alpha_{23}) t^2 +
\\+ 4 \cos \alpha_{13} (\cos \alpha_{23} + 2 \cos \alpha_{12} \cos \alpha_{13}) t + 1-4 \cos^2 \alpha_{13} =0
\end {array}
\label{ell_eq_4}
\end {equation}
Для данного уравнения значение $\cos \alpha_{23}= \frac{1}{2}$ является особым. При таких значениях угла $\alpha_{23}$ уравнение 4-ой степени превращается в уравнение 3-ей степени. Кроме того, значение корня $t=\pm 1$ приводит к частным случаям. Эти обстоятельства не позволяют, вообще говоря, использовать уравнение \eqref{ell_eq_4}   для численного решения исходной системы.

\section {Решение системы методом Ньютона}

Вернемся к системе уравнений \eqref{eq:system}. Его можно свести к одному уравнению четвертого порядка, но с практической точки зрения это не самое оптимальное решение.
Более удобным представляется применение приближенного метода решения "--- метода Ньютона для нелинейных систем. Мы исходим из того предположения, что начальные условия для применения метода Ньютона нам известны, поскольку мы совершаем постоянный мониторинг воздушного объекта. Решение системы, вообще говоря, не единственно. Но паразитные решения в рабочем режиме полета достаточно удалены от реального. Поэтому метод Ньютона сходится именно к интересующему нас решению. Для получения требуемой точности достаточно сделать 5-6 итераций, причем обратная матрица выписывается явно и нет никаких других операций над числами, кроме арифметических (значения $\cos\alpha_{ij}$
вычисляются один раз после измерения данных углов).
Выпишем систему \eqref {ell_123} в векторной форме
\begin{equation}
  \mathbf{F} (\mathbf{L})=
  \left(
    \begin{array}{c}
      \ell_1^2+\ell_2^2-2 \ell_1 \ell_2 \cos \alpha_{12} - d_{12}^2 \\
      \ell_1^2+\ell_3^2-2 \ell_1 \ell_3 \cos \alpha_{13} - d_{13}^2 \\
      \ell_2^2+\ell_3^2-2 \ell_2 \ell_3 \cos \alpha_{23} - d_{23}^2
    \end{array}
  \right)
  =\mathbf{0}.
\label{ell_123_vector}
\end{equation}
Здесь $\mathbf{L}=(\ell_1,\ell_2,\ell_3)$, а координаты вектора $\mathbf{F}=(F_1,F_2,F_3)$ задаются в соответствии с \eqref {ell_123_vector}. Обозначим через  $DF$ матрицу из частных производных $\frac {\partial F_i} {\partial \ell_j}$,
\begin{equation}
DF=
\left(
  \begin{array}{ccc}
    2\ell_1 - 2\ell_2 \cos \alpha_{12} & 2\ell_2 - 2\ell_1 \cos \alpha_{12} & 0  \\
    2\ell_1 - 2\ell_3 \cos \alpha_{13} & 0 & 2\ell_3 - 2\ell_1 \cos \alpha_{13}  \\
    0 & 2\ell_2 - 2\ell_3 \cos \alpha_{23} & 2\ell_3 - 2\ell_2 \cos \alpha_{23}  \\
  \end{array}
\right),
\label {DF}
\end{equation}
а через $\mathbf{L}^m$ -- $m$-ую итерацию метода Ньютона. Тогда
\begin{equation}
  \mathbf{L}^{m+1}=\mathbf{L}^{m} - (DF)^{-1} \mathbf{F}(\mathbf{L}^{m}).
\label {Newton_vector}
\end{equation}

Обратная матрица $(DF)^{-1}$ легко выписывается в явном виде. Введем следующие обозначения:
\begin {equation}
  \begin {matrix}
   a= 2\ell_1 - 2\ell_2 \cos \alpha_{12}, \; b= 2\ell_2 - 2\ell_1 \cos \alpha_{12}, \\
   c= 2\ell_1 - 2\ell_3 \cos \alpha_{13}, \; d= 2\ell_3 - 2\ell_1 \cos \alpha_{13}, \\
   e= 2\ell_2 - 2\ell_3 \cos \alpha_{23}, \; f= 2\ell_3 - 2\ell_2 \cos \alpha_{23}. \\
  \end {matrix}
  \label {abc}
\end {equation}
Тогда справедлива формула
\begin{equation}
  (DF)^{-1} = \frac {-1} {ade+bcf}
  \left(
    \begin{array}{ccc}
      -de & -bf & bd \\
      -cf &  af & -ad \\
      ce  &\ -ae & -bc
    \end{array}
  \right) .
\label {DF_m1}
\end{equation}

Если рассмотреть процедуру вертикального подъема воздушного объекта из точки, расположенной внутри треугольника $M_1 M_2 M_3$, то выстраивается следующая картина. На малых высотах решение системы единственно, далее появляется второе решение в одной из вершин треугольника и также движется вверх. Проекция полученного решения на плоскость треугольника начинает удаляться от вершины наружу. При достижении следующей критической высоты появляется второе паразитное решение, оно также движется вверх и проекция удаляется от треугольника. Затем появляется и третье паразитное решение. При определенных симметриях паразитные решения могут появляться парой или даже тройкой. Это зависит от конфигурации треугольника, лежащего в основании, и траектории подъема воздушного объекта.

В принципе, допустим случай, когда проекция воздушного объекта изначально выходит за границы треугольника. Структура паразитных решений, соответственно, также перестраивается. Здесь сложно ответить на вопрос, возможно ли полное описание для всех возможных случаев. Гораздо проще провести расчеты по описанной выше методике для заданной конфигурации источников и возможных траекториях подъема и маневрирования воздушного объекта. Сюда же следует отнести и анализ погрешностей, который несложно осуществить стандартными средствами. Дело в том, что у нас есть возможность для любого набора входных данных найти единственное решение и рассчитать среднеквадратичные отклонения, генерируя погрешности по заданной статистической закономерности.


%
%
% TODO: Привести в соответствие с окончательным вариантом второй статьи
%
%
\subsection{Вариации локальных угломерных навигационных систем}
В данном разделе предлагаются другие варианты ЛУНС, в которых за счет усложнения структуры системы упрощается процедура расчетов. Все эти варианты предполагают обмен измеряемыми данными между подвижными объектами и радиоориентирами, поэтому описанный выше вариант обладает тем преимуществом, что подвижный объект может работать полностью в пассивном режиме, только принимая сигналы. Далее будет показано, что получающиеся системы уравнений для рассмотренных модификаций ЛУНС оказываются существенно проще, чем \eqref{eq:system}. Собираемая информация оказывается, как правило, избыточной, что требует специального анализа в рамках используемого в работе детерминированного подхода. В некоторых случаях разбиение задачи на три этапа также оказывается избыточным, можно одновременно определять расстояния и декартовы координаты. В ра-боте ставится задача рассмотрения не всех возможных конфигураций, а только минимально возможных. Например, если предположить использование на объектах воздушного базирования высотомеров, то можно ограничиться одним наземным пунктом управления (НПУ) и двумя воздушными объектами. Наличие нескольких модификаций позволяет также осуществлять перестройку системы, что повышает ее гибкость и надежность.

\subsubsection{Локальная угломерная навигационная система с НПУ}
Ранее рассматривалась простейшая локальная угломерная навигационная система (ЛУНС), состоящая из трех наземных радиоориентиров и воздушного объекта, оснащенного радиолокационным оборудованием, которое способно определять азимут и угол места каждого из наземных радиоориентиров. Авторами было показано, что определение пространственных координат и угловой ориентации воздушного объекта сводится к решению нелинейной системы уравнений следующего вида:
\begin{equation}\label{eq:luns_start}
  \begin{cases}
    \ell_1^2 + \ell_2^2 - 2 \ell_1 \ell_2 \cos \alpha_{12} = d_{12}^2 \\
    \ell_1^2 + \ell_3^2 - 2 \ell_1 \ell_3 \cos \alpha_{13} = d_{13}^2 \\
    \ell_2^2 + \ell_3^2 - 2 \ell_2 \ell_3 \cos \alpha_{23} = d_{23}^2 \\
  \end{cases},
\end{equation}
где $\ell_i$ "--- это расстояние от воздушного объекта до $i$-го радиоориентира, $d_{i,j}$ "--- расстояние между $i$-м и $j$-м наземным радиоориетиром, $\cos \alpha_{i,j}$ "--- плоский угол, образованный наземными ориентирами $i$ и $j$ и воздушным объектом. Из этой системы далее выводятся координаты и угловая ориентация воздушного объекта.

Система \eqref{eq:luns_start} может иметь несколько решений, что ведет к неоднозначности определения координат и угловой ориентации летательного аппарата. Поэтому, в статье~\cite{antennas} приводятся методики для расчета областей, где решение можно определить однозначно. Более того, решение системы \eqref{eq:luns_start} производится численно, что потенциально может привести к дополнительным ошибкам в определении пространственных параметров воздушного объекта. В связи с этим, далее предлагается еще один вариант ЛУНС, которая лишена вышеперечисленных недостатков ценой усложнения структуры системы.

Предлагается заменить один из пассивных радиоориентиров на наземный пункт управления (НПУ), оснащенный радиоориентиром. Требуется, чтобы НПУ мог определять азимут и угол места двух других радиоориентиров и воздушного объекта.

По аналогии с~\cite{antennas}, будем считать, что воздушный объект находится в точке $M_0\left(x, y, z\right)$, НПУ "--- в $M_1\left(x_1, y_1, z_1\right)$, а $i$-й РО "--- в точке $M_i\left(x_i, y_i, z_i \right)$, $i = 1,2$. Азимут и угол места радиоориентира $M_i$, полученные в результате радиопеленгования, обозначим через $\theta_i$ и $\beta_i$ соответственно (см. рис.~\ref{fig:systems:pic1}). Предполагается, что измерения этих углов проводятся в локальной системе координат НПУ, центр которой совпадает с координатами НПУ в НЗСК, а направления осей совпадают с НЗСК. В этой системе координат можно определить три единичных вектора направлений на радиоориентиры $M_0$, $M_2$ и $M_3$:
\begin{equation*}
  \mathbf{s}_{\text{лн}i} = \left(\cos\theta_i \cos\beta_i, \sin\theta_i\cos\beta_i, \sin\beta_i\right),
\end{equation*}
где $i = 0, 2, 3$. Тогда, косинусы углов $\alpha_{02} = \angle M_0 M_1 M_2$ и $\alpha_{03} = \angle M_0 M_1 M_3$ определяются как
\begin{align*}
  \cos\alpha_{02} = \left(\mathbf{s}_{\text{лн}0}, \mathbf{s}_{\text{лн}2}\right) =
  \cos\beta_0 \cos\beta_2 \cos\left(\theta_0 - \theta_2\right) + \sin\beta_0 \sin\beta_2, \\
  \cos\alpha_{03} = \left(\mathbf{s}_{\text{лн}0}, \mathbf{s}_{\text{лн}3}\right) =
  \cos\beta_0 \cos\beta_3 \cos\left(\theta_0 - \theta_3\right) + \sin\beta_0 \sin\beta_3.
\end{align*}

\begin{figure}[htbp]
  \centering

  \fbox{\includegraphics[width=0.8\columnwidth]{3/systems/pic1}}

  \caption{Схемы размещения на местности БпЛА, НПУ и РО для реализации ЛУНС}
  \label{fig:systems:pic1}
\end{figure}

По теореме синусов для треугольника $M_1 M_0 M_2$:
\begin{equation}\label{eq:luns_sin_1}
  \frac{\ell_2}{\sin\alpha_{02}} = \frac{d_{12}}{\sin\alpha_{12}} = \frac{\ell_1}{\sin\left(\alpha_{12} + \alpha_{02}\right)}.
\end{equation}
Аналогично для треугольника $M_1 M_0 M_3$:
\begin{equation}\label{eq:luns_sin_2}
    \frac{\ell_3}{\sin\alpha_{03}} = \frac{d_{13}}{\sin\alpha_{13}} = \frac{\ell_1}{\sin\left(\alpha_{13} + \alpha_{03}\right)}.
\end{equation}
Таким образом, из~(\ref{eq:luns_sin_1}):
\begin{equation*}
    \ell_2 = \frac{d_{12}\sin\alpha_{02}}{\sin\alpha_{12}}.
\end{equation*}
В то же время, из~(\ref{eq:luns_sin_2}):
\begin{equation*}
    \ell_3 = \frac{d_{13}\sin\alpha_{03}}{\sin\alpha_{13}}.
\end{equation*}
Длина $\ell_1$ может быть найдена из любого уравнения:
\begin{equation*}
    \ell_1 = \frac{d_{12}\sin\left(\alpha_{12} + \alpha_{02}\right)}{\sin\alpha_{12}} = \frac{d_{13}\sin\left(\alpha_{13} + \alpha_{03}\right)}{\sin\alpha_{13}}.
\end{equation*}
Далее пространственные координаты и угловая ориентация БПА определяется согласно~\cite{antennas}.

\subsubsection{Локальная система с уменьшенным количеством наземных РО}
В работе~\cite{antennas} было показано, что минимально возможное число наземных радиоориентиров, которое позволяет однозначно определять координаты и угловую ориентацию воздушного объекта, равно трем. Однако, это число можно уменьшить, если добавить в систему еще один подвижный объект с радиоориентиром и бортовой пеленгационной антенной.

Пусть подвижные объекты находятся в точках $M_1\left(x_1, y_1, z_1\right)$ и $M_2\left(x_2, y_2, z_2\right)$, а наземные ориентиры находятся в точках $M_3\left(x_3, y_3, z_3\right)$ и $M_4\left(x_4, y_4, z_4\right)$, а расстояние между ними равно $d_{34}$. В результате азимутально-угломестной радиопеленгации  $j$-го радиоориентира с борта $i$-го подвижного объекта ($i \ne j$), определяются углы азимута ($\alpha_{ij}$) и угла места ($\varepsilon_{ij}$) (см Рис.~\ref{fig:systems:pic2}). Пусть также $\ell_ij$ "--- расстояние между $i$-м и $j$-м радиоориентиром.

\begin{figure}[htbp]
    \begin{center}

    \fbox{\includegraphics[width=0.8\columnwidth]{3/systems/pic2}}

    \caption{Схемы размещения на местности БпЛА, НПУ и РО для реализации ЛУНС}
    \label{fig:systems:pic2}
    \end{center}
\end{figure}

Обозначим через $\mathbf{s}_{ij}$ единичные векторы, которые определяются следующим образом:
\begin{equation}
    \mathbf{s}_{ij} = \left(\cos\alpha_{ij} \cos\varepsilon_{ij}, \cos\alpha_{ij} \sin\varepsilon_{ij}, \sin\varepsilon_{ij}\right),
\end{equation}
где $i \ne j$, $i = 1,2$, $j = 1,2,3,4$. Также обозначим через $\gamma_{ij}$ углы, образованные воздушным объектом $M_1$ и радиоориентирами $i$ и $j$ ($i \ne j$), а через $\varphi_{ij}$ "--- углы, образованные воздушным объектом $M_2$ и теми же радиоориентирами. В таком случае, косинусы этих углов находятся по формулам:
\begin{align*}
    \cos\gamma_{ij} &= \left(\mathbf{s}_{1i}, \mathbf{s}_{1j}\right), i \ne j \ne 1, \\
    \cos\varphi_{ij} &= \left(\mathbf{s}_{2i}, \mathbf{s}_{2j}\right), i \ne j \ne 2.
\end{align*}

По теореме синусов для треугольника $M_1 M_2 M_3$:
\begin{equation}\label{eq:luns_2_sin_1}
    \frac{\ell_{12}}{\sin{M_1 M_3 M_2}} = \frac{\ell_{13}}{\sin{\varphi_{13}}} = \frac{\ell_{23}}{\sin{\gamma_{13}}}.
\end{equation}
Для треугольника $M_1 M_2 M_4$:
\begin{equation}\label{eq:luns_2_sin_2}
    \frac{\ell_{12}}{\sin{M_1 M_4 M_2}} = \frac{\ell_{14}}{\sin{\varphi_{14}}} = \frac{\ell_{24}}{\sin{\gamma_{24}}}.
\end{equation}
Из (\ref{eq:luns_2_sin_1}) и (\ref{eq:luns_2_sin_2}) получим отношения:
\begin{align}
     \frac{\ell_{23}}{\ell_{12}} = \frac{\sin{\gamma_{13}}}{\sin{M_1 M_3 M_2}}, \frac{\ell_{24}}{\ell_{12}} = \frac{\sin{\gamma_{14}}}{\sin{M_1 M_4 M_2}} \label{eq:luns_2_triag_1}\\
     \frac{\ell_{13}}{\ell_{12}} = \frac{\sin{\varphi_{13}}}{\sin{M_1 M_3 M_2}}, \frac{\ell_{14}}{\ell_{12}} = \frac{\sin{\varphi_{14}}}{\sin{M_1 M_4 M_2}} \label{eq:luns_2_triag_2}.
\end{align}
По теореме косинусов для треугольника $M_2 M_4 M_3$:
\begin{equation}\label{eq:luns_2_cos_1}
    \ell_{23}^2 + \ell_{24}^2 - 2 \ell_{23} \ell_{24} \cos\varphi_{34} = \ell_{34}^2.
\end{equation}
С учетом (\ref{eq:luns_2_triag_1}), (\ref{eq:luns_2_cos_1}) можно переписать в виде:
\begin{equation}
    \ell_{12}^2 k_1^2 + \ell_{12}^2 k_2^2 - 2 \ell_{12} \ell_{12} k_1 k_2 \cos\varphi_{34} = \ell_{34}^2,
\end{equation}
где $k_1 = \sin\gamma_{13} / \sin{M_1 M_3 M_2}$, а $k_2 = \sin\gamma_{14} / \sin{M_4 M_1 M_2}$.
Тогда $\ell_{12}$ выражается следующим образом:
\begin{equation}\label{eq:luns_2_ell_12}
    \ell_{12} = \frac{\ell_{34}}{\sqrt{k_1^2 + k_2 ^2 - 2 k_1 k_2 \cos\varphi_{34}}}.
\end{equation}
С учетом (\ref{eq:luns_2_ell_12}), из (\ref{eq:luns_2_triag_1}) и (\ref{eq:luns_2_triag_2}) находятся расстояния $\ell_{13}$, $\ell_{14}$, $\ell_{23}$ и $\ell_{24}$.

Для нахождения координат воздушного объекта $M_i$, введем радиус-вектор $\mathbf{r}_{i3}$, который определяется следующим образом:
\begin{equation}
    \mathbf{r}_{i3} = \mathbf{s}_{i3} \ell_{i3} = M_3 - M_i.
\end{equation}
Отсюда, координаты точки $M_i$ в НЗСК находятся из системы уравнений:
\begin{equation}
    \begin{cases}
        x_i = x_3 - \ell_{i3} \cos\alpha_{i3} \cos\varepsilon_{i3}, \\
        y_i = y_3 - \ell_{i3} \cos\alpha_{i3} \sin\varepsilon_{i3}, \\
        z_i = y_3 - \ell_{i3} \sin\varepsilon_{i3}.
    \end{cases}
\end{equation}
Следует заметить, что аналогичные рассуждения справедливы и относительно радиоориентира $M_4$. Совокупность расчетов координат относительно обоих наземных радиоориентиров позволяет ввести дополнительный контроль ошибок.

Отметим, что работоспособность такой конфигурации системы достигается только в случае, когда все четыре точки не лежат в одной плоскости. Это реализуется при расположении подвижных аппаратов $M_1$ и $M_2$ по разные стороны от прямой, образованной наземными радиоориентирами $M_3$ и $M_4$ (см. рис.~\ref{fig:systems:pic2}).

Матрица поворота и угловая ориентация каждого из объектов находится по алгоритму, представленному в работе~\cite{antennas}.

\subsubsection{Случай автономной системы}
Автономная угломерная радионавигационная система (АУНС) предназначена для определения координат и угловой ориентации в пространстве двух воздушных объектов, оснащенных высотомерами и бортовыми радиоориентирами с наземного пункта управления (НПУ), оснащенного радиоориентиром.
% Схема размещения на местности воздушных объектов и наземного пункта управления приведены на рис.~\ref{figure:pic3}.

Пусть радиоориентир НПУ расположен в точке $M_0$ с заранее известными координатами $M_0\left(x_0, y_0, z_0\right)$ в Балтийской системе координат (БСК), а воздушные объекты "--- в точках $M_1$ и $M_2$ с координатами $M_i\left(x_i, y_i, z_i\right)$, при этом $z_i$ совпадает с измерениями высотомера $h_i$. НПУ $M_0$ способен измерять радиопеленг $\theta_i$ и угол возвышения $\beta_i$ $i$-го воздушного объекта в БСК. Воздушные объект $M_n$ способен измерять азимут $\alpha_{ni}$ и угол места $\varepsilon_{ni}$ $i$-го радиоориентира (наземного или воздушного) в связанной системе координат БПА \cite{antennas}. Схема размещения с указанными величинами указана на рис.~\ref{fig:systems:pic3}. Пространственное положение радиоориентиров в БСК также можно охарактеризовать радиус-векторами $\mathbf{r}_i = \left(x_i, y_i, z_i\right)$, где $i = 1,2,3$.

\begin{figure}[htbp]
    \begin{center}

    \fbox{\includegraphics[width=0.8\columnwidth]{3/systems/pic3}}

    \caption{Схемы размещения на местности БпЛА, НПУ и РО для реализации АУНС}
    \label{fig:systems:pic3}
    \end{center}
\end{figure}

При детерминированном подходе для такой системы возможно однозначно определить координаты и угловую ориентацию воздушных объектов. Для этого нужно выполнить следующие ключевые шаги:
\begin{enumerate}
    \item Определить координаты воздушных объектов в БСК.
    \item Определить координаты радиоориентиров в связанной системе координат воздушного объекта $M_1$ ($\Sigma_{\text{св}1}$).
    \item Составить матрицу поворота системы координат $\Sigma_{\text{св}1}$ по алгоритму, представленному ниже.
    \item Определить углы поворота $\Sigma_{\text{св}1}$ по алгоритму, представленному в \cite{antennas}.
    \item Повторить предыдущие шаги для воздушного ориентира $M_2$.
\end{enumerate}

Первая часть алгоритма реализуется явно "--- совокупность данных с высотомеров воздушных объектов и углов $\theta_i$, $\beta_i$ позволяют определить координаты летательных аппаратов однозначно. Таким образом, координаты радиоориентира $M_1$ и $M_2$ определяются следующим отношениями:
\begin{equation}
    \begin{cases}
        x_1 = \rho_1 \cos\theta_1 \cos\beta_1 \\
        y_1 = \rho_1 \sin\theta_1 \cos\beta_1 \\
        z_1 = h_1 = \rho_1 \sin\beta_1 \\
        \rho_1 = ~^{z_1}/_{\sin\beta_1}
    \end{cases},
    \begin{cases}
        x_2 = \rho_2 \cos\theta_2 \cos\beta_2 \\
        y_2 = \rho_2 \sin\theta_2 \cos\beta_2 \\
        z_2 = h_2 = \rho_2 \sin\beta_2 \\
        \rho_2 = ~^{z_2}/_{\sin\beta_2}
    \end{cases}.
\end{equation}

Далее необходимо определить координаты радиоориентиров $M_0$ и $M_2$ в связанной системе координат воздушного объекта $M_1$:
\begin{equation}
    \begin{cases}
        x'_0 = \rho_{10} \cos\alpha_{10} \cos\varepsilon_{10} \\
        y'_0 = \rho_{10} \sin\alpha_{10} \cos\varepsilon_{10} \\
        z'_0 = z_1 - z_0 = \rho_{10} \sin\varepsilon_{10} \\
        \rho_{10} = |\mathbf{r}_1 - \mathbf{r}_0|
    \end{cases},
    \begin{cases}
        x'_2 = \rho_{12} \cos\alpha_{12} \cos\varepsilon_{12} \\
        y'_2 = \rho_{12} \sin\alpha_{12} \cos\varepsilon_{12} \\
        z'_2 = z_1 - z_2 = \rho_{12} \sin\varepsilon_{12} \\
        \rho_{12} = |\mathbf{r}_1 - \mathbf{r}_2|
    \end{cases},
\end{equation}
где $M_0'\left(x'_0, y'_0, z'_0\right)$ и $M_2'\left(x'_2, y'_2, z'_2\right)$
"--- координаты $M_0$ и $M_2$ в связанной СК $M_1$.

Для определения матрицы поворота системы координат $\Sigma_{\text{св}1}$, необходимо сначала ввести радиус-вектора $\mathbf{r}'_0 = \left(x'_0, y'_0, z'_0\right)$ и $\mathbf{r}'_2 = \left(x'_2, y'_2, z'_2\right)$, которые определяют положения радиоориентиров $M_0$ и $M_2$ в связанной системе координат $M_1$. Далее, зададим единичные векторы $\mathbf{s}'_1$, $\mathbf{s}'_2$ и $\mathbf{s}'_3$ следующим образом:
\begin{equation}\label{eq:vec_1_local}
    \begin{split}
    \mathbf{s}'_1 = \left(s'_{1x}, s'_{1y}, s'_{1z}\right) = &\left(\cos\alpha_{10} \cos\varepsilon_{10}, \sin\alpha_{10}\cos\varepsilon_{10}, \sin\varepsilon_{10}\right),\\
    \mathbf{s}'_2 = \left(s'_{2x}, s'_{2y}, s'_{2z}\right) = &\left(\cos\alpha_{12} \cos\varepsilon_{12}, \sin\alpha_{12}\cos\varepsilon_{12}, \sin\varepsilon_{12}\right),\\
    \mathbf{s}'_3 = \mathbf{s}'_1 \times \mathbf{s}'_2 = \left(s'_{3x}, s'_{3y}, s'_{3z}\right) = &\left(\sin\alpha_{12}\cos\varepsilon_{12}\sin\varepsilon_{10} - \sin\alpha_{12}\sin\alpha_{10}\cos\varepsilon_{10},\right.\\
    &\ \sin\alpha_{10}\cos\varepsilon_{10}\sin\varepsilon_{12} - \cos\alpha_{10}\cos\varepsilon_{12}\sin\varepsilon_{10},\\
    &\ \left.\sin\left(\alpha_{10} - \alpha_{12}\right)\cos\varepsilon_{10}\cos\varepsilon_{12}\right).
    \end{split}
\end{equation}
Те же вектора в балтийской системе координат:
\begin{equation}\label{eq:vec_1_bsk}
    \begin{split}
        \mathbf{s}_{01} &= \left(s_{01x}, s_{01y}, s_{01z}\right) = \frac{\mathbf{r}_0 - \mathbf{r}_1}{|\mathbf{r}_0 - \mathbf{r}_1|},\\
        \mathbf{s}_{21} &= \left(s_{21x}, s_{21y}, s_{21z}\right) = \frac{\mathbf{r}_2 - \mathbf{r}_1}{|\mathbf{r}_2 - \mathbf{r}_1|},\\
        \mathbf{n}_1 &= \left(n_{1x}, n_{1y}, n_{1z}\right) = \mathbf{s}_{01} \times \mathbf{s}_{21}.\\
    \end{split}
\end{equation}
Определим квадратную матрицу $\mathbf{S}$ размера $3 \times 3$ координат трех полученных по формуле (\ref{eq:vec_1_bsk}) единичных векторов $\mathbf{s}_{01}$, $\mathbf{s}_{21}$ и $\mathbf{n}_{1}$, записав в столбцы, в соответствии с отношением:
\begin{equation}\label{eq:vec_1_bsk_matrix}
    \mathbf{S}_1 =
    \left(
        \begin{matrix}
            s_{01x} & s_{01y} & s_{01z} \\
            s_{21x} & s_{21y} & s_{21z} \\
            n_{1x} & n_{1y} & n_{1z}
        \end{matrix}
    \right).
\end{equation}
По аналогии с (\ref{eq:vec_1_bsk_matrix}) определим квадратную матрицу $\mathbf{S}'$ размера $3 \times 3$ координат трех полученных по формуле (\ref{eq:vec_1_local}):
\begin{equation}\label{eq:vec_1_local_matrix}
    \mathbf{S}' =
    \left(
        \begin{matrix}
            s'_{1x} & s'_{1y} & s'_{1z} \\
            s'_{2x} & s'_{2y} & s'_{2z} \\
            s'_{3x} & s'_{3y} & s'_{3z}
        \end{matrix}
    \right).
\end{equation}
Отсюда получим следующее преобразование координат:
\begin{equation*}
    \mathbf{R_1} \times \mathbf{S}_1 = \mathbf{S}'
\end{equation*}

В таком случае, матрицу поворота связанной системы координат $\Sigma_{\text{св}1}$ можно найти следующим образом:
\begin{equation}
    \mathbf{R_1} = \mathbf{S}' \times \mathbf{S}_1^{-1}
\end{equation}

По аналогии можно получить матрицу поворота $\mathbf{R}_2$ связанной системы координат $M_2$. Вводятся радиус-вектора $\mathbf{r}''_0 = \left(x''_0, y''_0, z''_0\right)$,
$\mathbf{r}''_1 = \left(x''_1, y''_1, z''_1\right)$, по ним же определяются единичные вектора
$\mathbf{s}''_1$, $\mathbf{s}''_2$ и $\mathbf{s}''_3$:
\begin{equation}\label{eq:vec_2_local}
    \begin{split}
    \mathbf{s}''_1 = \left(s''_{1x}, s''_{1y}, s''_{1z}\right) =  &\left(\cos\alpha_{20} \cos\varepsilon_{20}, \sin\alpha_{20}\cos\varepsilon_{20}, \sin\varepsilon_{20}\right),\\
    \mathbf{s}''_2 = \left(s''_{2x}, s''_{2y}, s''_{2z}\right) = &\left(\cos\alpha_{21} \cos\varepsilon_{21}, \sin\alpha_{21}\cos\varepsilon_{21}, \sin\varepsilon_{21}\right),\\
    \mathbf{s}''_3 = \mathbf{s}''_1 \times \mathbf{s}''_2 = \left(s''_{3x}, s''_{3y}, s''_{3z}\right) =  &\left(\sin\alpha_{21}\cos\varepsilon_{21}\sin\varepsilon_{20} - \sin\alpha_{21}\sin\alpha_{20}\cos\varepsilon_{20},\right.\\
    &\ \sin\alpha_{20}\cos\varepsilon_{20}\sin\varepsilon_{21} - \cos\alpha_{20}\cos\varepsilon_{21}\sin\varepsilon_{20},\\
    &\ \left.\sin\left(\alpha_{20} - \alpha_{21}\right)\cos\varepsilon_{20}\cos\varepsilon_{21}\right).
    \end{split}
\end{equation}
В балтийской системе координат:
\begin{equation}\label{eq:vec_2_bsk}
    \begin{split}
        \mathbf{s}_{02} &= \left(s_{01x}, s_{01y}, s_{01z}\right) = \frac{\mathbf{r}_0 - \mathbf{r}_2}{|\mathbf{r}_0 - \mathbf{r}_1|},\\
        \mathbf{s}_{21} &= \left(s_{21x}, s_{21y}, s_{21z}\right) = \frac{\mathbf{r}_2 - \mathbf{r}_1}{|\mathbf{r}_2 - \mathbf{r}_1|},\\
        \mathbf{n}_1 &= \left(n_{1x}, n_{1y}, n_{1z}\right) = \mathbf{s}_{02} \times \mathbf{s}_{21}.\\
    \end{split}
\end{equation}
Определим квадратные матрицы $\mathbf{S}_2$ и $\mathbf{S}''$ размера $3 \times 3$ по аналогии с (\ref{eq:vec_1_bsk_matrix}) и (\ref{eq:vec_1_local_matrix}):
\begin{equation}\label{eq:vec_1_bsk_matrix}
    \mathbf{S}_2 =
    \left(
        \begin{matrix}
            s_{01x} & s_{01y} & s_{01z} \\
            s_{21x} & s_{21y} & s_{21z} \\
            n_{1x} & n_{1y} & n_{1z}
        \end{matrix}
    \right),
    \mathbf{S}'' =
    \left(
        \begin{matrix}
            s''_{1x} & s''_{1y} & s''_{1z} \\
            s''_{2x} & s''_{2y} & s''_{2z} \\
            s''_{3x} & s''_{3y} & s''_{3z}
        \end{matrix}
    \right).
\end{equation}
Отсюда, матрица поворота определяется следующим образом:
\begin{equation}
    \mathbf{R_2} = \mathbf{S}'' \times \mathbf{S}_2^{-1}
\end{equation}

Углы курса $\psi_1$, $\psi_2$, крена $\mu_1$, $\mu_2$ и тангажа $\vartheta_1$, $\vartheta_2$ находятся
из матриц $\mathbf{R}_1$ и $\mathbf{R}_2$ в соответствии с~\cite{antennas}.

%
%
% TODO: Привести в соответствие с финальной версией статьи
%
\subsection{Влияние подстилающей поверхности на точность измерений}
Рассматривается трехэлементная вертикальная система датчиков с расстоянием $\Delta h$  между соседними элементами. Предполагается, что принимаемый сигнал формируется как суперпозиция двух плоских электромагнитных волн одинаковой частоты. Первая из них, приходящая от регистрируемого источника, имеет амплитуду $E_0$ и фазу на среднем элементе $\varphi _0$. Вторая представляет отраженную от поверхности Земли волну с амплитудой $E_1$ и фазой на среднем элементе $\varphi _1$. Соответствующие комплексные компоненты принимаемого сигнала задаются формулами
\begin{equation*}
  \dot{E}_0=E_0 e^{i\varphi _0}, \; \dot{E}_1=E_1 e^{i\varphi _1}.
\end{equation*}

Пусть $\beta_0$ "--- угол отклонения волны источника от горизонта (считаем, что источник находится выше датчиков), $\beta_1$ "--- угол отклонения отраженной волны от горизонта (отражение происходит ниже датчиков), $\lambda$ "--- длина волны приходящего сигнала. Введем следующие обозначения:
\begin{equation}\label{eq:w_0_w_1}
  \omega_0= \frac{2\pi}{\lambda} \Delta h \sin \beta_0, \;
  \omega_1= \frac{2\pi}{\lambda} \Delta h \sin \beta_1.
\end{equation}
Тогда для комплексных сигналов $\dot{z}_0, \dot{z}_{\hbox{н}}, \dot{z}_{\hbox{в}}$, регистрируемых на среднем, нижнем и верхнем датчиках, соответственно, справедливы следующие соотношения
\begin{equation} \label{eq:0_down_up}
  \begin{cases}
    \dot{z}_0  = \dot{E}_0 + \dot{E}_1 \\
    \dot{z}_\text{н} = \dot{E}_0 e^{-i\omega_0} + \dot{E}_1 e^{i\omega_1} \\
    \dot{z}_\text{в} = \dot{E}_0 e^{i\omega_0} + \dot{E}_1 e^{-i\omega_1}
  \end{cases}
\end{equation}
Требуется по заданным величинам $\dot{z}_0, \dot{z}_{\hbox{н}}, \dot{z}_{\hbox{в}}$ определить угол $\beta_0$.

\begin{figure}[htbp]
  \begin{center}

  \fbox{\includegraphics[width=0.9\columnwidth]{3/surface/pic1}}

  \caption{Геометрия расчетной модели эквидистантной трехэлементной антенной решетки из соосных вертикальных вибраторных антенн}
  \label{figure:surface:pic1}
  \end{center}
\end{figure}

Система \eqref{eq:0_down_up} содержит три комплексных уравнения относительно шести неизвестных вещественных величин $\beta_0$, $\beta_1$, $E_0$, $E_1$, $\varphi_0$, $\varphi_1$ и является нелинейной. Выбор данной модели обусловлен тем, что удается аналитически исключить из этой системы все переменные кроме $\omega_0$, которая включает, согласно \eqref{eq:w_0_w_1}, интересующую нас величину $\beta_0$.

\subsubsection{Решение системы}
Для исключения $\dot{E}_0$ умножим второе уравнение \eqref{eq:0_down_up} на $e^{i\omega_0}$, третье уравнение "--- на $e^{-i\omega_0}$. Тогда
\begin{equation*}
  \begin{cases}
    \dot{z}_0 = \dot{E}_0 + \dot{E}_1 \\
    \dot{z}_\text{н} e^{ i\omega_0} = \dot{E}_0 + \dot{E}_1 e^{ i \left(\omega_1 + \omega_0 \right)} \\
    \dot{z}_\text{в} e^{-i\omega_0} = \dot{E}_0 + \dot{E}_1 e^{-i \left(\omega_1 + \omega_0\right)}
  \end{cases}
\end{equation*}
Вычитая первое уравнение полученной системы из второго и третьего уравнений, приходим к системе двух комплексных уравнений
\begin{equation*}
  \begin{cases}
    \dot{z}_\text{н} e^{ i\omega_0} - \dot{z}_0 = \dot{E}_1 \left( e^{ i\left(\omega_1 + \omega_0 \right)} - 1\right) \\
    \dot{z}_\text{в} e^{-i\omega_0} - \dot{z}_0 = \dot{E}_1 \left( e^{-i\left(\omega_1 + \omega_0 \right)} - 1 \right)
  \end{cases}
\end{equation*}
Для исключения $\dot{E}_1$ домножим второе уравнение на $e^{i \left(\omega_1 + \omega_0 \right)}$ и сложим с первым:
\begin{equation*}
  \dot{z}_\text{н} e^{i\omega_0} - \dot{z}_0 + \left(\dot{z}_\text{в} e^{-i\omega_0} - \dot{z}_0 \right) e^{i \left(\omega_1 + \omega_0 \right)} = 0.
\end{equation*}
Таким образом,
\begin{equation*}
  \dot{z}_\text{н} e^{i\omega_0} - \dot{z}_0 = -\left(\dot{z}_\text{в} e^{-i\omega_0} - \dot{z}_0 \right) e^{i\left(\omega_1+\omega_0\right)}.
\end{equation*}
Осталось исключить $\omega_1$, для чего достаточно взять модуль от левой и правой частей последнего равенства
\begin{equation} \label{eq:w_0}
  \left|\dot{z}_\text{н} e^{i\omega_0} - \dot{z}_0 \right| = \left|\dot{z}_\text{в} e^{-i\omega_0} - \dot{z}_0\right|.
\end{equation}
Аналогичное соотношение получается и для $\omega_1$:
\begin{equation} \label{eq:w_1}
  \left|\dot{z}_\text{н} e^{-i\omega_1} - \dot{z}_0 \right| = \left|\dot{z}_\text{в} e^{i\omega_1} - \dot{z}_0 \right|.
\end{equation}

\subsubsection{Анализ разрешимости уравнения относительно $\beta_0$}

Запишем уравнение \eqref{eq:w_0} в следующем виде
\begin{equation*}
  \left|\dot{z}_\text{н} e^{ i \omega_0} - \dot{z}_0 \right|^2 =
  \left|\dot{z}_\text{в} e^{-i \omega_0} - \dot{z}_0 \right|^2.
\end{equation*}
Воспользовавшись тем, что $\left|\dot{z}\right|^2= \dot{z} \dot{z}^*$, где $\dot{z}^*$ означает комплексное сопряжение, придем к равенству
\begin{equation*}
  \left(\dot{z}_\text{н} e^{ i\omega_0} - \dot{z}_0 \right) \left(\dot{z}_\text{н} e^{ i\omega_0} - \dot{z}_0 \right)^* = \left(\dot{z}_\text{в} e^{-i\omega_0} - \dot{z}_0 \right) \left(\dot{z}_\text{в} e^{-i\omega_0} - \dot{z}_0 \right)^*.
\end{equation*}
Раскрывая скобки, получим
\begin{equation*}
  \left|\dot{z}_\text{н}\right|^2 - \dot{z}_\text{н} \dot{z}_{0}^* e^{ i\omega_0} \dot{z}_\text{н}^* \dot{z}_{0} e^{-i\omega_0} + \left|\dot{z}_{0}\right|^2 =
  \left|\dot{z}_\text{в} \right|^2 - \dot{z}_\text{в} \dot{z}_{0}^* e^{-i\omega_0} - \dot{z}_\text{в}^* \dot{z}_{0} e^{i\omega_0} + \left|\dot{z}_{0}\right|^2.
\end{equation*}
Перегруппируем слагаемые:
\begin{equation*}
  \left(\dot{z}_\text{в}^* \dot{z}_{0} - \dot{z}_\text{н} \dot{z}_{0}^* \right) e^{ i\omega_0} + \left(\dot{z}_\text{в} \dot{z}_{0}^* - \dot{z}_\text{н}^* \dot{z}_{0} \right) e^{-i\omega_0} = \left|\dot{z}_\text{в}\right|^2 - \left|\dot{z}_\text{н}\right|^2.
\end{equation*}
Обозначим
\begin{equation} \label{eq:u_v}
  \dot{z}_\text{в}^* \dot{z}_{0} - \dot{z}_\text{н} \dot{z}_{0}^* = u - iv.
\end{equation}
Тогда $\dot{z}_\text{в} \dot{z}_{0}^* - \dot{z}_\text{н}^* \dot{z}_{0} = u + iv$, и
\begin{equation*}
  \left(u - iv\right) \left(\cos\omega_0 + i \sin\omega_0\right) + \left(u + iv\right) \left(\cos\omega_0 - i \sin\omega_0\right) = \left|\dot{z}_\text{в}\right|^2 - \left|\dot{z}_\text{н}\right|^2.
\end{equation*}
Соответственно,
\begin{equation} \label{eq:u_v_omega_0}
  u \cos\omega_0 + v \sin\omega_0 =
  \frac{ \left|\dot{z}_\text{в}\right|^2 - \left|\dot{z}_\text{н}\right|^2} {2}=c.
\end{equation}

Условие разрешимости уравнения \eqref{eq:u_v_omega_0} может быть записано в следующем виде
\begin{equation} \label{eq:c_good}
  u^2 + v^2 \geq c^2.
\end{equation}
Для решения \eqref{eq:u_v_omega_0} перейдем к половинному углу
\begin{equation*}
  u \left( \cos^2\frac{\omega_0}{2} - \sin^2\frac{\omega_0}{2} \right) + 2v \sin\frac{\omega_0}{2} \cos\frac{\omega_0}{2} =
  c \left(\cos^2\frac{\omega_0}{2} + \sin^2\frac{\omega_0}{2} \right)
\end{equation*}
и перегруппируем слагаемые
\begin{equation*}
  \left(u - c\right) \cos^2\frac{\omega_0}{2} + 2v \sin\frac{\omega_0}{2} \cos\frac{\omega_0}{2} - \left(u + c\right) \sin^2\frac{\omega_0}{2} = 0.
\end{equation*}
Если $\left|u - c\right| > \left|u + c\right|$, то делим данное уравнение на $\sin^2 \sfrac{\omega_0}{2}$, в противном случае делим на $\cos^2 \sfrac{\omega_0}{2}$. В первом случае получается квадратное уравнение относительно $\ctg \sfrac{\omega_0}{2}$:
\begin{equation*}
    \left(u - c\right) \ctg^2 \frac{\omega_0}{2} + 2v \ctg\frac{\omega_0}{2} - \left(u + c\right) = 0,
\end{equation*}
во втором случае "--- относительно $\tg \frac{\omega_0}{2}$:
\begin{equation*}
    \left(u + c\right) \tg^2 \frac{\omega_0}{2} - 2v \tg \frac{\omega_0}{2} - \left(u - c\right) = 0.
\end{equation*}
Дискриминант для обоих уравнений равен $4\left(u^2 + v^2 - c^2\right)$ и в силу \eqref{eq:c_good} неотрицателен.

Поскольку $\beta_0 > \ang{0}$, то и $\omega_0 > 0$, а из ограничений сверху на параметры рассматриваемой модели $\beta_0 \leq \ang{15}$ и $ \Delta h^{}/\lambda \leq 3.5$ следует, что $\omega_0 < 2 \pi $. Соответственно, $0 < \rfrac{\omega_0}{2} < \pi$. Пусть, например, $t_1$, $t_2$ "--- корни квадратного уравнения относительно $\tg \rfrac{\omega_0}{2}$. Если оба корня положительные, то выбираем меньший из них $t_1$. Тогда
\begin{equation*}
  \omega_0 = 2 \arctg t_1, \; \beta_0 = \arcsin \frac{\omega_0 \lambda} {2\pi \Delta h}.
\end{equation*}
Если корни разных знаков, то выбираем положительный. Если оба корня отрицательные, то выбираем больший по модулю $t_1$ и полагаем $\omega_0 = 2\left( \arctg t_1 + \pi\right)$. В случае кратного корня делаем то же самое, только не требуется выбирать один из двух корней.
Несколько проще случай уравнения относительно $\ctg\rfrac{\omega_0}{2}$. Из двух возможных значений корней выбирается больший.

Возможен и численный метод решения уравнения \eqref{eq:u_v_omega_0}, например, метод половинного деления. В качестве начального отрезка для $\beta_0$ выбирается отрезок $[\ang{0}, \ang{15}]$, для середины отрезка вычисляется  $\omega_0$ по формуле \eqref{eq:w_0_w_1} и отрезок сужается вдвое. Количество итераций определяется поставленной точностью вычислений. Подводя итог, можно сказать, что для решения исходной задачи определения угла $\beta_0$ надо по заданным входным данным $\dot{z}_\text{в}$, $\dot{z}_{0}$, $\dot{z}_\text{н}$ определить согласно формуле \eqref{eq:u_v} вспомогательные величины $u$, $v$ и решить аналитически или численно тригонометрическое уравнение \eqref{eq:u_v_omega_0}.

\begin{figure}[htp]
  \centering
  \subfloat[\textit{а)}]{\includegraphics[width=0.3\linewidth]{3/surface/pic2_a}}
  \subfloat[\textit{б)}]{\includegraphics[width=0.3\linewidth]{3/surface/pic2_b}}
  \subfloat[\textit{в)}]{\includegraphics[width=0.3\linewidth]{3/surface/pic2_c}}
  \\
  \subfloat[\textit{г)}]{\includegraphics[width=0.3\linewidth]{3/surface/pic2_d}}
  \subfloat[\textit{д)}]{\includegraphics[width=0.3\linewidth]{3/surface/pic2_e}}
  \subfloat[\textit{е)}]{\includegraphics[width=0.3\linewidth]{3/surface/pic2_f}}

  \caption{Графики угловых зависимостей случайных (\textit{а}, \textit{б}, \textit{в}) и методических (\textit{г}, \textit{д}, \textit{е}) СКО определения угла при различных значениях параметров принятого сигнала: амплитуды (\textit{а}, \textit{г}), угла приема отраженной волны (\textit{б}, \textit{д}) и разности фаз принятых сигналов (\textit{в}, \textit{е}).}
  \label{fig:surface:pic2}
\end{figure}

\subsubsection{Статистический анализ погрешности}

Наличие явного метода решения системы существенно облегчает анализ погрешности. Стандартным образом возмущаются входные данные $\dot{z}_\text{в}$, $\dot{z}_{о}$ и $\dot{z}_\text{н}$ с заданным отношением сигнал/шум (ОСШ) $q$, решается полученная система и вычисляется среднеквадратичная ошибка $\sigma_\beta$ определения угла $\beta_0$.

Возмущения исходных значений сигналов на каждой из антенн производилось путем добавления к ним некоторой случайной величины, распределенной по нормальному закону с нулевым математическим ожиданием и дисперсией, соответствующей заданному ОСШ. Устойчивость предложенного метода определения угла $\beta_0$ оценивалась при значении ОСШ $q = \SI{20}{\deciBells}$. Также при оценке рассматривался идеальный случай, в котором влияние шума пренебрежимо мало, т.е. ОСШ $q = \SI{100}{\deciBells}$. На Рис.~\ref{fig:surface:pic2} приведены графики зависимостей СКО $\sigma_\beta$ от определяемого угла $\beta_0$.

\begin{figure}[hpb]
  \centering
  \subfloat[\textit{а)}]{\includegraphics[width=0.3\linewidth]{3/surface/pic3_a}}
  \subfloat[\textit{б)}]{\includegraphics[width=0.3\linewidth]{3/surface/pic3_b}}
  \subfloat[\textit{в)}]{\includegraphics[width=0.3\linewidth]{3/surface/pic3_c}}
  \\
  \subfloat[\textit{г)}]{\includegraphics[width=0.3\linewidth]{3/surface/pic3_d}}
  \subfloat[\textit{д)}]{\includegraphics[width=0.3\linewidth]{3/surface/pic3_e}}
  \subfloat[\textit{е)}]{\includegraphics[width=0.3\linewidth]{3/surface/pic3_f}}

  \caption{Графики частотных зависимостей случайных СКО определения угла при различных значениях параметров принятого сигнала: амплитуды (\textit{а}, \textit{г}), угла приема отраженной волны (\textit{б}, \textit{д}) и разности фаз принятых сигналов (\textit{в}, \textit{е}).}
  \label{fig:surface:pic3}
\end{figure}

Также была произведена оценка зависимости СКО от уровня взаимного влияния между антеннами $\Delta h^{}/\lambda$. На Рис.~\ref{fig:surface:pic3} приведены графики этих зависимостей при различных значениях углов $\beta_0$ и значений параметров принятых сигналов.

С помощью моделирования было так же выяснено, что при увеличении ОСШ кривая угловых зависимостей СКО приобретает ярко выраженный L-образный характер, т.е. погрешность резко убывает при определенном значении определяемого угла $\beta_0$. Рис.~\ref{fig:surface:pic4} показывает характер этой зависимости при ОСШ $q = \SI{30}{\deciBells}$.

\begin{figure}[hpb]
  \centering
  \subfloat[\textit{а)}]{\includegraphics[width=0.3\linewidth]{3/surface/pic4_a}}
  \subfloat[\textit{б)}]{\includegraphics[width=0.3\linewidth]{3/surface/pic4_b}}
  \subfloat[\textit{в)}]{\includegraphics[width=0.3\linewidth]{3/surface/pic4_c}}

  \caption{Графики угловых зависимостей случайных СКО определения угла при различных значениях параметров принятого сигнала: амплитуды (\textit{а}), угла приема отраженной волны (\textit{б}) и разности фаз принятых сигналов (\textit{в})}
  \label{fig:surface:pic4}
\end{figure}

Главное отличие от модели, предложенной в [ВГМНП], состоит в том, что система содержит шесть неизвестных вещественных переменных, а не пять, и поэтому не является переопределенной. Соответственно, нет областей изменения параметров, в которых система несовместна. Сложности, как и следовало ожидать, начинаются при стремлении $\beta_0$ к нулю "--- в этом случае все три коэффициента $u$, $v$ и $c$ уравнения \eqref{eq:u_v_omega_0} становятся малыми, и преобразования, связанные с делением на эти величины, являются неустойчивыми.

Из приведенных в статье графиков среднеквадратичных погрешностей (рис. 2 и 3) можно сделать следующий вывод: если $\beta_0$ имеет величину порядка $\ang{1}$ "--- $\ang{3}$, то погрешность становится больше самой определяемой величины, т.е. вычисления с практической точки зрения бесполезны. Существенным оказывается и соотношение углов $\beta_0$ и $\beta_1$. С увеличением разности между ними растет и погрешность. Большая разность между этими углами возможна в том случае, когда объект пеленгации находится слишком близко к датчикам. Достаточно важным параметром является разность фаз $\Delta \varphi$ – если пришедшие сигналы приходят в фазе или противофазе, то предложенная модель оказывается неэффективной. Это объясняется тем, что величины $\dot{z}_\text{в}$ и $\dot{z}_\text{н}$ становятся слишком близки друг к другу и критерий разрешимости \eqref{eq:c_good} нарушается для большинства значений $\beta_0$.

\end{document}