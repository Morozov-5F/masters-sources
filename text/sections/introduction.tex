% !TEX root = ../main.tex
\documentclass[../main.tex]{subfiles}
\begin{document}

В настоящее время большое распространение получили глобальные навигационные спутниковые системы (ГНСС), такие как отечественная ГЛОНАСС или американская NAVSTAR. Излучателями электромагнитных волн в данном случае являются навигационные спутники. Основной задачей таких систем является определение координат подвижных объектов. Угловая ориентация (к примеру, угол курса самолета) определяются другими системами, например, угломерно-дальномерной VOR/DME, используемой в гражданских аэропортах, или военной радиотехнической системой ближней навигации (РСБН). Помимо этого, угловая ориентация может определяться с помощью навигационных приборов, таких как высотомеры.

В связи с распространением ГНСС, в работах~\cite{KORNEV:2016} и~\cite{STEPANOV:2006} были приведены способы определения угловой ориентации с помощью спутниковых систем навигации. Хотя эти системы предоставляют высокую точность, они довольны сложны в реализации и накладывают требование синхронности излучения сигналов навигационными космическими аппаратами.

В работе~\cite{BBELAVIN:1977} исследованы различные способы определения координат подвижного объекта угломерным методом, которые не накладывают требования синхронности излучения радиосигналов радиоориентирами. Однако, возможности одновременного и однозначного определения координат и угловой ориентации подвижного объекта путем азимутально-угломестного радиопеленгования с борта объекта без использования какой-либо вспомогательной информации от автономных навигационных датчиков и систем не изучены в современных работах по радионавигации. При этом, в некоторых задачах возникает необходимость определения ограничений на количество радиоориентиров и на взаимное расположение объекта и РО при условии сохранения однозначности определения координат и угловой ориентации~\cite{VINOGRADOV:2016}.

Профессор А.~Д.~Виноградов из Военно-воздушной академии им.~Н.~Е.~Жуковского и Ю.~А.~Гагарина предложил вариант единой системы, основанной на измерении углов с помощью фазированных антенных решеток. В такой системе источниками сигналов являются радиоориентиры, размещенные в точках с известными координатами. Подвижный объект оснащен бортовой пеленгаторной антенной, способной определять азимут и угол места каждого радиоориентира.

Измерения азимутов и углов места проводятся в системе координат, связанной с подвижным объектом. Всего необходимо определить шесть параметров "--- три координаты фазового центра бортовой пеленгаторной антенны и три угла Эйлера, определяющих угловую ориентацию подвижного объекта. Минимальное число радиоориентиров в таком случае равно трем, так как бортовая пеленгаторная антенна подвижного объекта измеряет два параметра для каждого из радиоориентиров "--- азимут и угол места.

Можно выделить три этапа решения поставленной задачи:
\begin{enumerate}
    \item Нахождение совокупности расстояний от подвижного объекта до радиоориентиров;
    \item Определение координат подвижного объекта;
    \item Нахождение матрицы вращения и связанных с нею углов Эйлера, определяющих угловую ориентацию подвижного объекта.
\end{enumerate}
Задачи второго и третьего этапов являются стандартными для радионавигации подвижных объектов~\cite{SOSULIN:1992}. Проведенное компьютерное моделирование показало, что система уравнений возникающая на первом этапе, имеет несколько решений. Это приводит сразу к нескольким проблемам:
\begin{enumerate}
    \item Как определить количество решений? \label{q:1}
    \item Как выбрать правильное решение? \label{q:2}
    \item Как оценить погрешность нахождения решения при заданных погрешностях входных данных? \label{q:3}
\end{enumerate}
Проблемы 1 и 2 были решены в рамках этой работы, в то время как третья остается нерешенной "--- этот вопрос требует дополнительного исследования.

В первых двух главах этой работы производится постановка задачи на формальном уровне, а также вводятся основные понятия и методика, необходимая для решения поставленной задачи.

Основным результатом данной работы являются алгоритмы определения координат и угловой ориентации подвижных объектов для различных вариаций локальных угломерных навигационных систем. В третьей главе приводятся описания этих алгоритмов, а также присутствует детальный разбор минимально возможной локальной угломерной системы. Здесь же приведен анализ влияния подстилающей поверхности на результаты измерения направлений на радиоориентиры.

В главе четыре производится систематизация и обсуждение результатов, полученных в третьей главе.

В приложениях можно найти определения систем координат, в которых оперируют рассматриваемые в данной работе локальные навигационные системы. Также в приложение вынесен листинг программы на языке \texttt{Wolfram Mathematica}, с помощью которого можно смоделировать поведение простейших ЛНС, описанных в третьей главе.

\end{document}