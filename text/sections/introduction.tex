% !TEX root = ../main.tex
\documentclass[../main.tex]{subfiles}
\begin{document}

В настоящее время большое распространение получили глобальные навигационные спутниковые системы (ГНСС), такие как отечественная ГЛОНАСС или американская NAVSTAR. Излучателями электромагнитных волн в данном случае являются навигационные спутники. Основной задачей таких систем является определение координат подвижных объектов. Угловая ориентация (например, угол курса самолета) определяются другими системами, например, угломерно-дальномерной VOR/DME, используемой в гражданских аэропортах, или военной радиотехнической системой ближней навигации (РСБН). Помимо этого, угловая ориентация может определяться с помощью навигационных приборов, таких как высотомеры.

В связи с распространением ГНСС, в работах~\cite{KORNEV:2016} и~\cite{STEPANOV:2006} были приведены способы определения угловой ориентации с помощью спутниковых систем навигации. Хотя эти системы предоставляют высокую точность, они довольны сложны в реализации и накладывают требование синхронности излучения сигналов навигационными космическими аппаратами.

Профессор А.~Д.~Виноградов из Военно-воздушной академии им.~Н.~Е.~Жуковского и Ю.~А.~Гагарина предложил вариант единой системы, основанной на измерении углов с помощью фазированных антенных решеток. В такой системе источниками сигналов являются радиоориентиры, размещенные в точках с известными координатами. Подвижный объект оснащен бортовой пеленгаторной антенной, способной определять азимут и угол места каждого радиоориентира.

Измерения азимутов и углов места проводятся в системе координат, связанной с подвижным объектом. Всего необходимо определить шесть параметров "--- три координаты фазового центра бортовой пеленгаторной антенны и три угла Эйлера, определяющих угловую ориентацию подвижного объекта. Минимальное число радиоориентиров в таком случае равно трем, так как бортовая пеленгаторная антенна подвижного объекта измеряет два параметра для каждого из радиоориентиров "--- азимут и угол места.

В первых главах этой работы производится постановка задачи на формальном уровне, а также вводятся основные понятия и методика, необходимая для решения поставленной задачи.

Основным результатом данной работы являются алгоритмы определения координат и угловой ориентации подвижных объектов для различных вариаций локальных угломерных навигационных систем. В третьей главе приводятся описания этих алгоритмов, а также присутствует детальный разбор минимально возможной локальной угломерной системы. Здесь же приведен анализ влияния подстилающей поверхности на результаты измерения направлений на радиоориентиры.

В главе четыре производится систематизация и обсуждение результатов, полученных в третьей главе.

В приложениях можно найти определения систем координат, в которых оперируют рассматриваемые в данной работе локальные навигационные системы. Также в приложение вынесен листинг программы на языке \texttt{Wolfram Mathematica}, с помощью которого можно смоделировать поведение простейших ЛНС, описанных в третьей главе.

\end{document}