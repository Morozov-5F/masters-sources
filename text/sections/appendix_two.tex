% !TEX root = ../main.tex
\documentclass[../main.tex]{subfiles}

\begin{document}
\lstset{style=wolfram_mathematica}

В данном приложении представлен листинги процедур, реализующие этапы представленного в пункте 3 алгоритма решения задачи. Для написания программ использовался язык {\ttfamily Wolfram Language}.

\subsection{Вспомогательные функции}
Функции в этом разделе используются в основном для подготовки входных параметров и представления данных.
\begin{lstlisting}
ComposeFaces[foundation_, apex_] := Module[{foundationPairs,
                                            points},
  foundationPairs = Append[Partition[foundation, 2, 1],
                            { Last[foundation],
                              First[foundation] }];
  points = Prepend[#, apex] & /@ foundationPairs;
  points
]

PlotFigure[foundation_, apex_] := Module[{apexLow},
  (* Предполагаем, что вершины отсортированы. Сортировка невыпуклых многоугольников -- нетривиальная задача *)

  apexLow = Flatten@{apex[[1 ;; 2]], 0};
  Graphics3D[{ EdgeForm@Thick, FaceForm@Opacity[0.9],
                Polygon /@ ComposeFaces[foundation, apex],
                Point /@ foundation, Point@apex, Point@apexLow,
                {Thickness[0.007], Line[{apex, apexLow}]} },
              Axes -> True, AxesLabel -> {"x", "y", "z"},
              FaceGrids -> All]
]

PlotPointsColored[foundation_, points_] := Module[{colors, colorsByLen, geomPoints, coloredPoints},
  colors = {1 -> Purple, 2 -> Blue, 3 -> LightOrange,
            4 -> Orange, 5 -> Red, 6 -> Magenta};
  coloredPoints = AssignColorsToPoints[points, colors];

  Row[
    Graphics3D[{{EdgeForm[Directive[Thick, Green]], Transparent, Polygon[foundation]}, coloredPoints}, Axes -> True, AxesLabel -> {"x", "y", "z"}, FaceGrids -> All],
    Graphics3D[{{EdgeForm[Directive[Thick, Green]], Transparent, Polygon[foundation]}, coloredPoints}, Axes -> True, AxesLabel -> {"x", "y", "z"}, FaceGrids -> All, ViewPoint -> {0, 0, \[Infinity]}]}
  ]
]

AssignColorsToPoints[points_, colors_] :=
                Module[{colorsByLen, geomPoints, coloredPoints},
  colorsByLen = Length /@ points /. colors;
  geomPoints = Map[Point, points, {2}];
  coloredPoints = Table[{colorsByLen[[i]], geomPoints[[i]]},
                        {i, Length@points}];
  coloredPoints
]

ExtractApex[foundation_, sides_, startEq_:  0, endEq_: 0] :=
                Module[{system, point, start, end},
  If [endEq == 0, end = Length@foundation];
  If [startEq == 0, start = 1];
  system = Table[(foundation[[i, 1]] - x)^2 +
                  (foundation[[i, 2]] - y)^2 +
                  (foundation[[i, 3]] - z)^2 == Subscript[l, i]^2,
                  {i, start, end}] /. sides;
  system = Append[system, z >= 0];
  point = NSolve[system, {x, y, z}];
  point
]
\end{lstlisting}

\subsection{Основные функции}
Функции, представленные в этом разделе, являются реализацией алгоритма, представленного в разделе 3. В частности, функция {\ttfamily SolveForThree} решает задачу для трех наземных источников, а функции {\ttfamily SolveForFour4} и {\ttfamily SolveForFour6} "--- для четырех, но используя различные подходы.

\begin{lstlisting}
SolveForThree[triangle_, apex_] :=
                Module[{system, \[Alpha], d, ll, solution},

  (* Подсчет длин ребер основания *)
  Subscript[d, 1, 2] = SquaredEuclideanDistance[triangle[[1]],
                                                triangle[[2]]];
  Subscript[d, 2, 3] = SquaredEuclideanDistance[triangle[[2]],
                                                triangle[[3]]];
  Subscript[d, 3, 1] = SquaredEuclideanDistance[triangle[[3]],
                                                triangle[[1]]];

  (* Подсчет длин боковых ребер *)
  ll = Table[SquaredEuclideanDistance[triangle[[i]], apex],
                                      {i, 3}];

  (* Подсчет плоских углов *)
  Subscript[\[Alpha], 1, 2] =
    -(Subscript[d, 1, 2] - ll[[1]] - ll[[2]]) /
        (2 * Sqrt[ll[[1]]] * Sqrt[ll[[2]]]);
  Subscript[\[Alpha], 2, 3] =
    -(Subscript[d, 2, 3] - ll[[2]] - ll[[3]]) /
        (2 * Sqrt[ll[[2]]] * Sqrt[ll[[3]]]);
  Subscript[\[Alpha], 3, 1] =
    -(Subscript[d, 3, 1] - ll[[3]] - ll[[1]]) /
        (2 * Sqrt[ll[[3]]] * Sqrt[ll[[1]]]);

  system = {
    Subscript[l, 1]^2 + Subscript[l, 2]^2 -
      2 * Subscript[l, 1] * Subscript[l, 2] *
        Subscript[\[Alpha], 1, 2] == Subscript[d, 1, 2],
    Subscript[l, 2]^2 + Subscript[l, 3]^2 -
      2 * Subscript[l, 2] * Subscript[l, 3] *
        Subscript[\[Alpha], 2, 3] == Subscript[d, 2, 3],
    Subscript[l, 3]^2 + Subscript[l, 1]^2 -
      2 * Subscript[l, 3] * Subscript[l, 1] *
        Subscript[\[Alpha], 3, 1] == Subscript[d, 3, 1],
    Subscript[l, 1] >= 0,
    Subscript[l, 2] >= 0,
    Subscript[l, 3] >= 0
  };
  solution = NSolve[system, { Subscript[l, 1],
                              Subscript[l, 2],
                              Subscript[l, 3]},
                              Reals, Method -> "Newton"];
  Flatten[{x, y, z} /. # & /@ ExtractApex[triangle, #] & /@ solution, 1]
]

SolveForFour4[quad_, apex_, prevRoot_: {}] :=
    Module[{system, \[Alpha], d, ll, solution, sol13, sol24},

  Subscript[d, 1, 2] = SquaredEuclideanDistance[quad[[1]],
                                                quad[[2]]];
  Subscript[d, 2, 3] = SquaredEuclideanDistance[quad[[2]],
                                                quad[[3]]];
  Subscript[d, 3, 4] = SquaredEuclideanDistance[quad[[3]],
                                                quad[[4]]];
  Subscript[d, 4, 1] = SquaredEuclideanDistance[quad[[4]],
                                                quad[[1]]];

  ll = Table[SquaredEuclideanDistance[quad[[i]], apex], {i, 4}];

  Subscript[\[Alpha], 1, 2] =
    -(Subscript[d, 1, 2] - ll[[1]] - ll[[2]]) /
      (2 * Sqrt[ll[[1]]] * Sqrt[ll[[2]]]);
  Subscript[\[Alpha], 2, 3] =
    -(Subscript[d, 2, 3] - ll[[2]] - ll[[3]]) /
      (2 * Sqrt[ll[[2]]] * Sqrt[ll[[3]]]);
  Subscript[\[Alpha], 3, 4] =
    -(Subscript[d, 3, 4] - ll[[3]] - ll[[4]]) /
      (2 * Sqrt[ll[[3]]] * Sqrt[ll[[4]]]);
  Subscript[\[Alpha], 4, 1] =
    -(Subscript[d, 4, 1] - ll[[4]] - ll[[1]]) /
      (2 * Sqrt[ll[[4]]] * Sqrt[ll[[1]]]);

  system = {
    Subscript[l, 1]^2 + Subscript[l, 2]^2 -
      2 * Subscript[l, 1] * Subscript[l, 2] * Subscript[\[Alpha], 1, 2] == Subscript[d, 1, 2],
    Subscript[l, 2]^2 + Subscript[l, 3]^2 -
      2 * Subscript[l, 2] * Subscript[l, 3]*Subscript[\[Alpha], 2, 3] == Subscript[d, 2, 3],
    Subscript[l, 3]^2 + Subscript[l, 4]^2 -
      2 * Subscript[l, 3] * Subscript[l, 4]*Subscript[\[Alpha], 3, 4] == Subscript[d, 3, 4],
    Subscript[l, 4]^2 + Subscript[l, 1]^2 -
      2 * Subscript[l, 4] * Subscript[l, 1]*Subscript[\[Alpha], 4, 1] == Subscript[d, 4, 1],
    Subscript[l, 1] >= 0,
    Subscript[l, 2] >= 0,
    Subscript[l, 3] >= 0,
    Subscript[l, 4] >= 0
  };
  solution = NSolve[system, { Subscript[l, 1], Subscript[l, 2],
                              Subscript[l, 3], Subscript[l, 4] },
                              Reals];
  sol13 = Flatten[{x, y, z} /. # & /@ ExtractApex[quad, #, 1, 3] & /@ solution, 1];
  sol24 = Flatten[{x, y, z} /. # & /@ ExtractApex[quad, #, 2, 4] & /@ solution, 1];
  Intersection[sol13, sol24,
               SameTest->(EuclideanDistance[#1, #2] < 10^(-5) &)]
]

SolveForFour6[quad_, apex_] :=
    Module[{system, \[Alpha], d, ll, solution, sol13, sol24},

  Subscript[d, 1, 2] = SquaredEuclideanDistance[quad[[1]],
                                                quad[[2]]];
  Subscript[d, 2, 3] = SquaredEuclideanDistance[quad[[2]],
                                                quad[[3]]];
  Subscript[d, 3, 4] = SquaredEuclideanDistance[quad[[3]],
                                                quad[[4]]];
  Subscript[d, 4, 1] = SquaredEuclideanDistance[quad[[4]],
                                                quad[[1]]];
  Subscript[d, 1, 3] = SquaredEuclideanDistance[quad[[1]],
                                                quad[[3]]];
  Subscript[d, 2, 4] = SquaredEuclideanDistance[quad[[2]],
                                                quad[[4]]];

  ll = Table[SquaredEuclideanDistance[quad[[i]], apex], {i, 4}];

  Subscript[\[Alpha], 1, 2] =
    -(Subscript[d, 1, 2] - ll[[1]] - ll[[2]]) /
      (2 * Sqrt[ll[[1]]] * Sqrt[ll[[2]]]);
  Subscript[\[Alpha], 2, 3] =
    -(Subscript[d, 2, 3] - ll[[2]] - ll[[3]]) /
      (2 * Sqrt[ll[[2]]] * Sqrt[ll[[3]]]);
  Subscript[\[Alpha], 3, 4] =
    -(Subscript[d, 3, 4] - ll[[3]] - ll[[4]]) /
      (2 * Sqrt[ll[[3]]] * Sqrt[ll[[4]]]);
  Subscript[\[Alpha], 4, 1] =
    -(Subscript[d, 4, 1] - ll[[4]] - ll[[1]]) /
      (2 * Sqrt[ll[[4]]] * Sqrt[ll[[1]]]);
  Subscript[\[Alpha], 1, 3] =
    -(Subscript[d, 1, 3] - ll[[1]] - ll[[3]]) /
      (2 * Sqrt[ll[[1]]] * Sqrt[ll[[3]]]);
  Subscript[\[Alpha], 2, 4] =
    -(Subscript[d, 2, 4] - ll[[2]] - ll[[4]]) /
      (2 * Sqrt[ll[[2]]] * Sqrt[ll[[4]]]);

  system = {
    Subscript[l, 1]^2 + Subscript[l, 2]^2 -
      2 * Subscript[l, 1] * Subscript[l, 2]*Subscript[\[Alpha], 1, 2] == Subscript[d, 1, 2],
    Subscript[l, 2]^2 + Subscript[l, 3]^2 -
      2 * Subscript[l, 2] * Subscript[l, 3]*Subscript[\[Alpha], 2, 3] == Subscript[d, 2, 3],
    Subscript[l, 3]^2 + Subscript[l, 4]^2 -
      2 * Subscript[l, 3] * Subscript[l, 4]*Subscript[\[Alpha], 3, 4] == Subscript[d, 3, 4],
    Subscript[l, 4]^2 + Subscript[l, 1]^2 -
      2 * Subscript[l, 4] * Subscript[l, 1]*Subscript[\[Alpha], 4, 1] == Subscript[d, 4, 1],
    Subscript[l, 1]^2 + Subscript[l, 3]^2 -
      2 * Subscript[l, 1] * Subscript[l, 3]*Subscript[\[Alpha], 1, 3] == Subscript[d, 1, 3],
    Subscript[l, 2]^2 + Subscript[l, 4]^2 -
      2 * Subscript[l, 2] * Subscript[l, 4]*Subscript[\[Alpha], 2, 4] == Subscript[d, 2, 4],
    Subscript[l, 1] >= 0,
    Subscript[l, 2] >= 0,
    Subscript[l, 3] >= 0,
    Subscript[l, 4] >= 0
  };

  solution = Solve[system, {Subscript[l, 1], Subscript[l, 2],
                            Subscript[l, 3], Subscript[l, 4]},
                            Reals];
  sol13 = Flatten[{x, y, z} /. # & /@ ExtractApex[quad, #, 1, 3] & /@ solution, 1];
  sol24 = Flatten[{x, y, z} /. # & /@ ExtractApex[quad, #, 2, 4] & /@ solution, 1];
  Intersection[sol13, sol24,
               SameTest -> (EuclideanDistance[#1, #2] < 10^(-5) &)]
]
\end{lstlisting}

\subsection{Пример использования функций}
Листинг программы для нахождения координат подвижного объекта на различных высотах в случае, когда РО расположены в верщинах правильного треугольника, а подвижный объект начинает подъем из центра масс треугольника:
\begin{lstlisting}
  M1 = {0, 0, 0};
  M2 = {6, 9 , 0};
  M3 = {10, 0, 0};
  M0 = {5, 4, 6};
  foundation = {M1, M2, M3};
  PlotFigure[foundation, M0]
  solutions =
    Table[{{M0[[1]], M0[[2]], i},
          SolveForThree[foundation,
                        {M0[[1]], M0[[2]], i}]},
                        {i, 0, 10, 0.1}];
  Grid@solutions;
  PlotPointsColored[foundation, solutions[[1 ;;, 2]]]
\end{lstlisting}

\end{document}