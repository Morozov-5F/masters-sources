% !TEX root = ../main.tex
\documentclass[../main.tex]{subfiles}

\begin{document}

\newpage
\section*{\centering Реферат}

\par\noindent Магистерская диссертация \zpageref{LastPage}~с., \total{citnum}\ источников, \total{totalappendices} приложения.\\
\vspace{0.5cm}

\par\noindent  МЕТОД НЬЮТОНА, ОРТОГОНАЛЬНЫЕ ПРЕОБРАЗОВАНИЯ, ПРОСТРАНСТВЕННАЯ ОРИЕНТАЦИЯ, АЗИМУТАЛЬНО-УГЛОМЕСТНОЕ РАДИОПЕЛЕНГОВАНИЕ, КУРС, ТАНГАЖ, КРЕН. \\

\par\noindent Объект исследования "--- локальные угломерные навигационные системы.\\
\par\noindent Цель работы "--- разработка математических моделей, алгоритмов и программной реализации, позволяющих определить координаты и пространственную ориентацию подвижных объектов.\\
\par\noindent Метод исследования и аппаратура "--- исследование структуры решений нелинейных систем алгебраических уравнений, метод Ньютона, ортогональные преобразования систем координат, теория алгоритмов, математический пакет символьных вычислений {\ttfamily Wolfram Mathematica}, персональный компьютер.\\

\par\noindent Исследована возможность определения координат и угловой ориентации подвижных объектов с помощью локальных угломерных навигационных систем по результатам азимутально-угломестного радиопеленгования радиоориентиров. Предложены различные конфигурации локальных угломерных навигационных систем, способных одновренменно определять пространственную ориентацию подвижных объектов. С помощью компьютерного моделирования исследовано воздействие подстилающей поверхности на результаты угломестного радиопеленгования радиоориентиров.

\clearpage
\normalsize

\end{document}