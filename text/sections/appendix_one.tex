% !TEX root = ../main.tex
\documentclass[../main.tex]{subfiles}
\begin{document}

Государственный стандарт $20058$ 1980 года (ГОСТ $20058$-$80$) содержит в себе определения различных систем координат, используемых при работе с подвижными объектами, в частности, летательными аппаратами. Данное приложение описывает системы координат, используемые в предложенной работе. Следует отметить, что определения, данные в разделах~\ref{sec:nzsk} и~\ref{sec:ssk} настоящего приложения несколько отличаются от тех, что предоставлены в ГОСТ $20058$-$80$. В частности, системы координат, используемые в данной работе, являются левыми (а не правыми), а нормальной осью является ось аппликат (а не ординат).

\subsection{Нормальная Земная Система Координат}\label{sec:nzsk}
Нормальная земная система координат (НЗСК) $\Sigma_{\text{НЗ}} = \left\{O, X, Y, Z\right\}$ "--- левая прямоугольная декартова система координат, начало $O$ которой фиксировано по отношению к Земле, ось абсцисс $OX$ которой, находящаяся в горизонтальной плоскости, совпадает с северным направлением истинного или магнитного меридиана или вертикальной линии координатной сетки плоской прямоугольной геодезической системы координат, ось аппликат $OZ$ которой перпендикулярна горизонтальной плоскости и направлена вверх по вертикали, а ось ординат $OY$ которой, находящаяся в горизонтальной плоскости $XOY$, дополняет систему до левой пространственной прямоугольной декартовой системы координат.

\subsection{Связанная система координат}\label{sec:ssk}
Связанная система координат $\Sigma_{\text{СВ}} = \left\{O', X', Y', Z'\right\}$ "--- левая пространственная прямоугольная декартова система координат, осями которой являются продольная ось $O'X'$, поперечная ось $O'Y'$ и нормальная ось $O'Z'$, фиксированные относительно летательного аппарата.

\textit{Продольная ось} $O'X'$ "--- ось связанной системы координат, расположенная в плоскости симметрии летательного аппарата или в плоскости, параллельной ей, если начало координат $O'$ помещено вне плоскости симметрии, и направленная от хвостовой к носовой части летательного аппарата

\textit{Поперечная ось} $O'Y'$ "--- ось связанной системы координат, перпендикулярная плоскости симметрии летательного аппарата и направленная к левой части летательного аппарата или части, условно ей соответствующей.

\textit{Нормальная ось} $O'Z'$ "--- ось связанной системы координат, расположенная в плоскости симметрии летательного аппарата или в плоскости, параллельной ей, если начало координат $O'$, и направленная к верхней части летательного аппарата или части, условно ей соответствующей.



\end{document}