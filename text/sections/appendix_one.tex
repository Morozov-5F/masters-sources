% !TEX root = ../main.tex
\documentclass[../main.tex]{subfiles}
\begin{document}
\setcounter{equation}{0}
\renewcommand{\theequation}{ПА.\arabic{equation}}
В данном приложении перечислены выражения, которые слишком велики для того, чтобы помещать их напрямую в текст.

\subsection{Компоненты $T(\mathrm{\RN{1}})$, $T(\mathrm{\RN{2}})$ и $T(\mathrm{\RN{3}})$ в сужении на $M$}
\begin{multline}\label{eq:appendix_t1}
\left.\left\{ \left.\frac{\mathrm d}{\mathrm{d}t}T(\mathrm{\RN{1}})\right|_{t = 0}\right\} \right|_M = \frac{1}{{x_2}^2}\left[Q\left(x_1,y_1,x_2,y_2\right) \left(-u x_2 \beta_{3,1}+2 \mu  x_2^3 \beta_{3,2}-x_2^2 \beta _{2,1} - \right.\right. \\ 
\left. -x_1 x_2 \beta_{1,1} -x_2^2 \gamma_{3,1}-x_2\sigma_{2,1} + 2 \nu  x_2 y_2^2 \beta_{3,2}+x_2 y_1 \beta _{1,2}+x_2 y_2 \beta_{2,2}\right) - \\
- \beta_{3,2} Q\left(x_1,y_1,x_2,y_2\right){}^2 + \mu  u x_2^4 \beta _{3,1}+u x_2^3 \gamma _{3,2}+\nu  u x_2^2 y_2^2 \beta _{3,1}-\mu ^2 x_2^6 \beta_{3,2}+\mu  x_2^5 \beta _{2,1} + \\
+ \mu  x_1 x_2^4 \beta_{1,1}+\mu  x_2^5 \gamma _{3,1}+x_2^4 \gamma _{2,2}+x_1 x_2^3 \gamma _{1,2}+\mu  x_2^4 \sigma _{2,1}+x_2^3 \sigma _{3,2}-2 \mu  \nu  x_2^4 y_2^2 \beta _{3,2}-\mu  x_2^4 y_1 \beta _{1,2} - \\
- \mu  x_2^4 y_2 \beta _{2,2}-\nu ^2 x_2^2 y_2^4 \beta _{3,2}+\nu  x_2^3 y_2^2 \beta _{2,1}+\nu  x_1 x_2^2 y_2^2 \beta _{1,1}-\nu  x_2^2 y_1 y_2^2 \beta _{1,2}-\nu  x_2^2 y_2^3 \beta _{2,2} +\\
\left.+\nu  x_2^3 y_2^2 \gamma _{3,1}+x_2^3 y_1 \gamma_{1,1}+x_2^3 y_2 \gamma _{2,1}+\nu  x_2^2 y_2^2 \sigma _{2,1}\right]
\end{multline}

\begin{flalign}\label{eq:appendix_t2}
\left.\left\{ \left.\frac{\mathrm d}{\mathrm{d}t}T(\mathrm{\RN{2}})\right|_{t = 0}\right\} \right|_M &= \frac{1}{x_2}F(x_1, y_1, x_2, y_2, u, v),&
\end{flalign}
где $F(x_1, y_1, x_2, y_2, u, v)$ "--- некоторая форма четвертой степени, общий вид которой зависит от квадратичной формы $Q(x_1, x_2, y_1, y_2)$.

\begin{multline}\label{eq:appendix_t3}
\left.\left\{ \left.\frac{\mathrm d}{\mathrm{d}t}T(\mathrm{\RN{3}})\right|_{t = 0}\right\} \right|_M = \frac{1}{x_2} \left[Q\left(x_1,y_1,x_2,y_2\right) \left(3 \mu  x_2^2 \beta _{3,2}-2 \nu  x_2 y_2 \beta _{3,1}+\nu  y_2^2 \beta _{3,2}\right) + \right. \\
\left.+3 \mu  u x_2^3 \beta_{3,1}+2 \nu  u x_2^2 y_2 \beta _{3,2}+\nu  u x_2 y_2^2 \beta _{3,1}-3 \mu ^2 x_2^5 \beta _{3,2}+3 \mu  x_2^4 \beta _{2,1}+3 \mu x_1 x_2^3 \beta_{1,1} +\right. \\ 
\left. + 3 \mu  x_2^3 \sigma _{2,1}+2 \mu \nu  x_2^4 y_2 \beta _{3,1}-4 \mu  \nu  x_2^3 y_2^2 \beta _{3,2}-3 \mu  x_2^3 y_1 \beta _{1,2}-3 \mu x_2^3 y_2 \beta _{2,2}+2 \nu ^2 x_2^2 y_2^3 \beta _{3,1} - \right. \\ 
\left. -\nu ^2 x_2 y_2^4 \beta _{3,2}+2 \nu  x_2^3 y_2 \beta _{2,2}+2 \nu  x_2^2 y_1 y_2 \beta_{1,1}+2 \nu  x_1 x_2^2 y_2 \beta _{1,2}+3 \nu  x_2^2 y_2^2 \beta _{2,1} + \right. \\
\left. +\nu  x_1 x_2 y_2^2 \beta _{1,1}-\nu  x_2 y_1 y_2^2 \beta _{1,2}-\nu  x_2 y_2^3 \beta _{2,2}+2 \nu  x_2^2 y_2 \sigma _{2,2}+\nu  x_2 y_2^2 \sigma _{2,1}\right]
\end{multline}

\end{document}