% !TEX root = ../main.tex
\documentclass[../main.tex]{subfiles}
\begin{document}
В данной работа рассмотрены локальные угломерные навигационные системы и алгоритмы для нахождения пространственных координат и угловой ориентации подвижных объектов посредством азимутально-углометсного пеленгования радиоориентиров. В рамках данной работы были получены следующие результаты:
\begin{itemize}
    \item С помощью компьютерного моделирования была установлена неоднозначность решения задачи о нахождении расстояний от радиоориентиров до фазового центра бортовой пеленгаторной антенны посредством угловых измерений.
    \item Проведен математический анализ выбора наилучшего способа решения задачи нахождения расстояний и показано, что метод Ньютона решения систем нелинейных уравнений дает наиболее устойчивый и надежный способ решения.
    \item Рассмотрены различные конфигурации взаимного расположения подвижных объектов и радиоориентиров и связанные с ними вычислительные и математические задачи.
    \item Изучен вопрос о влиянии отраженного от Земли сигнала при малой высоте подвижных объектов.
\end{itemize}

Характер поставленной задачи требовал произведения компьютерного моделирования. Для этого были использованы пакет \texttt{Wolfram Mathematica} и язык программирования C++. В работе имеются приложения в виде распечаток отдельных фрагментов вычислительных программ.

Показано, что даже при использовании современных высокотехнологичных информационных систем и, в частности, пакетов символьных вычислений, полное решение изучаемой задачи невозможно без участия человеческого (естественного) интеллекта. По результатом данной работы было опубликовано две статьи~\cite{WMM:2018} и~\cite{WMMU:2019:IIS} в журналах, аккредитованных ВАК. Помимо этого, раздел~\ref{sec:lunx_simple} данной работы был представлен на XXV международной научно-технической конференции  <<Радиолокация, навигация, связь>>~\cite{WMMU:2019:RLNC}.



\end{document}