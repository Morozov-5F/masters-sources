% !TEX root = ../main.tex
\documentclass[../main.tex]{subfiles}
\begin{document}

% Как вариант -- расписать краткую историю навигации, про угломерную, про дальномерную и прочее. Потом вставка про ГЛОНАСС, плавный переход к патенту на определение ориентации. После этого расказать, что угломерная навигация никуда не делась, что практически удобно выстраивать локальные навигационные системы, которые не зависят от дорогостощих спутников в космосе. Такие системы есть, но все они требуют вспомогательного навигационного оборудования, чтобы определять местоположение. Тут-то и рассказ об актуальности работы, что наша работа пригодилась, мы рассматриваем принципиальную возможность реализации локальной навигационной системы, используя только пеленгование радиоориентиров.

Для определения координат и угловой ориентации подвижных объектов воздушного, морского или наземного базирования в настоящее время используются спутниковые радионавигационные системы ГЛОНАСС и GPS, основанные на излучении радиосигналов реперными источниками радиоизлучения (ИРИ), которые размещены на спутниках с известными координатами. В данной главе предлагается краткий обзор этих технологий, включая как концептуальную, так и математическую части системы.

\subsection{Глобальная Навигационная Спутниковая Система (ГЛОНАСС)}
Глобальная Навигационная Спутниковая Система (ГЛОНАСС) "--- спутниковая система навигации, разработанная в России. На сегодняшний день является второй в мире действующей глобальной спутниковой системой навигации. Другая система "--- американская NAVSTAR, также известная под названием GPS. Обе этих системы создавались исходя из определенных требований, таких как:
\begin{itemize}
    \item \textit{доступность} "--- степень вероятности работоспособности навигационной системы перед ее применением и в процессе применения;
    \item \textit{целостность} "--- степень вероятности обнаружения отказа системы в течении заданного времени;
    \item \textit{непрерывность обслуживания} "--- степень вероятности сохранения непрерывной работоспособности системы на заданном промежутке времени.
\end{itemize}
В данном контексте заданный промежуток времени "--- это отрезок времени, который наиболее важен с практической точки зрения, например, время захода самолета на посадку.

В ГЛОНАСС высокие эксплуатационные характеристики на структурном уровне достигаются путем совместного функционирования трех различных сегментов:
\begin{itemize}
    \item космического сегмента;
    \item сегмента управления;
    \item сегмента потребителей.
\end{itemize}

Кроме основных сегментов существует такое дополнение подсистема дифференциальной коррекции и мониторинга (СДКМ), благодаря которой предоставляется информация о целостности навигационного поля, корректируется информация о точных координатах спутников и частотно-временных параметрах системы, а также даются данные о величине вертикальной ионосферной задержке.

\begin{figure}[htbp]
    \begin{center}

    \fbox{\includegraphics[width=0.8\columnwidth]{1/glonass/pic1}}

    \caption{Орбитальная структура ГЛОНАСС}
    \label{fig:glonass:pic1}
    \end{center}
\end{figure}

Основной концепцией ГЛОНАСС является независимость и баззапросность навигационных определений. Под независимостью понимается то, что навигационные данные определяются целиком в аппаратуре абонента. Беззапросность означает, что все вычисления, необходимые для определения навигационных данных, производятся на основе пассивно принятых сигналов со спутников с заранее известными координатами. Это позволяет сделать абонентское оборудование компактными и практичным.

Точность определения и стабильность ГЛОНАСС в большей степени зависит от орбитального расположения спутников и параметров их сигналов. Полная группировка спутников системы ГЛОНАСС состоит из 24 основных и 2 резервных спутников, равномерно распределенных в трех орбитальных плоскостях (см. рис.~\ref{fig:glonass:pic1}). Орбитальные плоскости разнесены друг от друга на $\ang{120}$ и имеют условные номера 1, 2 и 3, возрастающие по направлению вращения Земли. Орбитальная структура сети спутников построена таким образом, что в каждой точке Земли наблюдается не менее четырех спутников. Следует отметить, что точность измерения координат может быть достигнута при наличии 7 спутников в каждой из орбитальных плоскостей. Система сохраняет полную функциональность при одновременном выходе из строя до 6 спутников (по два в каждой из плоскостей). Данная конфигурация, в отличии от системы GPS NAVSTAR, позволяет стабильно определять местоположение абонента в любой точке Земного шара, тогда как конфигурация орбитальной структуры системы NAVSTAR не позволяет вести стабильное определение навигационных данных в высоких северных и южных широтах.

Решение навигационной задачи, заключающейся в определении координат и поправки к шкале времени потребителя, в ГЛОНАСС может быть осуществлено несколькими способами. Эти способы справедливы не только для системы ГЛОНАСС, но для любой схожей спутниковой навигационной системы (например NAVSTAR).

\subsubsection{Дальномерный метод}
В случае, когда можно считать, что абонент находится на поверхности Земли, нахождение координат может производиться следующим образом. На основе задержки сигнала $\tau$ для $i$-го навигационно-космического аппарата (НКА) определяется расстояние $R_i = c\tau$, где $c$ "--- скорость света. TODO

\subsubsection{Псевдодальномерный метод}
TODO

\subsubsection{Разностно-дальномерный метод}
TODO

\end{document}