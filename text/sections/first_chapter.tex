% !TEX root = ../main.tex
\documentclass[../main.tex]{subfiles}
\begin{document}
\renewcommand{\figurename}{Рисунок}
% Как вариант -- расписать краткую историю навигации, про угломерную, про дальномерную и прочее. Потом вставка про ГЛОНАСС, плавный переход к патенту на определение ориентации. После этого расказать, что угломерная навигация никуда не делась, что практически удобно выстраивать локальные навигационные системы, которые не зависят от дорогостощих спутников в космосе. Такие системы есть, но все они требуют вспомогательного навигационного оборудования, чтобы определять местоположение. Тут-то и рассказ об актуальности работы, что наша работа пригодилась, мы рассматриваем принципиальную возможность реализации локальной навигационной системы, используя только пеленгование радиоориентиров.
В радионавигации при поиске координат и пространственной ориентации объекта вводятся понятия радионавигационного параметра, линий и поверхностей положения.

\textit{Радионавигационным параметром} (РНП) называют физическую величину, непосредственно измеряемую радионавигационной системой (расстояние, разность или сумма расстояний, угол).

\textit{Поверхность положения} "-— геометрическое место точек в пространстве, имеющих одно и то же значение РНП.

\textit{Линией положения} называют линию пересечения двух поверхностей положения. Местоположение объекта задается пересечением трех поверхностей положения или поверхности и линии положения.

В соответствии с видом непосредственно измеряемых координат различают три основных метода определения местоположения объекта: угломерный, дальномерный и разностно-дальномерный~\cite{KAZARINOV:2008}.

\subsection{Методы определения местоположения объекта}
Алгоритмы, предназначенные для определения местоположения и угловой ориентации объектов могут различаться, но базовые методы радионавигации остаются неизменными. Все три приведенных здесь метода известны уже довольно давно, и точность каждого из них совершенствуется с каждым годом. Новые методы, как правило, представляют собой вариации данных способов нахождения РНП. Предлагается рассмотреть эти методы подробнее.

\subsubsection{Угломерный метод}
Этот метод является самым старым, поскольку возможность определения направления прихода радиоволн была установлена А. С. Поповым еще в 1897 г. при проведении опытов по радиосвязи на Балтийском море.

При этом используются направленные свойства антенны при передаче или приеме радиосигнала. Существует два варианта построения угломерных систем: радиопеленгаторный и радиомаячный. В радиопеленгаторной системе направленной является антенна приемника (радиопеленгатора), а передатчик (радиомаяк) имеет ненаправленную антенну. При расположении радиопеленгатора (РП) и радиомаяка (РМ) в одной плоскости, например на поверхности Земли, направление на маяк характеризуется пеленгом. Если пеленг отсчитывают от географического меридиана (направление север—юг), то его называют истинным пеленгом или \textit{азимутом}. Часто азимутом считают угол в горизонтальной плоскости, отсчитанный от любого направления, принятого за нулевое. Определение направления производят в месте расположения приемника, который может быть как на Земле, так и на объекте. В первом случае пеленгование объекта осуществляют с Земли и при необходимости измеренное значение пеленга передают на объект (борт) по каналу связи. При расположении радиопеленгатора на объекте пеленг на радиомаяк измеряют непосредственно на борту. В качестве примера можно привести международную систему VOR для координации полетов или советскую навигационную систему <<Чайка>>.

В радиомаячной системе используют радиомаяк с направленной антенной и ненаправленный приемник. В этом случае в месте расположения приемника измеряют обратный пеленг относительно пулевого направления, проходящего через точку, в которой расположен радиомаяк. Часто применяют маяк с вращающейся антенной. В момент совпадения оси антенны с нулевым направлением (например, северным) вторая, ненаправленная, антенна РМ излучает специальный нулевой (северный) сигнал, который принимается приемником системы и является началом отсчета углов. Фиксируя момент совпадения оси вращающейся антенной маяка с направлением на приемник (например, по максимуму сигнала), можно найти обратный пеленг, который при равномерном вращении антенны маяка пропорционален промежутку времени между приемом нулевого сигнала и сигнала в момент пеленга. Так функционирует, например, советская радиотехническая система ближней навигации (РСБН).

В этом случае приемник упрощается, что важно при его расположении на борту. Поверхностью положения угломерной РНС является вертикальная плоскость, проходящая через линию пеленга.

При использовании наземных РП и РМ линией положения будет ортодромия "--- дуга большого круга, проходящего через пункты расположения РП и РМ. Она является линией пересечения поверхности положения с поверхностью Земли. Истинный пеленг "--- угол между меридианом и ортодромией. При расстояниях, малых по сравнению с радиусом Земли, ортодромия аппроксимируется отрезком прямой линии. Для определения местоположения РП необходим второй РМ. По двум пеленгам и можно найти местоположение РП как точку пересечения двух линий положения (двух ортодромий на земной поверхности). Если система расположена в пространстве, то для определения местоположения РП необходим третий радиомаяк. Каждая пара (РП-РМ) позволяет найти лишь поверхность положения, которая будет в данном случае плоскостью. При определении местоположения приемника предполагают, что координаты РМ известны.

\subsubsection{Дальномерный метод}
Этот метод основан на измерении расстояния $D$ между точками излучения и приема сигнала по времени его распространения между этими точками.

В радионавигации дальномеры работают с активным ответным сигналом, излучаемым антенной передатчика ответчика при приеме запросного сигнала. Если время распространения сигналов запроса $\tau_{\text{з}}$ и ответа $\tau_{\text{о}}$ одинаково, а время формирования ответного сигнала в ответчике пренебрежимо мало, то измеряемая запросчиком (радиодальномером) дальность
\begin{equation*}
    D = c \frac{\tau_{\text{з}} + \tau_{\text{о}}}{2},
\end{equation*}
где $c$ "--- скорость света. В качестве ответного может быть использован также и отраженный сигнал, что и делается при измерении дальности РЛС или высоты радиовысотомером.

Поверхностью положения дальномерной системы является поверхность шара радиусом $D$. Линиями положения на фиксированной плоскости либо сфере (например, на поверхности Земли) будут окружности, поэтому иногда дальномерные системы называют круговыми. Нахождение координат сводится к решению следующей системы уравнений:
\begin{equation*}
    \begin{cases}
        \left(x_1 - x\right)^2 + \left(y_1 - y\right)^2 + \left(z_1 - z\right)^2 = D_1^2, \\
        \left(x_2 - x\right)^2 + \left(y_2 - y\right)^2 + \left(z_2 - z\right)^2 = D_2^2, \\
        \hfil \cdots \\
        \left(x_n - x\right)^2 + \left(y_n - y\right)^2 + \left(z_n - z\right)^2 = D_n^2, \\
    \end{cases}
\end{equation*}
где $D_i$ "--- расстояние между приемником и $i$-м передатчиком, $x_i$, $y_i$, $z_i$ "--- известные на момент измерения координаты $i$-го передатчика, $x$, $y$, $z$ "--- неизвестные координаты приемника, $n$ "--- число передатчиков, $i = 1,\cdots, n$. Так как окружности пересекаются в двух точках то возникает двузначность отсчета, для исключения которой применяют дополнительные средства ориентирования, точность которых может быть невысокой, но достаточной для достоверного выбора одной из двух точек пересечения. Поскольку измерение времени задержки сигнала может производиться с малыми погрешностями, дальномерные РНС позволяют найти координаты с высокой точностью.

\subsubsection{Разностно-дальномерный метод}
С помощью приемоиндикатора, расположенного на борту объекта, определяют разность времени приема сигналов от передатчиков двух опорных станций: $A$ и $B$. Станцию $А$ называют ведущей, так как с помощью ее сигналов осуществляется синхронизация работы ведомой станции $В$. Измерение разности расстояний, пропорциональной временному сдвигу сигналов от станции $А$ и $В$, позволяет найти лишь поверхность положения, соответствующую этой разности и имеющую форму гиперболоида. Если приемоиндикатор и станции $А$ и $В$ расположены на поверхности Земли, то измерение $\Delta D = D_B - D_A$ позволяет получить линию положения на земной поверхности в виде гиперболы с $\Delta D = \textrm{const}$.

Для двух станций можно построить семейство гипербол с фокусами в точках расположения станций $А$ и $В$. Расстояние между станциями называют базой. Для заданной базы семейство гипербол наносят на карту заранее оцифровывают. Однако одна пара станций позволяет определить лишь линию положения, на которой расположен объект. Для нахождения его местоположения необходима вторая пара станций, база которой $d_2$ должна быть расположена под углом к базе $d_1$ первой пары. Обычно ведущая станция $А$ является общей и синхронизирует работу обеих ведомых станций $B_1$ и $B_2$. Сетка линий положения такой системы образуется двумя семействами пересекающихся гипербол, позволяющих найти местоположение приемоиндикатора, расположенного на борту объекта.

Точность разностно-дальномерной системы выше точности угломерной и приближается к точности дальномер-ной. Но основным ее преимуществом является неограниченная пропускная способность, так как наземные станции могут обслуживать неограниченное число ПИ, находящихся в пределах дальности действия системы, поскольку на борту определяющегося объекта нет необходимости иметь передатчик, как в дальномерной системе. Следует заметить, что асимптотами гипербол являются прямые линии, проходящие через центр базы каждой пары станций системы Таким образом, на расстояниях, в несколько раз превышающих длину базы, линии положения вырождаются в прямые, в результате чего разностно-дальномерная система может быть использована как угломерная. Примером такой системы может служить навигационная система LORAN, разработанная в США.

Для определения координат и угловой ориентации подвижных объектов воздушного, морского или наземного базирования в настоящее время используются спутниковые радионавигационные системы ГЛОНАСС и GPS, основанные на излучении радиосигналов реперными источниками радиоизлучения (ИРИ), которые размещены на спутниках с известными координатами. В данной главе предлагается краткий обзор этих технологий, включая как концептуальную, так и математическую части системы.

\subsection{Глобальная Навигационная Спутниковая Система (ГЛОНАСС)}
Глобальная Навигационная Спутниковая Система (ГЛОНАСС) "--- спутниковая система навигации, разработанная в России. На сегодняшний день является второй в мире действующей глобальной спутниковой системой навигации. Другая система "--- американская NAVSTAR, также известная под названием GPS. Обе этих системы создавались исходя из определенных требований, таких как:
\begin{itemize}
    \item \textit{доступность} "--- степень вероятности работоспособности навигационной системы перед ее применением и в процессе применения;
    \item \textit{целостность} "--- степень вероятности обнаружения отказа системы в течении заданного времени;
    \item \textit{непрерывность обслуживания} "--- степень вероятности сохранения непрерывной работоспособности системы на заданном промежутке времени.
\end{itemize}
В данном контексте заданный промежуток времени "--- это отрезок времени, который наиболее важен с практической точки зрения, например, время захода самолета на посадку.

В ГЛОНАСС высокие эксплуатационные характеристики на структурном уровне достигаются путем совместного функционирования трех различных сегментов~\cite{YATSENKOV:2005}:
\begin{itemize}
    \item космического сегмента;
    \item сегмента управления;
    \item сегмента потребителей.
\end{itemize}

\begin{figure}[tb]
    \begin{center}

    \fbox{\includegraphics[width=0.8\columnwidth]{1/glonass/pic1}}

    \caption{Орбитальная структура ГЛОНАСС}
    \label{fig:glonass:pic1}
    \end{center}
\end{figure}

Кроме основных сегментов существует такое дополнение подсистема дифференциальной коррекции и мониторинга (СДКМ), благодаря которой предоставляется информация о целостности навигационного поля, корректируется информация о точных координатах спутников и частотно-временных параметрах системы, а также даются данные о величине вертикальной ионосферной задержке.

Основной концепцией ГЛОНАСС является независимость и баззапросность навигационных определений. Под независимостью понимается то, что навигационные данные определяются целиком в аппаратуре абонента. Беззапросность означает, что все вычисления, необходимые для определения навигационных данных, производятся на основе пассивно принятых сигналов со спутников с заранее известными координатами. Это позволяет сделать абонентское оборудование компактными и практичным.

Точность определения и стабильность ГЛОНАСС в большей степени зависит от орбитального расположения спутников и параметров их сигналов. Полная группировка спутников системы ГЛОНАСС состоит из 24 основных и 2 резервных спутников, равномерно распределенных в трех орбитальных плоскостях (см. рис.~\ref{fig:glonass:pic1}). Орбитальные плоскости разнесены друг от друга на $\ang{120}$ и имеют условные номера 1, 2 и 3, возрастающие по направлению вращения Земли. Орбитальная структура сети спутников построена таким образом, что в каждой точке Земли наблюдается не менее четырех спутников. Следует отметить, что точность измерения координат может быть достигнута при наличии 7 спутников в каждой из орбитальных плоскостей. Система сохраняет полную функциональность при одновременном выходе из строя до 6 спутников (по два в каждой из плоскостей). Данная конфигурация, в отличии от системы GPS NAVSTAR, позволяет стабильно определять местоположение абонента в любой точке Земного шара, тогда как конфигурация орбитальной структуры системы NAVSTAR не позволяет вести стабильное определение навигационных данных в высоких северных и южных широтах.

Решение навигационной задачи, заключающейся в определении координат и поправки к шкале времени потребителя, в ГЛОНАСС может быть осуществлено дальномерным или разностно-дальномерным способами, описанными в предыдущих разделах. Это справедливо не только для системы ГЛОНАСС, но для любой схожей спутниковой навигационной системы (например NAVSTAR)~\cite{GOST-GLONASS:2009}.

\subsubsection{Определение координат и угловой ориентации с помощью ГЛОНАСС}
Один из способов определения координат и угловой ориентации с помощью ГЛОНАСС описан в~\cite{STEPANOV:2006}. На объекте устанавливают антенны $A$, $B$, $C$ и $D$, которые принимают навигационные сигналы от спутников $S_1$ и $S_2$ (см. рис~\ref{fig:glonass:pic2}). Антенны расположены таким образом, что они образуют пеленгационные пары $AB$ и $CD$ с неколлинеарными базами.

\begin{figure}[tb]
    \begin{center}

    \fbox{\includegraphics[width=0.6\columnwidth]{1/glonass/pic2}}

    \caption{Схематическое изображение навигационной системы, способной определять координаты и угловую ориентацию с помощью ГЛОНАСС.}
    \label{fig:glonass:pic2}
    \end{center}
\end{figure}

Координаты объекта определяются одним из стандартных для ГЛОНАСС методов (к примеру, дальномерным), а угловая ориентация определяется следующим образом:
\begin{enumerate}
    \item с помощью разностей фаз определяются косинусы углы визирования $\alpha_1$, $\alpha_2$, $\beta_1$ и $\beta_2$;
    \item определяются координаты спутников $S_1$ и $S_2$ в базовой системе координат и векторы $\overline{OS_1}$ и $\overline{OS_2}$, выходящие из фазового центра $O$ антенной решетки $ABCD$ и приходящие в спутники $S_1$ и $S_2$;
    \item составляется система уравнений из углов визирования и векторов-направлений для нахождения координат векторов $\overline{AB}$ и $\overline{CD}$.
    \item находится векторное произведение векторов $\overline{AB}$ и $\overline{CD}$, которое обозначается через $\overline{\mathbf{a}}$.
    \item так как векторы  $\overline{AB}$, $\overline{CD}$ и $\overline{\mathbf{a}}$ совпадают с осями связанной с объектом системы координат, то они определяют ориентацию объекта относительно базовой системы координат;
    \item составляется матричное уравнение для нахождения матрицы перехода из базовой системы координат в связанную, из которой могут быть найдены углы курса, крена и тангажа.
\end{enumerate}
Таким образом, для определения координат в такой системе требуется как минимум три спутника в зоне видимости, а для определения угловой ориентации "--- не менее двух. При этом, заявлена погрешность определения угловой ориентации в $\pm 3$ угловых минуты. Система, представленная в работе~\cite{STEPANOV:2006} предоставляет хорошую точность, но обладает несколькими недостатками "--- требованием на синхронность излучения радиоориентиров и сложностью технической реализации. Есть и другие работы на эту тему (~\cite{KORNEV:2016}), в которых сохраняются указанные недостатки.

В известных работах по радионавигации исследованы различные способы определения координат в пространстве подвижного объекта угломерным методом путем радиопеленгования с борта подвижного объекта, оснащенного бортовыми автономными навигационными датчиками и системами (инерциальными, геомагнитными), радиоориентиров, реализуемым без предъявления требований к синхронности излучения радиосигналов радиоориентирами~\cite{BBELAVIN:1977}. Однако такие системы без помощи вспомогательных навигационных приборов способны определять либо координаты, либо угловую ориентацию подвижных объектов. Возможности одновременного и однозначного определения координат и угловой ориентации в пространстве подвижного объекта путем азимутально-угломестного радиопеленгования (определения азимута и угла места источников радиоизлучения) с борта подвижного объекта радиоориентиров без использования вспомогательной информации от автономных навигационных датчиков и систем в известных работах по радионавигации не исследованы~\cite{VINOGRADOV:2016}

Согласно~\cite{REPORT:2015}, существует потребность в разработке таких систем. При этом возникает необходимость решения задачи определения условий однозначности определения координат и угловой ориентации в пространстве подвижного объекта, оснащенного бортовой пелангаторной антенной (БПА), определяющих минимально возможное число и ограничения на взаимное пространственное расположение БПА и радиоориентиров.

\end{document}