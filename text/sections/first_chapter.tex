% !TEX root = ../main.tex
\documentclass[../main.tex]{subfiles}
\begin{document}

% Как вариант -- расписать краткую историю навигации, про угломерную, про дальномерную и прочее. Потом вставка про ГЛОНАСС, плавный переход к патенту на определение ориентации. После этого расказать, что угломерная навигация никуда не делась, что практически удобно выстраивать локальные навигационные системы, которые не зависят от дорогостощих спутников в космосе. Такие системы есть, но все они требуют вспомогательного навигационного оборудования, чтобы определять местоположение. Тут-то и рассказ об актуальности работы, что наша работа пригодилась, мы рассматриваем принципиальную возможность реализации локальной навигационной системы, используя только пеленгование радиоориентиров.

В радионавигации при поиске координат и пространственной ориентации объекта вводятся понятия радионавигационного параметра, линий и поверхностей положения.

\textit{Радионавигационным параметром} (РНП) называют физическую величину, непосредственно измеряемую радионавигационной системой (расстояние, разность или сумма расстояний, угол).

\textit{Поверхность положения} "-— геометрическое место точек в пространстве, имеющих одно и то же значение РНП.

\textit{Линией положения} называют линию пересечения двух поверхностей положения. Местоположение объекта задается пересечением трех поверхностей положения или поверхности и линии положения.

В соответствии с видом непосредственно измеряемых координат различают три основных метода определения местоположения объекта: угломерный, дальномерный и разностно-дальномерный. Широко применяют также комбинированный угломерно-дальномерный метод.

\subsection{Угломерный метод}

\subsection{Дальномерный метод}

\subsection{Разностно-дальномерный метод}

Для определения координат и угловой ориентации подвижных объектов воздушного, морского или наземного базирования в настоящее время используются спутниковые радионавигационные системы ГЛОНАСС и GPS, основанные на излучении радиосигналов реперными источниками радиоизлучения (ИРИ), которые размещены на спутниках с известными координатами. В данной главе предлагается краткий обзор этих технологий, включая как концептуальную, так и математическую части системы.

\subsection{Глобальная Навигационная Спутниковая Система (ГЛОНАСС)}
Глобальная Навигационная Спутниковая Система (ГЛОНАСС) "--- спутниковая система навигации, разработанная в России. На сегодняшний день является второй в мире действующей глобальной спутниковой системой навигации. Другая система "--- американская NAVSTAR, также известная под названием GPS. Обе этих системы создавались исходя из определенных требований, таких как:
\begin{itemize}
    \item \textit{доступность} "--- степень вероятности работоспособности навигационной системы перед ее применением и в процессе применения;
    \item \textit{целостность} "--- степень вероятности обнаружения отказа системы в течении заданного времени;
    \item \textit{непрерывность обслуживания} "--- степень вероятности сохранения непрерывной работоспособности системы на заданном промежутке времени.
\end{itemize}
В данном контексте заданный промежуток времени "--- это отрезок времени, который наиболее важен с практической точки зрения, например, время захода самолета на посадку.

В ГЛОНАСС высокие эксплуатационные характеристики на структурном уровне достигаются путем совместного функционирования трех различных сегментов:
\begin{itemize}
    \item космического сегмента;
    \item сегмента управления;
    \item сегмента потребителей.
\end{itemize}

\begin{figure}[tb]
    \begin{center}

    \fbox{\includegraphics[width=0.8\columnwidth]{1/glonass/pic1}}

    \caption{Орбитальная структура ГЛОНАСС}
    \label{fig:glonass:pic1}
    \end{center}
\end{figure}

Кроме основных сегментов существует такое дополнение подсистема дифференциальной коррекции и мониторинга (СДКМ), благодаря которой предоставляется информация о целостности навигационного поля, корректируется информация о точных координатах спутников и частотно-временных параметрах системы, а также даются данные о величине вертикальной ионосферной задержке.

Основной концепцией ГЛОНАСС является независимость и баззапросность навигационных определений. Под независимостью понимается то, что навигационные данные определяются целиком в аппаратуре абонента. Беззапросность означает, что все вычисления, необходимые для определения навигационных данных, производятся на основе пассивно принятых сигналов со спутников с заранее известными координатами. Это позволяет сделать абонентское оборудование компактными и практичным.

Точность определения и стабильность ГЛОНАСС в большей степени зависит от орбитального расположения спутников и параметров их сигналов. Полная группировка спутников системы ГЛОНАСС состоит из 24 основных и 2 резервных спутников, равномерно распределенных в трех орбитальных плоскостях (см. рис.~\ref{fig:glonass:pic1}). Орбитальные плоскости разнесены друг от друга на $\ang{120}$ и имеют условные номера 1, 2 и 3, возрастающие по направлению вращения Земли. Орбитальная структура сети спутников построена таким образом, что в каждой точке Земли наблюдается не менее четырех спутников. Следует отметить, что точность измерения координат может быть достигнута при наличии 7 спутников в каждой из орбитальных плоскостей. Система сохраняет полную функциональность при одновременном выходе из строя до 6 спутников (по два в каждой из плоскостей). Данная конфигурация, в отличии от системы GPS NAVSTAR, позволяет стабильно определять местоположение абонента в любой точке Земного шара, тогда как конфигурация орбитальной структуры системы NAVSTAR не позволяет вести стабильное определение навигационных данных в высоких северных и южных широтах.

Решение навигационной задачи, заключающейся в определении координат и поправки к шкале времени потребителя, в ГЛОНАСС может быть осуществлено дальномерным или разностно-дальномерным способами. Это справедливо не только для системы ГЛОНАСС, но для любой схожей спутниковой навигационной системы (например NAVSTAR).

\subsubsection{Определение координат и угловой ориентации с помощью ГЛОНАСС}
Один из способов определения координат и угловой ориентации с помощью ГЛОНАСС описан в~\cite{patent}. На объекте устанавливают антенны $A$, $B$, $C$ и $D$, которые принимают навигационные сигналы от спутников $S_1$ и $S_2$ (см. рис~\ref{fig:glonass:pic2}). Антенны расположены таким образом, что они образуют пеленационные пары $AB$ и $CD$ с неколинеарными базами.

\begin{figure}[tb]
    \begin{center}

    \fbox{\includegraphics[width=0.6\columnwidth]{1/glonass/pic2}}

    \caption{Схематическое изображение навигацонной системы, способной определять координаты и угловую ориентацию с помощью ГЛОНАСС.}
    \label{fig:glonass:pic2}
    \end{center}
\end{figure}

Координаты объекта определяются одним из стандартных для ГЛОНАСС методов (к примеру, дальномерным), а угловая ориентация определяется следующим образом:
\begin{enumerate}
    \item с помощью разностей фаз определяются косинусы углы визирования $\alpha_1$, $\alpha_2$, $\beta_1$ и $\beta_2$;
    \item определяются координаты спутников $S_1$ и $S_2$ в базовой системе координат и векторы $\overline{OS_1}$ и $\overline{OS_2}$, выходящие из фазового центра $O$ антенной решетки $ABCD$ и приходящие в спутники $S_1$ и $S_2$;
    \item составляется система уравнений из углов визирования и векторов-направлений для нахождения координат векторов $\overline{AB}$ и $\overline{CD}$.
    \item находится векторное произведение векторов $\overline{AB}$ и $\overline{CD}$, которое обозначается через $\overline{\mathbf{a}}$.
    \item так как векторы  $\overline{AB}$, $\overline{CD}$ и $\overline{\mathbf{a}}$ совпадают с осями связанной с объектом системы координат, то они определяют ориентацию объекта относительно базовой системы координат;
    \item составляется матричное уравнение для нахождения матрицы перехода из базовой системы координат в связанную, из которой могут быть найдены углы курса, крена и тангажа.
\end{enumerate}
Таким образом, для определения координат в такой системе требуется как минимум три спутника в зоне видимости, а для определения угловой ориентации "--- не менее двух. При этом, зяявлена погрешность определения угловой ориентации в $\pm 3$ угловых минуты.


\end{document}