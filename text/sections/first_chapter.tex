% !TEX root = ../main.tex
\documentclass[../main.tex]{subfiles}
\begin{document}

Для определения координат и угловой ориентации подвижных объектов воздушного, морского или наземного базирования в настоящее время используются спутниковые радионавигационные системы ГЛОНАСС и GPS, основанные на излучении радиосигналов реперными источниками радиоизлучения (ИРИ), которые размещены на спутниках с известными координатами. В данной главе предлагается краткий обзор этих технологий, включая как концептуальную, так и математическую части системы.

\subsection{Глобальная Навигационная Спутниковая Система (ГЛОНАСС)}
Глобальная Навигационная Спутниковая Система (ГЛОНАСС) "--- спутниковая система навигации, разработанная в России. На сегодняшний день является второй в мире действующей глобальной спутниковой системой навигации. Другая система "--- американская NAVSTAR, также известная под названием GPS. Обе этих системы создавались исходя из определенных требований, таких как:
\begin{itemize}
    \item \textit{доступность} "--- степень вероятности работоспособности навигационной системы перед ее применением и в процессе применения;
    \item \textit{целостность} "--- степень вероятности обнаружения отказа системы в течении заданного времени;
    \item \textit{непрерывность обслуживания} "--- степень вероятности сохранения непрерывной работоспособности системы на заданном промежутке времени.
\end{itemize}
В данном контексте заданный промежуток времени "--- это отрезок времени, который наиболее важен с практической точки зрения, например, время захода самолета на посадку.

В ГЛОНАСС высокие эксплуатационные характеристики на структурном уровне достигаются путем совместного функционирования трех различных сегментов:
\begin{itemize}
    \item космического сегмента;
    \item сегмента управления;
    \item сегмента потребителей.
\end{itemize}

Кроме основных сегментов существует такое дополнение подсистема дифференциальной коррекции и мониторинга (СДКМ), благодаря которой предоставляется информация о целостности навигационного поля, корректируется информация о точных координатах спутников и частотно-временных параметрах системы, а также даются данные о величине вертикальной ионосферной задержке.

\end{document}