\documentclass[a4paper, 14pt]{extarticle}

\usepackage[T2A]{fontenc}
\usepackage[utf8]{inputenc}
\usepackage[english,russian]{babel}

\usepackage{cmap}
\usepackage{amsmath}

\usepackage[margin=2cm]{geometry}

\usepackage{titlesec}
\titleformat{\section}{\normalfont\Large\bfseries}{Слайд~\thesection:}{0.5em}{}{}

\title{Текст для RLNC}
\date{}

\begin{document}

\maketitle

\section{Приветствие}
Здравствуйте, уважаемые слушатели, меня зовут Морозов Евгений и сегодня
я хотел бы немного поговорить об угломерной навигации.

\section{Введение}
Я предлагаю начать с того, как определяется пространственная ориентация подвижных
объектов в современных системах навигации. Как правило, для этого используются системы
вроде GPS или ГЛОНАСС. В таких системах радиоориентиры (РО) расположены на спутниках,
а прием сигналов с них ведется в нескольких точках, находящихся на подвижном объекте.
При этом, для определения координат используется дальномерный подход (измеряются задержки приемов сигналов),
а для определения угловой ориентации "--- угломерный (измеряются углы наблюдения РО).
Условием работы данных систем является синхронное излучение радиосигналов РО.

Помимо этих систем, существуют угломерные системы, не накладывающие требования синхронности
на излучения радиосигналов РО. Однако, возможности систем, которые для определения пространственной
ориентации объектов используют только азимутально-угломестное радиопеленгование РО без
использования дополнительных навигационных датчиков, пока неизучены. Именно на таких
системах установлен главный фокус данной работы.

\section{Постановка задачи}
Перейдем к формальной постановке задачи. Пусть в пространстве расположены несколько
радиориентиров с известными координатами. Подвижный объект (воздушного, наземного или
морского базирования) оснащен Бортовой Пеленгаторной Антенной, которая способна измерять
азимут и угол места каждого реперного источника в связной с подвижным объектом системой
координат. Центром объекта будем считать фазовый центр БПА. Необходимо однозначно найти
координаты и угловую ориентацию (углы курса, крена и тангажа) подвижного объекта, используя
минимальное число радиоориентиров.

\section{Постановка задачи (продолжение)}
Попробуем определить минимально возможное число РО, при котором задача решается теоретически.
Так как координаты и угловая ориентация определяется шестью параметрами (три компоненты координат
и три угла), то для их определения нужно иметь как минимум шесть независимых измеряемых величин.
Таковыми величинами являются пары азимутов и углов места РО. Таким образом, минимальное число
радиоориентиров "--- три.

\section{Постановка задачи (картинка)}
На рисунке видно схему расположения радиоориентиров и подвижного объекта. Здесь через $\ell{i}$
обозначены длины боковых ребер $M_0M_i$, через $d_{ij}$ "--- длины ребер $M_iM_j$ основания, а через
$\alpha_{ij}$ "--- плоские углы при вершине $M_0$.

\section{Схема решения}
Решение задачи можно разбить на три основных этапа:
\begin{itemize}
    \item Найти расстояния $\ell_i$ от ФЦ БПА до РО (длины боковых ребер)
    \item Определить координаты подвижного объекта
    \item Определить матрицу поворота СК воздушного объекта и связанную с ней угловую
          ориентацию объекта.
\end{itemize}

Отметитм, что ключевым является именно первый этам, поскольку второй и третий этапы
являются стандартными задачами в навигации. Еще раз хочу отметить, что я буду рассказывать больше
о математической стороне этой задачи, избегая техническую составляющую.

\section{Схема решения (система уравнений)}
Для нахождения длинн необходимо решить систему уравнений (1), кравнения которой
представляют собой теорему косинусов для боковых граней тетраэдра. Система является
нелинейной, однородной второго порядка. Несмотря на внешнюю простоту системы, имеются
некоторые особенности.

\section{Схема решения}
Я попробую проиллюстрировать эти особенности. Представим, что изначально подвижный объект находится в
центре масс треугольника основания и начинает подъем вертикально вверх относительно плоскости основания.
До определенной высоты с решением не возникает проблем, длинны находятся однозначно. При достижении <<первой
критической высоты>>, у системы появляется второе решение, которое соответствует некторой вершине треугольника.
При дальнейшем подъеме это паразитное решение поднимается вместе с объектом. Всего таких критических высот несколько,
система может иметь от одного до четырех решений, в зависимости от того, где и как далеко от плоскости основания
находится подвижный объект.

Если останется время или если будет интерес, я попробую в конце объяснить, почему так происходит.

\section{Особенности решения. Вступление}
Так как же улучшить характеристики системы уравнений? Можно попробовать добавить больше РО, например,
четыре или пять. Однако, вычислительные эксперименты, произведенные в пакете Mathematica, показали, что
увеличение числа РО не улучшает разрешимость "--- система уравнений становится переорпеделенной и
однозначно решаемой только в стерильных условиях. Кстати говоря, если будет время, я продемонстрирую
иллюстрации того, как ведут себя решения для трех точек "--- зрелище довольно занимательное.

Можно попробовать найти близкое к истиному решению системы методом наименьших квадратов, но и этот путь
оказался непрактичным "--- он приводит только к увеличению степени уравнений и к усложнению вычислений.

\section{Особенности решения. Ньютон}
Оказалось, что благодаря нескольким факторам, систему уравнений (1) можно решать методом Ньютона. В связи с
тем, что подвижный объект способен <<помнить>> свое местоположение в предыдущий момент времени, мы имеем достаточно
хорошее начальное приближение (главное "--- производить рассчеты достаточно часто, чтобы объет не успел значительно
изменить местоположение). Как показал вычислительный эксперимент, данный метод сходится для решения достаточно быстро
(5-10 итераций достаточно для точности в 10 знаков после запятой). И, что немаловажно, в методе используются
только арифметические операции.

\section{Особенности решения. Второй этап}
Что ж, одной проблемой меньше! После определения длин, нас ждет второй этап "--- определение координат
подвижного объекта в глобальной системе координат. Это можно сделать более-менее стандартными методами,
но можно немного облегчить себе жизнь и решить систему уравнений на слайде в системе координат, связанной
с плоскостью, в которой лежат радиоориентиры. В таком случае, квадратичная система превращается в линейную
и решается методом Крамера. После достаточно перемножить матрицу перехода и полученные координаты, чтобы
координаты в глобальной СК.

\section{Оценка погрешностей}
Вот мы и добрались до вопроса, который так и напрашивается "--- какова погрешность определения пространственной
ориентации в данной системе? Я с радостью дал бы ответ на этот вопрос, но есть несколько прнципиальных моментов,
которые не дают мне дать полный ответ:
\begin{enumerate}
    \item Система уравнений имеет несколько решений
    \item Решается итерационным методом
    \item Точность решения сильно зависит от конфигурации
    \item Подвижный объект может находиться в точках бифуркации
\end{enumerate}
И вообще, что в таком случае называть погрешностью?

Хочется отметить, что буквально вчера мы узнали (благодаря этой конференции) о работе
\textsc{МАКСИМАЛЬНО ПРАВДОПОДОБНЫЙ АЛГОРИТМ ОПРЕДЕЛЕНИЯ КООРДИНАТ И УГЛОВОЙ ОРИЕНТАЦИИ
БОРТОВОЙ ПЕЛЕНГАТОHНОЙ АНТЕННЫ ПО РЕЗУЛЬТАТАМ РАДИОПЕЛЕНГОВАНИЯ РЕПЕРНЫХ ИСТОЧНИКОВ РАДИОИЗЛУЧЕНИЯ}
под авторством Виноградова, Вострова и Дмитриева, в которой приходят к очень похожим выводам,
что и мы, только совершенно другим методом. Кстати говоря, в этой работе произведена оценка погрешностей,
которая подтверждает сказанное "--- однозначно дать ответ на ворос <<какова погрешность?>> достаточно сложно.

\section{Дальнейшие планы}
Пару слов о том, в каком направлении мы планируем двигаться. В данный момент идут исследования
других конфигураций локальных угломерных навигационных систем, в частности, с подключением
дополнительных подвижных объектов или включения <<активных>> наземных ориентиров, способных производить
радиопеленгацию подвижных объектов с земли и обмениваться данными с подвижными объектами.

Отметим, что пока представленная система "--- единственная, в которой аппарат способен работать в пассивном режиме,
не обмениваясь информацией с другими РО.

\end{document}