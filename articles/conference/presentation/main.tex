%!TEX TS-program = xelatex
%!TEX encoding = UTF-8 Unicode
\documentclass[russian,hyperref={unicode}]{beamer}
% Other packages
\usepackage{amsmath, amssymb, amsthm}
% \usepackage{beamerthemesplit} // Activate for custom appearance

% Theme selection
\usetheme[sectionpage=none, numbering=fraction]{metropolis}
% Locale packages
\usepackage{polyglossia}
\setmainlanguage{russian}
\setotherlanguage{english}
\setkeys{russian}{babelshorthands=true}

% \setsansfont[BoldFont={Fira Sans SemiBold}]{Fira Sans Book}
% \newfontfamily{\cyrillicfonttt}{Fira Mono}
\setsansfont[BoldFont={FiraSans-Bold.ttf}]{FiraSans-Light.ttf}
\setmonofont{FiraMono-Regular.ttf}

\newcommand{\backupbegin}{
   \newcounter{finalframe}
   \setcounter{finalframe}{\value{framenumber}}
}
\newcommand{\backupend}{
   \setcounter{framenumber}{\value{finalframe}}
}

\title{Исследование симметрий алгебраических уравнений}
\institute
{
	Воронежский Государственный Университет \\
	Факультет Компьютерных Наук \\
	Кафедра Цифровых Технологий
}
\author
{
	Выполнил: студент 4 курса Морозов Евгений \\
	Руководитель: А.В. Лобода
}
%\date{}

\begin{document}

\frame{\titlepage}

\section{Постановка задачи}
\frame
{
  \frametitle{Постановка задачи}
  Дано семейство вещественных гиперповерхностей в $\mathbb{C}^3$:
  \begin{equation}\label{eq:family}
    v x_2 + Q(x_1, y_1, x_2, y_2) = x_2 (\mu\,{x_2}^2 + \nu\,{y_2}^2),
  \end{equation}
  $x_1 = \Re(z_1)$, $y_1 = \Im(z_1)$, $x_2 = \Re(z_2)$, $y_2 = \Im(z_2)$, $u = \Re(w)$,
  $v = \Im(w)$, $\mu$, $\nu \in \mathbb{R}$ и одновременно не равны нулю, $Q(x_1, y_1,
  x_2, y_2)$ "--- некоторая квадратичная форма.

  \begin{itemize}
	\item \textbf{Симметрия поверхности} "--- любое аффинное преобразование, сохраняющее эту поверхность.
    \item \textbf{Задача состоит} в изучении групп аффинных преобразований, сохраняющих поверхности из семейства~(\ref{eq:family}), и их размерностей.
	\item Задача изучается с точностью до аффинных преобразований исходных объектов.
  \end{itemize}
}
\section{Метод решения}
\frame
{
  \frametitle{Метод решения}
  Любая поверхность из~(\ref{eq:family}) сохраняется сдвигом по переменной $u = \Re(w)$. Следовательно, размерность каждой из изучаемых групп преобразований не меньше 1.

  Пусть $F_t$ "--- однопараметрическая группа аффинных преобразований, сохраняющих уравнение поверхности $\Phi = 0$, $F_0 = \mathsf{Id}$.

  Факт сохранения поверхности $M$ из~(\ref{eq:family}) группой $F_t$:
  \begin{equation}\label{eq:preservation}
  	F_t(\Phi) = \Phi(z, \overline{z}, v) \cdot \psi(z, \overline{z}, w, t),
  \end{equation}
  где $\psi(z, \overline{z}, w, t)$ "--- некоторая ненулевая функция, $\Phi(z, \overline{z}, v) = v x_2 + Q(x_1, y_1, x_2, y_2) - x_2 (\mu\,{x_2}^2 + \nu\,{y_2}^2)$ "--- определяющая функция поверхности $M$.
}
\frame
{
  \frametitle{Метод решения (продолжение)}
  Продифференцируем~(\ref{eq:preservation}) по $t$ в точке $t=0$ и сузим результат на поверхность $M$:
  \begin{equation}\label{eq:main_eq}
  	\left.\left\{ \left.\frac{\mathsf d}{\mathsf d t}F_t(\Phi) \right|_{t=0}\right\}\right|_{M} \equiv 0
  \end{equation}

  Далее перейти к однородной СЛАУ, ранг матрицы которой определяет искомую размерность группы преобразований:
  $$
  	\dim G = 24 - \mathsf{rank}\ W.
  $$
  Таким образом, исследование размерности группы аффинных преобразований, сохраняющих поверхность $M$, сводится к исследованию ранга полученной матрицы.
}
\section{Частный случай}
\frame
{
	\frametitle{Частный случай}
	Детально был рассмотрен случай, в котором квадратичная форма $Q$ не зависит от $x_2$
   и $y_2$:
	\begin{equation}\label{eq:surface_partial}
	v x_2 + k_1 {x_1}^2 + k_2 x_1 y_1 + k_3 {y_1}^2 = x_2 (\mu\,{x_2}^2 + \nu\,{y_2}^2).
	\end{equation}

	В данном случае возникает СЛАУ из 57 уравнений относительно 24 неизвестных.

	\textbf{Замечание:}
	за счёт аффинных преобразований можно сократить число параметров в~(\ref{eq:surface_partial}).
}
\frame
{
	\frametitle{Частный случай: оценка размерности}
	\textbf{Теорема 1:}
	размерность группы Ли аффинных преобразований, сохраняющих любую
	поверхность вида~(\ref{eq:surface_partial}), удовлетворяет неравенствам
	$ 3 \le \dim G \le 5$, причем:
	\begin{itemize}
		\item $\dim G = 3$ достигается на поверхностях вида
		\begin{align}\label{eq:dim3}
			v x_2 = {x_1}^2 + A x_1 y_1 + B {y_1}^2 + &x_2 (\mu\,{x_2}^2 + \nu\, {y_2}^2) \nonumber\\
			&(A \ne 2 \sqrt{B},\ B\ne 1);
		\end{align}
		\item $\dim G = 4$ достигается на поверхностях вида
		\begin{equation}\label{eq:dim4}
			v x_2 = {|z_1|}^2 + x_2 (\mu\,{x_2}^2 + \nu\,{y_2}^2);
		\end{equation}
		\item $\dim G = 5$ достигается на аффинно-однородных~\cite{arxiv} поверхноcтях вида
		\begin{equation}\label{eq:dim5}
			v x_2 = {x_1}^2 + x_2 (\mu\,{x_2}^2 + \nu\,{y_2}^2).
		\end{equation}
	\end{itemize}
}
\frame
{
	\frametitle{Частный случай: допустимые движения}
	Для произвольной поверхности вида~(\ref{eq:surface_partial}) существуют три основных вида движения:
	\begin{itemize}
		\item \emph{сдвиг} по переменной $u$: $w \to w^{\ast} + t$;
		\item \emph{масштабирование}: $z_1 \to e^{3t} \cdot z_1^{\ast}$, $z_2 \to e^{2t} \cdot z_2^{\ast}$, $w \to e^{4t} \cdot w^{\ast}$;
		\item \emph{поворот со сдвигом ("скользящий поворот")}: $z_2 \to z_2^{\ast} + i t$, $w \to w^{\ast} + 2 t \nu \cdot z_2^{\ast} + i t^2 \nu$.
	\end{itemize}

	Для поверхностей~(\ref{eq:dim4}) имеется еще один тип движений:
	\begin{itemize}
		\item \emph{повороты (вращения)} в плоскости $z_1$: $z_1 \to e^{i \varphi}  \cdot z_1^{\ast}$.
	\end{itemize}

	Поверхности~(\ref{eq:dim5}) имеют, в дополнение к основным, допускают еще два типа движений:
	\begin{itemize}
		\item \emph{сдвиг} по переменной $y_1$: $z_1 \to z_1^{\ast} + i t$;
		\item \emph{поворот}: $z_1 \to z_1^{\ast} + t$, $z_2 \to z_2^{\ast}$, $w \to 2 i t \cdot z_1^{\ast} +  i t^2 \cdot z_2^{\ast} + w^{\ast}$.
	\end{itemize}
}
\frame
{
	\frametitle{Частный случай (продолжение)}
	\textbf{Теорема 2:}
	для любой тройки вещественных коэффициентов $\mu$, $\nu$, $\lambda$,
	одновременно не равных нулю, поверхность
	\begin{equation*}
		v x_2 = x_1^2 + x_2 (\mu\,{x_2}^2 + \lambda\,x_2 y_2 + \nu\,{y_2}^2)
	\end{equation*}
	является аффинно-однородной, а размерность группы Ли аффинных преобразований,
    сохраняющих такую поверхность, равна пяти.
}
\section{Общий случай}
\frame
{
	\frametitle{Общий случай}
	Явный вид квадратичной формы $Q(x_1, y_1, x_2, y_2)$ для произвольной поверхности из~(\ref{eq:family}):
	\begin{align*}
	Q(x_1, y_1, x_2, y_2) &= k_1 x_1^2 + k_2 x_1 x_2 + k_3 x_2^2 + k_4 x_1 y_1 + k_5 x_2 y_1 \nonumber\\&+ k_6 y_1^2 + k_7 x_1 y_2 + k_8 x_2 y_2 + k_9 y_1 y_2 + k_{10} y_2^2
	\end{align*}

	Возникает СЛАУ из 83 уравнений относительно 24 неизвестных.

	Сложность исследования заключается не только в размере системы, но и в том, что
	коэффициенты этой системы зависят \textbf{полиномиально} от $k_1, \dots,
	k_{10}$.
}
\frame
{
	\frametitle{Общий случай: оценка размерности}
	\textbf{Теорема 3 (основная):}
	размерность группы Ли аффинных преобразований, сохраняющих 	любую поверхность
	из семейства~(\ref{eq:family}), удовлетворяет неравенствам
	$
		1 \le \dim G \le 5,
	$ и каждая из размерностей достижима.

	\textbf{Пример 1:}
	единичная размерность группы достигается на поверхности
	$$
		v x_2 = {|z_1|}^2 +x_1 y_1 + {y_2}^2 + x_2 (\mu\,{x_2}^2 + \nu\,{y_2}^2).
	$$
	\textbf{Пример 2:}
	$\dim G = 2$ достигается на поверхности
	$$
		v x_2 = {|z_1|}^2 + {y_2}^2 + x_2 (\mu\,{x_2}^2 + \nu\,{y_2}^2).
	$$
}
\section{Заключение}
\frame
{
	\frametitle{Заключение}
	Итоги работы:
	\begin{itemize}
		\item Была получена оценка размерностей групп аффинных преобразований, сохраняющих поверхности исследуемого семейства кубических гиперповерхностей в $\mathbb{C}^3$
		\item Написана программа на языке {\ttfamily Wolfram Language} для определения размерности группы преобразований, действующих на поверхности
		\item Произведени обобщение известных классов аффинно-однородных поверхностей из~\cite{arxiv}.
	\end{itemize}

}
\section{Литература}
\frame[shrink=1]
{
	\frametitle{Литература}
	\begin{thebibliography}{10}
	\beamertemplatebookbibitems
  		\bibitem{lie}
    	Ли С.
    	\newblock {\em Теория групп преобразований: в 3-х частях. Часть 1}.
    	\newblock Ижевск: Институт компьютерных исследований, 2011.
	\beamertemplatearticlebibitems
		\bibitem{izvestiya}
    	А. В. Лобода, А. С. Ходарев
    	\newblock {\em Об одном семействе аффинно-однородных вещественных гиперповерхностей 3-мерного комплексного пространства}.
    	\newblock Известия вузов, 2003
	\beamertemplatearticlebibitems
		\bibitem{arxiv}
    	A.V. Atanov, A.V. Loboda, A.V. Shipovskaya
    	\newblock {\em Affine homogeneous strictly pseudoconvex hypersurfaces of the type (1/2,0) in $\mathbb{C}^3$ }.
    	\newblock ArXiv e-prints, 2014
	\beamertemplatearticlebibitems
		\bibitem{arxiv_2}
    	A. Isaev, B. Kruglikov
    	\newblock {\em On the symmetry algebras of 5-dimensional CR-manifolds}.
    	\newblock ArXiv e-prints, 2016
    \end{thebibliography}
}
\frame
{
	\begin{center}
		\Huge Спасибо за внимание
	\end{center}
}
\appendix
\backupbegin
\frame
{
	\frametitle{Получение системы}
	Однопараметрическая группа  в $\mathbb{C}^3$:
	\begin{equation*}
	\underbrace{
	\begin{pmatrix}
		A_1(\mathbf t) & A_2(\mathbf t) & A_3(\mathbf t) \\
		B_1(\mathbf t) & B_2(\mathbf t) & B_3(\mathbf t) \\
		C_1(\mathbf t) & C_2(\mathbf t) & C_3(\mathbf t) \\
	\end{pmatrix}}_\text{вращательная компонента}
	\cdot
	\begin{pmatrix}
		z_1 \\
		z_2 \\
		w
	\end{pmatrix}
	+
	\underbrace{
	\begin{pmatrix}
		P_1(\mathbf t) \\
		P_2(\mathbf t) \\
		q(\mathbf t)
	\end{pmatrix}}_\text{сдвиговая компонента},
	\end{equation*}
	$A_i$, $B_i$, $C_i$, $P_i$, $q$ "--- комплексные функции.

	Инфинитезимальное преобразование:
\begin{equation*}
\begin{pmatrix}
\alpha_{1,1} + i\cdot\alpha_{1,2} & \alpha_{2,1} + i\cdot\alpha_{2,2} & \alpha_{3,1} + i\cdot\alpha_{3,2} & \sigma_{1,1} + i\cdot\sigma_{1,2} \\
 \beta_{1,1} +  i\cdot\beta_{1,2} &  \beta_{2,1} +  i\cdot\beta_{2,2} &  \beta_{3,1} +  i\cdot\beta_{3,2} & \sigma_{2,1} + i\cdot\sigma_{2,2} \\
\gamma_{1,1} + i\cdot\gamma_{1,2} & \gamma_{2,1} + i\cdot\gamma_{2,2} & \gamma_{3,1} + i\cdot\gamma_{3,2} & \sigma_{3,1} + i\cdot\sigma_{3,2} \\
\end{pmatrix},
\end{equation*}
	здесь элементами матрицы являются производные в точке ноль соответствующих элементов из матрицы однопараметрической группы
}
\frame
{
	\frametitle{Получение системы}
	 Основное тождество:
 	$$
  	\left.\left\{ \left.\frac{\mathsf d}{\mathsf d t}F_t(\Phi) \right|_{t=0}\right\}\right|_{M} \equiv 0
	$$

  Умножая на возникающий после подстановки $M$ знаменатель, получаем вещественное полиномиальное уравнение:
  $$
  	S(x_1, y_1, x_2, y_2, u) \equiv 0,
  $$
  где $S(x_1, y_1, x_2, y_2, u)$ "--- многочлен степени 6.

  Если полином тождественен нулю, то все коэффициенты при его одночленах равны нулю.
}
\frame
{
	\frametitle{Пример сокращения числа параметров}
	Семейство в частном случае:
	$$
	v x_2 + k_1 {x_1}^2 + k_2 x_1 y_1 + k_3 {y_1}^2 = x_2 (\mu\,{x_2}^2 + \nu\,{y_2}^2).
	$$

	Предположим, что $k_1 \ne 0$. Произведем замену переменных: ${z_1}^* = z_1 / \sqrt{k_1}$. Тогда:
		$$
	v x_2 + {x_1}^2 + {k_2}^* x_1 y_1 + {k_3}^* {y_1}^2 = x_2 (\mu\,{x_2}^2 + \nu\,{y_2}^2),
	$$
	где ${k_2}^* = k_2 / \sqrt{k_1}$, ${k_3}^* = k_3 / \sqrt{k_1}$
}
\frame
{
	\frametitle{Ценность в <<реальном мире>>}
	Сложно найти применение в повседневной жизни трехмерным комплексным пространствам. Современная физика, однако, очень сильно полагается на аппарат групп Ли и многомерный комплексный анализ.

	Теория струн утверждает, что мы живем в 10-мерном пространстве "--- привычные нам 4 измерения дополняются 6-мерной компактной добавкой. Эта добавка имеет название пространство Калаби-Яу и является трехмерным комплексным пространством с определенными свойствами.
}
\backupend


\end{document}