%!TEX TS-program = xelatex
%!TEX encoding = UTF-8 Unicode
\documentclass[russian,hyperref={unicode}]{beamer}
% Other packages
\usepackage{amsmath, amssymb, amsthm}
% \usepackage{beamerthemesplit} // Activate for custom appearance

% Theme selection
\usetheme[sectionpage=none, numbering=fraction]{metropolis}
% Locale packages
\usepackage{polyglossia}
\setmainlanguage{russian}
\setotherlanguage{english}
\setkeys{russian}{babelshorthands=true}

\title{Способ определения координат и угловой ориентации бортовой пеленгаторной антенны по результатам радиопеленгования радиоориентиров}
\institute
{
  \inst{1}%
  Военно-воздушная академия имени профессора Н.Е.Жуковского и Ю.А.Гагарина\\
  \inst{2}%
	Воронежский Государственный Университет
}
\author
{
  Виноградов Д.А.\inst{1}, Минин Л.А.\inst{2}, Морозов Е.Ю.\inst{2}, Ушаков С.Н\inst{2}.
}
\date[RLNC 2019]{XXV Международная научно-техническая конференция <<Радиолокация, навигация, связь>>, 2019}

\begin{document}
  \frame{\titlepage}

  \section{Введение}
  \begin{frame}{Введение}
    Определение координат и угловой ориентации подвижных объектов производится с
    помощью GPS и ГЛОНАСС.

    Такие системы используют дальномерно-угломерный подход и работоспособны только
    при синхронном излучении радиосигналов радиоориентиров.

    Существуют угломерные системы, не накладывающие требования синхронности излучения.

    Угломерные системы, способные однозначно и одновременно определять координаты и
    угловую ориентацию подвижного объекта только с помощью азимутально-угломестного радиопеленгования
    радиоориетиров, не исследованы.
  \end{frame}

  \section{Постановка задачи}
  \begin{frame}{Постановка задачи}
    Пусть $N$ радиоориентиров размещены в точках пространства с известными координатами. Угловая ориентация
    и координаты подвижного объекта неизвестны.

    Подвижный объект оснащен бортовой пеленгаторной антенной, способной для каждого РО опеределить азимут и
    угол места в связанной системе координат.

    Необходимо определить минимальное количество радиоориентиров, при котором
    представляется возможным однозначно найти координаты и угловую ориентацию
    подвижного объекта.
  \end{frame}
  \begin{frame}{Схема решения}
    Детерминированный подход к определению координат и пространственной ориентации подвижного объекта можно представить
    в виде трехэтапной процедуры:
    \begin{enumerate}
        \item Нахождение совокупности расстояний от фазового центра (ФЦ) БПА до радиоориентиров;
        \item Определение координат подвижного объекта;
        \item Нахождение матрицы вращения и связанных с нею углов Эйлера, определяющих угловую ориентацию БПА.
    \end{enumerate}
    Задачи второго и третьего этапов являются стандартными для радионавигации подвижных объектов.
  \end{frame}

\end{document}