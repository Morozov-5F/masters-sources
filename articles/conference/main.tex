\documentclass[a4paper,12pt]{article}

\usepackage[T2A]{fontenc}
\usepackage[utf8]{inputenc}
\usepackage[english,russian]{babel}

\usepackage{cmap}
\usepackage{amsmath}
\usepackage{graphicx}
\graphicspath{ {./pictures/} }

\usepackage[margin=1in]{geometry}

\author{Виноградов А.Д., Минин Л.А., Морозов Е.Ю., Ушаков С.Н.}
\title{Исследование различных детерминированных подходов в угломерной радиопеленгации}
\date{}

\begin{document}

\maketitle

\begin{abstract}
Рассмотрена задача определения координат и угловой ориентации бортовой пеленгаторной антенны,
размещенной на подвижном объекте воздушного, морского или наземного базирования, по результатам
азимутально-угломестного радиопеленгования радиоориентиров (реперных источников радиоизлучения).
Определены аналитические соотношения, описывающие взаимосвязь азимута и угла места реперного
источника радиоизлучения, измеряемые бортовым азимутально-угломестным радиопеленгатором,
с параметрами пространственного положения и угловой ориентации бортовой пеленгаторной антенны
и получено соответствующие выражения для расчета координат и углов курса, тангажа и крена бортовой
пеленгаторной антенны.
\end{abstract}

\section*{Введение}
Задача определения координат и угловой ориентации подвижных объектов в воздухе, на суше и в море
в настоящее время решается с помощью спутниковых радионавигационных систем (GPS или ГЛОНАСС). В таких
системах радиоориентиры (РО) расположены на спутниках с известными координатами, а прием этих сигналов
ведется в нескольких точках, расположенных на подвижном объекте. При этом, для определения координат
используется дальномерный подход (измеряются задержки приходов сигналов), а для определения угловой
ориентации "--- угломерный (измеряются углы наблюдения РО). Условием работы данных систем является
синхронное излучение радиосигналов РО.

В работах [??] исследованы различные способы определения координат подвижного объекта угломерным методом,
которые не накладывают требования синхронности излучения радиосигналов радиоориентирами. Однако, возможности
одновременного и однозначного определения координат и угловой ориентации подвижного объекта путем азимутально-
угломестного радиопеленгования с борта объекта без использования какой-либо вспомогательной информации
от автономных навигационных датчиков и систем не изучены в современных работах по радионавигации. При этом,
в некоторых задачах возникает необходимость определения ограничений на количество РО и на взаимное расположение
объекта и РО при условии сохранения однозначности определения координат и угловой ориентации.

Таким образом, цель данной работы заключается в исследовании возможности и условий однозначного определения
координат и угловой ориентации воздушного объекта по результатам азимутально-угломестного радиопеленгования
радиоориентиров.

\section{Постановка задачи}
Пусть $N$ радиоориентиров размещены в $i$-х точках $M_i$ пространства (где $i = 1, \cdots, N$) с известными
координатами $M_i\left(x_i, y_i, z_i\right)$ в нормальной земной системе координат (НЗСК). Подвижный объект
находится в точке $M_0$ с неизвестными координатами $M_0\left(x, y, z\right)$, а его угловая ориентация
определяется тремя неизвестными углами Эйлера: углами курса $\psi$, крена $\mu$ и тангажа $\vartheta$.
В результате азимутально-угломестного радиопеленгования реперного источника $M_i$, определяются углы
азимута ($\alpha_i$) и угла места ($\varepsilon_i$) в связанной системе координат подвижного объекта. Необходимо
определить минимальное количество радиоориентиров $M_i$, при котором представляется возможным однозначно найти
координаты $M_0\left(x, y, z\right)$ и угловую ориентацию (углы $\psi$, $\mu$ и $\vartheta$) подвижного
объекта.
% Возможно, убрать требование про минимальность

Так как положение и угловая ориентация бортовой пеленгаторной антенны (БПА) определяется шестью параметрами
(тремя координатами $x$, $y$, $z$ и тремя углами $\psi$, $\mu$ и $\vartheta$), то для их определения необходимо
иметь как минимум шесть измеряемых величин. Этими величинами являются пары азимутов $\alpha_i$ и углов места
$\varepsilon_i$. Таким образом, для однозначного определения положения и пространственной ориентации подвижного
объекта, число радиоориентиров должно быть не менее трех.

Детерминированный подход к определению координат и пространственной ориентации подвижного объекта можно представить
в трехэтапной процедуры, заключающейся в следующем:
\begin{enumerate}
    \item Нахождение совокупности расстояний от фазового центра (ФЦ) БПА до радиоориентиров;
    \item Определение координат подвижного объекта;
    \item Нахождение матрицы вращения и связанных с нею углов Эйлера, определяющих угловую ориентацию БПА.
\end{enumerate}

Следует отметить, что задачи второго и третьего этапов являются стандартными для радионавигации подвижных объектов [??],
поэтому для их решения не требуется разрабатывать специальных методов. Ключевым оказывается именно первый этап.

Система уравнений для нахождения расстояний от ФЦ БПА до радиоориентиров на первом этапе нелинейна. При произвольном взаимном
расположении БПА и радиоориентиров, решение упомянутой системы не однозначно, и может включать в себя от одного
до четырех наборов расстояний, что ведет к неоднозначности определений координат и угловой ориентации. По мере увеличения
числа источников, происходит переопределение системы, и, соответственно, она перестает быть совместной при малейших
погрешностях измерений азимутов и углов места. Использование стандартного метода наименьших квадратов приводит к повышению
степени уравнений в системе и, как следствие, к резкому усложнению процедуры определения искомых параметров. Поэтому, в данной
работе рассматривается система, состоящая из трех наземных опорных источников радиоизлучения.

% ОПределить, нужен ли абзац ниже:
Структура возможных решений системы уравнений на первом этапе оказывается достаточно сложной и включает в себя большое
количество вырожденных случаев. Построение эффективных алгоритмов не представляется возможным без понимания данных
особенностей.

\section{Особенности решения}

\section{Заключение}

\end{document}