\documentclass[a4paper,12pt]{article}

\usepackage[T2A]{fontenc}
\usepackage[utf8]{inputenc}
\usepackage[english,russian]{babel}

\usepackage{cmap}
\usepackage{amsmath}
\usepackage{graphicx}
\graphicspath{ {./pictures/} }

\usepackage[margin=1in]{geometry}

\author{Виноградов А.Д., Минин Л.А., Морозов Е.Ю., Ушаков С.Н.}
\title{Исследование различных детерминированных подходов в угломерной радиопеленгации}
\date{}

\begin{document}

\maketitle

\begin{abstract}
Рассмотрена задача определения координат и угловой ориентации бортовой пеленгаторной антенны,
размещенной на подвижном объекте воздушного, морского или наземного базирования, по результатам
азимутально-угломестного радиопеленгования радиоориентиров (реперных источников радиоизлучения).
Определены аналитические соотношения, описывающие взаимосвязь азимута и угла места реперного
источника радиоизлучения, измеряемые бортовым азимутально-угломестным радиопеленгатором,
с параметрами пространственного положения и угловой ориентации бортовой пеленгаторной антенны
и получено соответствующие выражения для расчета координат и углов курса, тангажа и крена бортовой
пеленгаторной антенны.
\end{abstract}

\section*{Введение}
Задача определения координат и угловой ориентации подвижных объектов в воздухе, на суше и в море
в настоящее время решается с помощью спутниковых радионавигационных систем (GPS или ГЛОНАСС). В таких
системах радиоориентиры (РО) расположены на спутниках с известными координатами, а прием этих сигналов
ведется в нескольких точках, расположенных на подвижном объекте. При этом, для определения координат
используется дальномерный подход (измеряются задержки приходов сигналов), а для определения угловой
ориентации "--- угломерный (измеряются углы наблюдения РО). Условием работы данных систем является
синхронное излучение радиосигналов РО.

В работах [??] исследованы различные способы определения координат подвижного объекта угломерным методом,
которые не накладывают требования синхронности излучения радиосигналов радиоориентирами. Однако, возможности
одновременного и однозначного определения координат и угловой ориентации подвижного объекта путем азимутально-
угломестного радиопеленгования с борта объекта без использования какой-либо вспомогательной информации
от автономных навигационных датчиков и систем не изучены в современных работах по радионавигации. При этом,
в некоторых задачах возникает необходимость определения ограничений на количество РО и на взаимное расположение
объекта и РО при условии сохранения однозначности определения координат и угловой ориентации.

Таким образом, цель данной работы заключается в исследовании возможности и условий однозначного определения
координат и уловой ориентации воздушного объекта по результатам азимутально-угломестного радиопеленгования
радиоориентиров.

% Принцип работы такой системы основан на излучении радиосигналов опорными источниками радиоизлучения (ИРИ),
% размещенными на спутниках с известными координатами, и приеме этих радиосигналов в нескольких точках,
% размещенных на подвижном объекте; эти точки выполняют функции бортовой пеленгаторной антенны и обеспечивают
% реализацию фазового метода радиопеленгования, в соответствии с [??]. В данном случае координаты
% объекта определяются по результатам измерения задержки сигналов, излучаемых синхронно не менее чем
% тремя радиоориентирами (РО), которые размещены на спутниках (прием может осуществляться в одной или
% нескольких точках). Угловая ориентации при этом определяется путем измерения углов наблюдения (визирования)
% как минимум трех РО не менее чем двумя пеленгационными парами, базы которых не пересекаются (неколлинеарны).
% Так как угол наблюдения "--- это угол, который образован отрезком, проходящим через пеленгационную
% пару точек приема, и направлением на РО, проходящим через одну из точек приема этой же пары, координаты которой
% были определены вышеупомянутым образом, можно считать, что координаты и угловая ориентация воздушного объекта
% определяются в данном случае дальномерно-угломерным методом. Данный метод дает результаты только при условии синхронного излучения радиосигналов РО.
% Переделать! Может, опустить некоторые излишние детали
\section{Постановка задачи}

\section{}

\end{document}