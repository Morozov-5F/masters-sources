\documentclass[a4paper,12pt]{article}

\usepackage[T2A]{fontenc}
\usepackage[utf8]{inputenc}
\usepackage[english,russian]{babel}

\usepackage{cmap}
\usepackage{amsmath}
\usepackage{graphicx}
\graphicspath{ {./pictures/} }

\usepackage[margin=1in]{geometry}

\author{Виноградов А.Д., Минин Л.А., Морозов Е.Ю., Ушаков С.Н.}
\title{Исследование различных детерминированных подходов в угломерной радиопеленгации}
\date{}

\begin{document}

\maketitle

\begin{abstract}
Рассмотрена задача определения координат и угловой ориентации бортовой пеленгаторной антенны,
размещенной на подвижном объекте воздушного, морского или наземного базирования, по результатам
азимутально-угломестного радиопеленгования радиоориентиров (реперных источников радиоизлучения).
Определены аналитические соотношения, описывающие взаимосвязь азимута и угла места реперного
источника радиоизлучения, измеряемые бортовым азимутально-угломестным радиопеленгатором,
с параметрами пространственного положения и угловой ориентации бортовой пеленгаторной антенны
и получено соответствующие выражения для расчета координат и углов курса, тангажа и крена бортовой
пеленгаторной антенны.
\end{abstract}

\section*{Введение}
Задача определения координат и угловой ориентации подвижных объектов в воздухе, на суше и в море
в настоящее время решается с помощью спутниковых радионавигационных систем (GPS или ГЛОНАСС). Принцип
работы такой системы основан на излучении радиосигналов опорными источниками радиоизлучения (ИРИ),
размещенными на спутниках с известными координатами, и приеме этих радиосигналов в нескольких точках,
размещенных на подвижном объекте; эти точки выполняют функции бортовой пеленгаторной антенны и обеспечивают
реализацию фазового метода радиопеленгования, в соответствии с [??]. В данном случае координаты
объекта определяются по результатам измерения задержки сигналов, излучаемых синхронно не менее чем
тремя радиоориентирами (РО), которые размещены на спутниках (прием может осуществляться в одной или
нескольких точках). Угловая ориентации при этом определяется путем измерения углов наблюдения (визирования)
как минимум трех РО не менее чем двумя пеленгационными парами, базы которых непересекаются (неколлинеарны).
Так как угол наблюдения "--- это угол, который образован отрезком, проходящим через пеленгационную
пару точек приема, и направлением на РО, проходящим через одну из точек приема этой же пары, координаты которой
были определены вышеупомянутым образом, можно считать, что координаты и угловая ориентация воздушного объекта
определяются в данном случае дальномерно-угломерным методом. Данный метод дает результаты только при условии синхронного излучения радиосигналов РО.
% Переделать! Может, опустить некоторые излишние детали
\section{Постановка задачи}

\section{}

\end{document}