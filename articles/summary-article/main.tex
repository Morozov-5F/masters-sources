\documentclass[a4paper,12pt]{article}

\usepackage[T2A]{fontenc}
\usepackage[utf8]{inputenc}
\usepackage[english,russian]{babel}

\usepackage{cmap}
\usepackage{amsmath}
\usepackage{graphicx}
\graphicspath{ {./pictures/} }

\usepackage[margin=1in]{geometry}

\author{Виноградов А.Д., Минин Л.А., Морозов Е.Ю., Ушаков С.Н.}
\title{Исследование различных детерминированных подходов в угломерной радиопеленгации}
\date{}

\begin{document}

\maketitle

\section{Случай локальной системы с НПУ}
В работе~\cite{antennas} рассматривалась локальная угломерная навигационная система (ЛУНС),
состоящая из трех наземных радиоориентиров и воздушного объекта, оснащенного радиолокационным
оборудованием, которое способно определять азимут и угол места каждого из наземных радиоориентиров.
Авторами было показано, что определение пространственных координат и угловой ориентации воздушного
объекта сводится к решению нелинейной системы уравнений следующего вида:
\begin{equation}\label{eq:luns_start}
    \begin{cases}
        \ell_1^2 + \ell_2^2 - 2 \ell_1 \ell_2 \cos \alpha_{12} = d_{12}^2 \\
        \ell_1^2 + \ell_3^2 - 2 \ell_1 \ell_3 \cos \alpha_{13} = d_{13}^2 \\
        \ell_2^2 + \ell_3^2 - 2 \ell_2 \ell_3 \cos \alpha_{23} = d_{23}^2 \\
    \end{cases},
\end{equation}
где $\ell_i$ "--- это расстояние от воздушного объекта до $i$-го радиоориентира, $d_{i,j}$
"--- расстояние между $i$-м и $j$-м наземным радиоориетиром, $\cos \alpha_{i,j}$ "---
плоский угол, образованный наземными ориентирами $i$ и $j$ и воздушным объектом. Из этой
системы далее выводятся координаты и угловая ориентация воздушного объекта.

Система~(\ref{eq:luns_start}) может иметь несколько решений, что ведет к неоднозначности определения
координат и угловой ориентации летательного аппарата. Поэтому, в статье~\cite{antennas} приводятся
методики для расчета областей, где решение можно определить однозначно. Более того, решение системы~(\ref{eq:luns_start})
производится численно, что потенциально может привести к дополнительным ошибкам в определении
пространственных параметров воздушного объекта. В связи с этим, далее предлагается еще один
вариант ЛУНС, которая лишена вышеперечисленных недостатков ценой усложнения структуры системы.

Предлагается заменить один из пассивных радиоориентиров на наземный пункт управления (НПУ), оснащенный
радиоориентиром. Требуется, чтобы НПУ мог определять азимут и угол места двух других радиоориентиров
и воздушного объекта.

По аналогии с~\cite{antennas}, будем считать, что воздушный объект находится в точке $M_0\left(x, y, z\right)$,
НПУ "--- в $M_1\left(x_1, y_1, z_1\right)$, а $i$-й РО "--- в точке $M_i\left(x_i, y_i, z_i \right)$, $i = 1,2$.
Азимут и угол места радиоориентира $M_i$, полученные в результате радиопеленгования, обозначим через
$\theta_i$ и $\beta_i$ соответственно (см. рис.~\ref{figure:pic1}). Предполагается, что измерения этих
углов проводятся в локальной системе координат НПУ, центр которой совпадает с координатами НПУ в НЗСК, а
направления осей совпадают с НЗСК. В этой системе координат можно определить три единичных вектора направлений
на радиоориентиры $M_0$, $M_2$ и $M_3$:
\begin{equation*}
    \mathbf{s}_{\text{лн}i} = \left(\cos\theta_i \cos\beta_i, \sin\theta_i\cos\beta_i, \sin\beta_i\right),
\end{equation*}
где $i = 0, 2, 3$. Тогда, косинусы углов $\alpha_{02} = \angle M_0 M_1 M_2$ и $\alpha_{03} = \angle M_0 M_1 M_3$
определяются как
\begin{align*}
    \cos\alpha_{02} = \left(\mathbf{s}_{\text{лн}0}, \mathbf{s}_{\text{лн}2}\right) =
    \cos\beta_0 \cos\beta_2 \cos\left(\theta_0 - \theta_2\right) + \sin\beta_0 \sin\beta_2, \\
    \cos\alpha_{03} = \left(\mathbf{s}_{\text{лн}0}, \mathbf{s}_{\text{лн}3}\right) =
    \cos\beta_0 \cos\beta_3 \cos\left(\theta_0 - \theta_3\right) + \sin\beta_0 \sin\beta_3.
\end{align*}

\begin{figure}[htbp]
    \begin{center}

    \fbox{\includegraphics{pic1}}

    \caption{Схемы размещения на местности БпЛА, НПУ и РО для реализации ЛУНС}
    \label{figure:pic1}
    \end{center}
\end{figure}

По теореме синусов для треугольника $M_1 M_0 M_2$:
\begin{equation}\label{eq:luns_sin_1}
    \frac{\ell_2}{\sin\alpha_{02}} = \frac{d_{12}}{\sin\alpha_{12}} = \frac{\ell_1}{\sin\left(\alpha_{12} + \alpha_{02}\right)}.
\end{equation}
Аналогично для треугольника $M_1 M_0 M_3$:
\begin{equation}\label{eq:luns_sin_2}
    \frac{\ell_3}{\sin\alpha_{03}} = \frac{d_{13}}{\sin\alpha_{13}} = \frac{\ell_1}{\sin\left(\alpha_{13} + \alpha_{03}\right)}.
\end{equation}
Таким образом, из~(\ref{eq:luns_sin_1}):
\begin{equation*}
    \ell_2 = \frac{d_{12}\sin\alpha_{02}}{\sin\alpha_{12}}.
\end{equation*}
В то же время, из~(\ref{eq:luns_sin_2}):
\begin{equation*}
    \ell_3 = \frac{d_{13}\sin\alpha_{03}}{\sin\alpha_{13}}.
\end{equation*}
Длина $\ell_1$ может быть найдена из любого уравнения:
\begin{equation*}
    \ell_1 = \frac{d_{12}\sin\left(\alpha_{12} + \alpha_{02}\right)}{\sin\alpha_{12}} = \frac{d_{13}\sin\left(\alpha_{13} + \alpha_{03}\right)}{\sin\alpha_{13}}.
\end{equation*}
Далее пространственные координаты и угловая ориентация БПА определяется согласно~\cite{antennas}.

\section{Локальная система с уменьшенным количеством наземных РО}
В работе~\cite{antennas} было показано, что минимально возможное число наземных радиоориентиров, которое позволяет
однозначно определять координаты и угловую ориентацию воздушного объекта, равно трем. Однако, это число можно уменьшить,
если добавить в систему еще один подвижный объект с радиоориентиром и бортовой пеленгационной антенной.

Пусть подвижные объекты находятся в точках $M_1\left(x_1, y_1, z_1\right)$ и $M_2\left(x_2, y_2, z_2\right)$, а
наземные ориентиры находятся в точках $M_3\left(x_3, y_3, z_3\right)$ и $M_4\left(x_4, y_4, z_4\right)$, а расстояние
между ними равно $d_{34}$. В результате азимутально-угломестной радиопеленгации  $j$-го радиоориентира с  борта
$i$-го подвижного объекта ($i \ne j$), определяются углы азимута ($\alpha_{ij}$) и угла места ($\varepsilon_{ij}$)
(см рис.~\ref{figure:pic2}). Пусть также $\ell_ij$ "--- расстояние между $i$-м и $j$-м радиоориентиром.

\begin{figure}[htbp]
    \begin{center}

    \fbox{\includegraphics[width=0.8\columnwidth]{pic2}}

    \caption{Схемы размещения на местности БпЛА, НПУ и РО для реализации ЛУНС}
    \label{figure:pic2}
    \end{center}
\end{figure}

Обозначим через $\mathbf{s}_{ij}$ единичные векторы, которые определяются следующим образом:
\begin{equation}
    \mathbf{s}_{ij} = \left(\cos\alpha_{ij} \cos\varepsilon_{ij}, \cos\alpha_{ij} \sin\varepsilon_{ij}, \sin\varepsilon_{ij}\right),
\end{equation}
где $i \ne j$, $i = 1,2$, $j = 1,2,3,4$. Также обозначим через $\gamma_{ij}$ углы, образованные воздушным объектом $M_1$ и
радиоориентирами $i$ и $j$ ($i \ne j$), а через $\varphi_{ij}$ "--- углы, образованные воздушным объектом $M_2$ и теми же
радиоориентирами. В таком случае, косинусы этих углов находятся по формулам:
\begin{align*}
    \cos\gamma_{ij} &= \left(\mathbf{s}_{1i}, \mathbf{s}_{1j}\right), i \ne j \ne 1, \\
    \cos\varphi_{ij} &= \left(\mathbf{s}_{2i}, \mathbf{s}_{2j}\right), i \ne j \ne 2.
\end{align*}

По теореме синусов для треугольника $M_1 M_2 M_3$:
\begin{equation}\label{eq:luns_2_sin_1}
    \frac{\ell_{12}}{\sin{M_1 M_3 M_2}} = \frac{\ell_{13}}{\sin{\varphi_{13}}} = \frac{\ell_{23}}{\sin{\gamma_{13}}}.
\end{equation}
Для треугольника $M_1 M_2 M_4$:
\begin{equation}\label{eq:luns_2_sin_2}
    \frac{\ell_{12}}{\sin{M_1 M_4 M_2}} = \frac{\ell_{14}}{\sin{\varphi_{14}}} = \frac{\ell_{24}}{\sin{\gamma_{24}}}.
\end{equation}
Из (\ref{eq:luns_2_sin_1}) и (\ref{eq:luns_2_sin_2}) получим отношения:
\begin{align}
     \frac{\ell_{23}}{\ell_{12}} = \frac{\sin{\gamma_{13}}}{\sin{M_1 M_3 M_2}}, \frac{\ell_{24}}{\ell_{12}} = \frac{\sin{\gamma_{14}}}{\sin{M_1 M_4 M_2}} \label{eq:luns_2_triag_1}\\
     \frac{\ell_{13}}{\ell_{12}} = \frac{\sin{\varphi_{13}}}{\sin{M_1 M_3 M_2}}, \frac{\ell_{14}}{\ell_{12}} = \frac{\sin{\varphi_{14}}}{\sin{M_1 M_4 M_2}} \label{eq:luns_2_triag_2}.
\end{align}
По теореме косинусов для треугольника $M_2 M_4 M_3$:
\begin{equation}\label{eq:luns_2_cos_1}
    \ell_{23}^2 + \ell_{24}^2 - 2 \ell_{23} \ell_{24} \cos\varphi_{34} = \ell_{34}^2.
\end{equation}
С учетом (\ref{eq:luns_2_triag_1}), (\ref{eq:luns_2_cos_1}) можно переписать в виде:
\begin{equation}
    \ell_{12}^2 k_1^2 + \ell_{12}^2 k_2^2 - 2 \ell_{12} \ell_{12} k_1 k_2 \cos\varphi_{34} = \ell_{34}^2,
\end{equation}
где $k_1 = \sin\gamma_{13} / \sin{M_1 M_3 M_2}$, а $k_2 = \sin\gamma_{14} / \sin{M_4 M_1 M_2}$.
Тогда $\ell_{12}$ выражается следующим образом:
\begin{equation}\label{eq:luns_2_ell_12}
    \ell_{12} = \frac{\ell_{34}}{\sqrt{k_1^2 + k_2 ^2 - 2 k_1 k_2 \cos\varphi_{34}}}.
\end{equation}
С учетом (\ref{eq:luns_2_ell_12}), из (\ref{eq:luns_2_triag_1}) и (\ref{eq:luns_2_triag_2}) находятся
расстояния $\ell_{13}$, $\ell_{14}$, $\ell_{23}$ и $\ell_{24}$.

Для нахождения координат воздушного объекта $M_i$, введем радиус-вектор $\mathbf{r}_{i3}$, который определяется следующим
образом:
\begin{equation}
    \mathbf{r}_{i3} = \mathbf{s}_{i3} \ell_{i3} = M_3 - M_i.
\end{equation}
Отсюда, координаты точки $M_i$ в НЗСК находятся из системы уравнений:
\begin{equation}
    \begin{cases}
        x_i = x_3 - \ell_{i3} \cos\alpha_{i3} \cos\varepsilon_{i3}, \\
        y_i = y_3 - \ell_{i3} \cos\alpha_{i3} \sin\varepsilon_{i3}, \\
        z_i = y_3 - \ell_{i3} \sin\varepsilon_{i3}.
    \end{cases}
\end{equation}
Следует заметить, что аналогичные рассуждения справедливы и относительно радиоориентира $M_4$. Совокупность
расчетов координат относительно обоих наземных радиоориентиров позволяет ввести дополнительный контроль ошибок.

Отметим, что работоспособность такой конфигурации системы достигается только в случае, когда все четыре точки не
лежат в одной плоскости. Это реализуется при расположении подвижных аппаратов $M_1$ и $M_2$ по разные стороны от прямой,
образованной наземными радиоориентирами $M_3$ и $M_4$ (см. рис.~\ref{figure:pic2}).

Матрица поворота и угловая ориентация каждого из объектов находится по алгоритму, представленному в работе~\cite{antennas}.
% Тут я, возможно, немного слукавил, наверное, следует дописать полностью

\section{Случай автономной системы}
Автономная угломерная радионавигационная система (АУНС) предназначена для определения
координат и угловой ориентации в пространстве двух воздушных объектов, оснащенных
высотомерами и бортовыми радиоориентирами с наземного пункта управления (НПУ),
оснащенного радиоориентиром.
% Схема размещения на местности воздушных объектов и наземного пункта управления приведены на рис.~\ref{figure:pic3}.

Пусть радиоориентир НПУ расположен в точке $M_0$ с заранее известными координатами
$M_0\left(x_0, y_0, z_0\right)$ в Балтийской системе координат (БСК), а воздушные объекты
-- в точках $M_1$ и $M_2$ с координатами $M_i\left(x_i, y_i, z_i\right)$, при этом $z_i$
совпадает с измерениями высотомера $h_i$. НПУ $M_0$ способен измерять радиопеленг $\theta_i$
и угол возвышения $\beta_i$ $i$-го воздушного объекта в БСК. Воздушные объект $M_n$
способен измерять азимут $\alpha_{ni}$ и угол места $\varepsilon_{ni}$ $i$-го
радиоориентира (наземного или воздушного) в связанной системе координат БПА \cite{antennas}.
Схема размещения с указанными величинами указана на рис.~\ref{figure:pic3}.
Пространственное положение радиоориентиров в БСК также можно охарактеризовать радиус-векторами
$\mathbf{r}_i = \left(x_i, y_i, z_i\right)$, где $i = 1,2,3$.

\begin{figure}[htbp]
    \begin{center}

    \fbox{\includegraphics{pic3}}

    \caption{Схемы размещения на местности БпЛА, НПУ и РО для реализации АУНС}
    \label{figure:pic3}
    \end{center}
\end{figure}

При детерминированном подходе для такой системы возможно однозначно определить координаты
и угловую ориентацию воздушных объектов. Для этого нужно выполнить следующие ключевые шаги:
\begin{enumerate}
    \item Определить координаты воздушных объектов в БСК.
    \item Определить координаты радиоориентиров в связанной системе координат воздушного объекта $M_1$ ($\Sigma_{\text{св}1}$).
    \item Составить матрицу поворота системы координат $\Sigma_{\text{св}1}$ по алгоритму, представленному ниже.
    \item Определить углы поворота $\Sigma_{\text{св}1}$ по алгоритму, представленному в \cite{antennas}.
    \item Повторить предыдущие шаги для воздушного ориентира $M_2$.
\end{enumerate}

Первая часть алгоритма реализуется явно "--- совокупность данных с высотомеров воздушных
объектов и углов $\theta_i$, $\beta_i$ позволяют определить координаты летательных
аппаратов однозначно. Таким образом, координаты радиоориентира $M_1$ и $M_2$ определяются следующим
отношениями:
\begin{equation}
    \begin{cases}
        x_1 = \rho_1 \cos\theta_1 \cos\beta_1 \\
        y_1 = \rho_1 \sin\theta_1 \cos\beta_1 \\
        z_1 = h_1 = \rho_1 \sin\beta_1 \\
        \rho_1 = ~^{z_1}/_{\sin\beta_1}
    \end{cases},
    \begin{cases}
        x_2 = \rho_2 \cos\theta_2 \cos\beta_2 \\
        y_2 = \rho_2 \sin\theta_2 \cos\beta_2 \\
        z_2 = h_2 = \rho_2 \sin\beta_2 \\
        \rho_2 = ~^{z_2}/_{\sin\beta_2}
    \end{cases}.
\end{equation}

Далее необходимо определить координаты радиоориентиров $M_0$ и $M_2$ в связанной системе координат
воздушного объекта $M_1$:
\begin{equation}
    \begin{cases}
        x'_0 = \rho_{10} \cos\alpha_{10} \cos\varepsilon_{10} \\
        y'_0 = \rho_{10} \sin\alpha_{10} \cos\varepsilon_{10} \\
        z'_0 = z_1 - z_0 = \rho_{10} \sin\varepsilon_{10} \\
        \rho_{10} = |\mathbf{r}_1 - \mathbf{r}_0|
    \end{cases},
    \begin{cases}
        x'_2 = \rho_{12} \cos\alpha_{12} \cos\varepsilon_{12} \\
        y'_2 = \rho_{12} \sin\alpha_{12} \cos\varepsilon_{12} \\
        z'_2 = z_1 - z_2 = \rho_{12} \sin\varepsilon_{12} \\
        \rho_{12} = |\mathbf{r}_1 - \mathbf{r}_2|
    \end{cases},
\end{equation}
где $M_0'\left(x'_0, y'_0, z'_0\right)$ и $M_2'\left(x'_2, y'_2, z'_2\right)$
"--- координаты $M_0$ и $M_2$ в связанной СК $M_1$.

Для определения матрицы поворота системы координат $\Sigma_{\text{св}1}$, необходимо сначала
ввести радиус-вектора $\mathbf{r}'_0 = \left(x'_0, y'_0, z'_0\right)$ и
$\mathbf{r}'_2 = \left(x'_2, y'_2, z'_2\right)$, которые определяют положения радиоориентиров
$M_0$ и $M_2$ в связанной системе координат $M_1$. Далее, зададим единичные векторы
$\mathbf{s}'_1$, $\mathbf{s}'_2$ и $\mathbf{s}'_3$ следующим образом:
\begin{equation}\label{eq:vec_1_local}
    \begin{split}
    \mathbf{s}'_1 = \left(s'_{1x}, s'_{1y}, s'_{1z}\right) = &\left(\cos\alpha_{10} \cos\varepsilon_{10}, \sin\alpha_{10}\cos\varepsilon_{10}, \sin\varepsilon_{10}\right),\\
    \mathbf{s}'_2 = \left(s'_{2x}, s'_{2y}, s'_{2z}\right) = &\left(\cos\alpha_{12} \cos\varepsilon_{12}, \sin\alpha_{12}\cos\varepsilon_{12}, \sin\varepsilon_{12}\right),\\
    \mathbf{s}'_3 = \mathbf{s}'_1 \times \mathbf{s}'_2 = \left(s'_{3x}, s'_{3y}, s'_{3z}\right) = &\left(\sin\alpha_{12}\cos\varepsilon_{12}\sin\varepsilon_{10} - \sin\alpha_{12}\sin\alpha_{10}\cos\varepsilon_{10},\right.\\
    &\ \sin\alpha_{10}\cos\varepsilon_{10}\sin\varepsilon_{12} - \cos\alpha_{10}\cos\varepsilon_{12}\sin\varepsilon_{10},\\
    &\ \left.\sin\left(\alpha_{10} - \alpha_{12}\right)\cos\varepsilon_{10}\cos\varepsilon_{12}\right).
    \end{split}
\end{equation}
Те же вектора в балтийской системе координат:
\begin{equation}\label{eq:vec_1_bsk}
    \begin{split}
        \mathbf{s}_{01} &= \left(s_{01x}, s_{01y}, s_{01z}\right) = \frac{\mathbf{r}_0 - \mathbf{r}_1}{|\mathbf{r}_0 - \mathbf{r}_1|},\\
        \mathbf{s}_{21} &= \left(s_{21x}, s_{21y}, s_{21z}\right) = \frac{\mathbf{r}_2 - \mathbf{r}_1}{|\mathbf{r}_2 - \mathbf{r}_1|},\\
        \mathbf{n}_1 &= \left(n_{1x}, n_{1y}, n_{1z}\right) = \mathbf{s}_{01} \times \mathbf{s}_{21}.\\
    \end{split}
\end{equation}
Определим квадратную матрицу $\mathbf{S}$ размера $3 \times 3$ координат трех полученных по формуле (\ref{eq:vec_1_bsk}) единичных
векторов $\mathbf{s}_{01}$, $\mathbf{s}_{21}$ и $\mathbf{n}_{1}$, записав в столбцы, в соответствии с отношением:
\begin{equation}\label{eq:vec_1_bsk_matrix}
    \mathbf{S}_1 =
    \left(
        \begin{matrix}
            s_{01x} & s_{01y} & s_{01z} \\
            s_{21x} & s_{21y} & s_{21z} \\
            n_{1x} & n_{1y} & n_{1z}
        \end{matrix}
    \right).
\end{equation}
По аналогии с (\ref{eq:vec_1_bsk_matrix}) определим квадратную матрицу $\mathbf{S}'$ размера $3 \times 3$ координат
трех полученных по формуле (\ref{eq:vec_1_local}):
\begin{equation}\label{eq:vec_1_local_matrix}
    \mathbf{S}' =
    \left(
        \begin{matrix}
            s'_{1x} & s'_{1y} & s'_{1z} \\
            s'_{2x} & s'_{2y} & s'_{2z} \\
            s'_{3x} & s'_{3y} & s'_{3z}
        \end{matrix}
    \right).
\end{equation}
Отсюда получим следующее преобразование координат:
\begin{equation*}
    \mathbf{R_1} \times \mathbf{S}_1 = \mathbf{S}'
\end{equation*}

В таком случае, матрицу поворота связанной системы координат $\Sigma_{\text{св}1}$ можно найти следующим образом:
\begin{equation}
    \mathbf{R_1} = \mathbf{S}' \times \mathbf{S}_1^{-1}
\end{equation}

По аналогии можно получить матрицу поворота $\mathbf{R}_2$ связанной системы
координат $M_2$. Вводятся радиус-вектора $\mathbf{r}''_0 = \left(x''_0, y''_0, z''_0\right)$,
$\mathbf{r}''_1 = \left(x''_1, y''_1, z''_1\right)$, по ним же определяются единичные вектора
$\mathbf{s}''_1$, $\mathbf{s}''_2$ и $\mathbf{s}''_3$:
\begin{equation}\label{eq:vec_2_local}
    \begin{split}
    \mathbf{s}''_1 = \left(s''_{1x}, s''_{1y}, s''_{1z}\right) =  &\left(\cos\alpha_{20} \cos\varepsilon_{20}, \sin\alpha_{20}\cos\varepsilon_{20}, \sin\varepsilon_{20}\right),\\
    \mathbf{s}''_2 = \left(s''_{2x}, s''_{2y}, s''_{2z}\right) = &\left(\cos\alpha_{21} \cos\varepsilon_{21}, \sin\alpha_{21}\cos\varepsilon_{21}, \sin\varepsilon_{21}\right),\\
    \mathbf{s}''_3 = \mathbf{s}''_1 \times \mathbf{s}''_2 = \left(s''_{3x}, s''_{3y}, s''_{3z}\right) =  &\left(\sin\alpha_{21}\cos\varepsilon_{21}\sin\varepsilon_{20} - \sin\alpha_{21}\sin\alpha_{20}\cos\varepsilon_{20},\right.\\
    &\ \sin\alpha_{20}\cos\varepsilon_{20}\sin\varepsilon_{21} - \cos\alpha_{20}\cos\varepsilon_{21}\sin\varepsilon_{20},\\
    &\ \left.\sin\left(\alpha_{20} - \alpha_{21}\right)\cos\varepsilon_{20}\cos\varepsilon_{21}\right).
    \end{split}
\end{equation}
В балтийской системе координат:
\begin{equation}\label{eq:vec_2_bsk}
    \begin{split}
        \mathbf{s}_{02} &= \left(s_{01x}, s_{01y}, s_{01z}\right) = \frac{\mathbf{r}_0 - \mathbf{r}_2}{|\mathbf{r}_0 - \mathbf{r}_1|},\\
        \mathbf{s}_{21} &= \left(s_{21x}, s_{21y}, s_{21z}\right) = \frac{\mathbf{r}_2 - \mathbf{r}_1}{|\mathbf{r}_2 - \mathbf{r}_1|},\\
        \mathbf{n}_1 &= \left(n_{1x}, n_{1y}, n_{1z}\right) = \mathbf{s}_{02} \times \mathbf{s}_{21}.\\
    \end{split}
\end{equation}
Определим квадратные матрицы $\mathbf{S}_2$ и $\mathbf{S}''$ размера $3 \times 3$ по аналогии с (\ref{eq:vec_1_bsk_matrix}) и (\ref{eq:vec_1_local_matrix}):
\begin{equation}\label{eq:vec_1_bsk_matrix}
    \mathbf{S}_2 =
    \left(
        \begin{matrix}
            s_{01x} & s_{01y} & s_{01z} \\
            s_{21x} & s_{21y} & s_{21z} \\
            n_{1x} & n_{1y} & n_{1z}
        \end{matrix}
    \right),
    \mathbf{S}'' =
    \left(
        \begin{matrix}
            s''_{1x} & s''_{1y} & s''_{1z} \\
            s''_{2x} & s''_{2y} & s''_{2z} \\
            s''_{3x} & s''_{3y} & s''_{3z}
        \end{matrix}
    \right).
\end{equation}
Отсюда, матрица поворота определяется следующим образом:
\begin{equation}
    \mathbf{R_2} = \mathbf{S}'' \times \mathbf{S}_2^{-1}
\end{equation}

Углы курса $\psi_1$, $\psi_2$, крена $\mu_1$, $\mu_2$ и тангажа $\vartheta_1$, $\vartheta_2$ находятся
из матриц $\mathbf{R}_1$ и $\mathbf{R}_2$ в соответствии с~\cite{antennas}.

% и $\mathbf{n}_1$,
% которые определяются следующим образом:
% \begin{align*}
%     \mathbf{r}'_0 =&~\left(x'_0, y'_0, z'_0\right) / |\left(x'_0, y'_0, z'_0\right)|, \\
%     \mathbf{r}'_2 =&~\left(x'_2, y'_2, z'_2\right) / |\left(x'_0, y'_0, z'_0\right)|, \\
%     \mathbf{n}_1 =&~\mathbf{r}'_0 \times \mathbf{r}'_2
% \end{align*}

\newpage
\begin{thebibliography}{9}
    \bibitem{antennas}
    \textit{Виноградов А.Д., Минин Л.А., Морозов Е.Ю., Ушаков С.Н.}
    Детерминированный подход к решению задачи определения координат и угловой
    ориентации бортовой пеленгаторной антенны по результатам радиопеленгования
    радиоориентиров // Информационно-измерительные и управляющие системы, 2019, №1.
\end{thebibliography}

\end{document}