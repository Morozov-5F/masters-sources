\documentclass[a4paper,12pt]{article}

\usepackage[T2A]{fontenc}
\usepackage[utf8]{inputenc}
\usepackage[english,russian]{babel}

\usepackage{cmap}
\usepackage{amsmath}
\usepackage{graphicx}
\graphicspath{ {./pictures/} }

\usepackage[margin=1in]{geometry}

\author{Виноградов А.Д., Минин Л.А., Морозов Е.Ю., Ушаков С.Н.}
\title{Исследование различных детерминированных подходов в угломерной радиопеленгации}
\date{}

\begin{document}

\maketitle

\section{Случай локальной системы с НПУ}
В работе~\cite{antennas} рассматривалась локальная угломерная навигационная система (ЛУНС),
состоящая из трех наземных радиоориентиров и воздушного объекта, оснащенного радиолокациооным
оборудованием, которое способно определять азимут и угол места каждого из наземных радиоориентиров.
Авторами было показано, что определение пространственных координат и угловой ориентации воздушного
объекта сводится к решению нелинейной системы уравнений следующего вида:
\begin{equation}\label{eq:luns_start}
    \begin{cases}
        \ell_1^2 + \ell_2^2 - 2 \ell_1 \ell_2 \cos \alpha_{12} = d_{12}^2 \\
        \ell_1^2 + \ell_3^2 - 2 \ell_1 \ell_3 \cos \alpha_{13} = d_{13}^2 \\
        \ell_2^2 + \ell_3^2 - 2 \ell_2 \ell_3 \cos \alpha_{23} = d_{23}^2 \\
    \end{cases},
\end{equation}
где $\ell_i$ "--- это расстояние от воздушного объекта до $i$-го радиоориентира, $d_{i,j}$
"--- расстояние между $i$-м и $j$-м наземным радиоориетиром, $\cos \alpha_{i,j}$ "---
плоский угол, образованный наземными ориентирами $i$ и $j$ и воздушным объектом. Из этой
системы далее выводятся координаты и угловая ориентация воздушного объекта.

Система~(\ref{eq:luns_start}) может иметь несколько решений, что ведет к неоднозначности определения
координат и угловой ориентации летательного аппарата. Поэтому, в статье~\cite{antennas} приводятся
области, где решение можно определить однозначно. Более того, решение системы~(\ref{eq:luns_start})
производится численно, что потенциально может привести к дополнительным ошибкам в определении
пространственных параметров воздушного объекта. В связи с этим, далее предлагается еще один
вариант ЛУНС, которая лишена вышеперечисленных недостатков ценой усложнения структуры системы.

Предлагается заменить один из пассивных радиоориентиров на наземный пункт управления (НПУ), оснащенный
радиоориентиром. Требуется, чтобы НПУ мог определять азимут и угол места двух других радиоориентиров
и воздушного объекта.

По аналогии с~\cite{antennas}, будем считать, что воздушный объект находится в точке $M_0\left(x, y, z\right)$,
НПУ "--- в $M_1\left(x_1, y_1, z_1\right)$, а $i$-й РО "--- в точке $M_i\left(x_i, y_i, z_i \right)$, $i = 1,2$.
Азимут и угол места радиоориентира $M_i$, полученные в результате радиопеленгования, обозначим через
$\beta_i$ и $\theta_i$ соответственно (см. рис.~\ref{figure:pic1}). Предполагается, что измерения этих
углов проводятся в локальной системе координат НПУ, центр которой совпадает с координатами НПУ в НЗСК, а
направления осей совпадают с НЗСК.

\begin{figure}[htbp]
    \begin{center}

    \fbox{\includegraphics{pic1}}

    \caption{Схемы размещения на местности БпЛА, НПУ и РО для реализации ЛУНС}
    \label{figure:pic1}
    \end{center}
\end{figure}

% Здесь слова про углы M_0 M_1 M_j и то, как они получаются. Необходимо еще определить систему
% координат -- предполагается, что измерения ведутся в локальной СК

По теореме синусов для треугольника $M_1 M_0 M_2$:
\begin{equation}\label{eq:luns_sin_1}
    \frac{\ell_2}{\sin\alpha_{02}} = \frac{d_{12}}{\sin\alpha_{12}} = \frac{\ell_1}{\sin\left(\alpha_{12} + \alpha_{02}\right)}.
\end{equation}
Аналогично для треугольника $M_1 M_0 M_3$:
\begin{equation}\label{eq:luns_sin_2}
    \frac{\ell_3}{\sin\alpha_{03}} = \frac{d_{13}}{\sin\alpha_{13}} = \frac{\ell_1}{\sin\left(\alpha_{13} + \alpha_{03}\right)}.
\end{equation}
Таким образом, из~(\ref{eq:luns_sin_1}):
\begin{equation*}
    \ell_2 = \frac{d_{12}\sin\alpha_{02}}{\sin\alpha_{12}}.
\end{equation*}
В то же время, из~(\ref{eq:luns_sin_2}):
\begin{equation*}
    \ell_3 = \frac{d_{13}\sin\alpha_{03}}{\sin\alpha_{13}}.
\end{equation*}
Длина $\ell_1$ может быть найдена из любого уравнения:
\begin{equation*}
    \ell_1 = \frac{d_{12}\sin\left(\alpha_{12} + \alpha_{02}\right)}{\sin\alpha_{12}} = \frac{d_{13}\sin\left(\alpha_{13} + \alpha_{03}\right)}{\sin\alpha_{13}}.
\end{equation*}
Далее пространственные координаты и угловая ориентация БПА определяется согласно~\cite{antennas}.


\section{Случай автономной системы}
Автономная угломерная радионавигационная система (АУНС) предназначена для определения
координат и угловой ориентации в пространстве двух воздушных объектов, оснащенных
высотомерами и бортовыми радиоориентирами с наземного пункта управления (НПУ),
оснащенного радиоориентиром.
% Схема размещения на местности воздушных объектов и наземного пункта управления приведены на рис.~\ref{figure:pic3}.

Пусть радиоориентир НПУ расположен в точке $M_0$ с заранее известными координатами
$M_0\left(x_0, y_0, z_0\right)$ в Балтийской системе координат (БСК), а воздушные объекты
-- в точках $M_1$ и $M_2$ с координатами $M_i\left(x_i, y_i, z_i\right)$, при этом $z_i$
совпадает с измерениями высотомера $h_i$. НПУ $M_0$ способен измерять радиопеленг $\theta_i$
и угол возвышения $\beta_i$ $i$-го воздушного объекта в БСК. Воздушные объект $M_n$
способен измерять азимут $\alpha_{ni}$ и угол места $\varepsilon_{ni}$ $i$-го
радиоориентара (наземного или воздушного) в связанной системе координат БПА \cite{antennas}.
Схема размещения с указанными величинами указана на рис.~\ref{figure:pic3}.
Пространственное положение радиоорениров в БСК также можно охарактеризовать радиус-векторами
$\mathbf{r}_i = \left(x_i, y_i, z_i\right)$, где $i = 1,2,3$.

\begin{figure}[htbp]
    \begin{center}

    \fbox{\includegraphics{pic3}}

    \caption{Схемы размещения на местности БпЛА, НПУ и РО для реализации АУНС}
    \label{figure:pic3}
    \end{center}
\end{figure}

При детерминированном подходе для такой системы возможно однозначно определить координаты
и уголовую ориентацию воздушных обектов. Для этого нунжно выполнить следующие ключевые шаги:
\begin{enumerate}
    \item Определить координаты водушных объектов в БСК.
    \item Определить координаты радиоориентиров в связанной системе координат воздушного объекта $M_1$ ($\Sigma_{\text{св}1}$).
    \item Составить матрицу поворота системы координат $\Sigma_{\text{св}1}$ по алгоритму, представленному ниже.
    \item Определить углы поворота $\Sigma_{\text{св}1}$ по алгоритму, представленному в \cite{antennas}.
    \item Повторить предыдущие шаги для воздушного ориентира $M_2$.
\end{enumerate}

Первая часть алгоритма реализуется явно "--- совокупность данных с высотомеров воздушных
объектов и углов $\theta_i$, $\beta_i$ позволяют определить координаты летательных
аппаратов однозначно. Таким образом, координаты радиоориентира $M_1$ и $M_2$ определяются следующим
отношениями:
\begin{equation}
    \begin{cases}
        x_1 = \rho_1 \cos\theta_1 \cos\beta_1 \\
        y_1 = \rho_1 \cos\theta_1 \cos\beta_1 \\
        z_1 = h_1 = \rho_1 \sin\beta_1 \\
        \rho_1 = ~^{z_1}/_{\sin\beta_1}
    \end{cases},
    \begin{cases}
        x_2 = \rho_2 \cos\theta_2 \cos\beta_2 \\
        y_2 = \rho_2 \cos\theta_2 \cos\beta_2 \\
        z_2 = h_2 = \rho_2 \sin\beta_2 \\
        \rho_2 = ~^{z_2}/_{\sin\beta_2}
    \end{cases}.
\end{equation}

Далее необходимо определить координаты радиоориентиров $M_0$ и $M_2$ в связанной системе координат
воздушного объекта $M_1$:
\begin{equation}
    \begin{cases}
        x'_0 = \rho_{10} \cos\alpha_{10} \cos\varepsilon_{10} \\
        y'_0 = \rho_{10} \sin\alpha_{10} \cos\varepsilon_{10} \\
        z'_0 = z_1 - z_0 = \rho_{10} \sin\varepsilon_{10} \\
        \rho_{10} = |\mathbf{r}_1 - \mathbf{r}_0|
    \end{cases},
    \begin{cases}
        x'_2 = \rho_{12} \cos\alpha_{12} \cos\varepsilon_{12} \\
        y'_2 = \rho_{12} \sin\alpha_{12} \cos\varepsilon_{12} \\
        z'_2 = z_1 - z_2 = \rho_{12} \sin\varepsilon_{12} \\
        \rho_{12} = |\mathbf{r}_1 - \mathbf{r}_2|
    \end{cases},
\end{equation}
где $M_0'\left(x'_0, y'_0, z'_0\right)$ и $M_2'\left(x'_2, y'_2, z'_2\right)$
"--- координаты $M_0$ и $M_2$ в связанной СК $M_1$.

Для определения матрицы поворота системы координат $\Sigma_{\text{св}1}$, необходимо сначала
ввести радиус-вектора $\mathbf{r}'_0 = \left(x'_0, y'_0, z'_0\right)$ и
$\mathbf{r}'_2 = \left(x'_2, y'_2, z'_2\right)$, которые определяют положения радиоориентиров
$M_0$ и $M_2$ в связанной системе координат $M_1$. Далее, зададим единичные векторы
$\mathbf{s}'_1$, $\mathbf{s}'_2$ и $\mathbf{s}'_3$ следующим образом:
\begin{equation}
    \mathbf{s}'_1 = \frac{\mathbf{r}'_0}{|\mathbf{r}'_0|},\
    \mathbf{s}'_2 = \frac{\mathbf{r}'_2}{|\mathbf{r}'_2|},\
    \mathbf{s}'_3 = \mathbf{s}'_1 \times \mathbf{s}'_2.
\end{equation}
Те же вектора в балтийской системе координат:
\begin{equation}
    \mathbf{s}_{01} = \frac{\mathbf{r}_0 - \mathbf{r}_1}{|\mathbf{r}_0 - \mathbf{r}_1|},\
    \mathbf{s}_{21} = \frac{\mathbf{r}_2 - \mathbf{r}_1}{|\mathbf{r}_2 - \mathbf{r}_1|},\
    \mathbf{n}_1 = \mathbf{s}_{01} \times \mathbf{s}_{21}.
\end{equation}
Остюда получим следующее преобразование координат:
\begin{equation}
    \mathbf{R_1} \times
    \left(
    \begin{matrix}
        \mathbf{s}_{01} \\
        \mathbf{s}_{21} \\
        \mathbf{n}_1 \\
    \end{matrix}
    \right) =
    \left(
    \begin{matrix}
        \mathbf{s}'_1 \\
        \mathbf{s}'_2 \\
        \mathbf{s}'_3 \\
    \end{matrix}
    \right)
\end{equation}
В таком случае, матрицу поворта связанной системы кординат $\Sigma_{\text{св}1}$ можно найти следующим образом:
\begin{equation}
    \mathbf{R_1} =
    \left(
    \begin{matrix}
        \mathbf{s}'_1 \\
        \mathbf{s}'_2 \\
        \mathbf{s}'_3 \\
    \end{matrix}
    \right) \times
    \left(
    \begin{matrix}
        \mathbf{s}_{01} \\
        \mathbf{s}_{21} \\
        \mathbf{n}_1 \\
    \end{matrix}
    \right)^{-1}
\end{equation}

По аналогии можно получить матрицу поворота $\mathbf{R}_2$ связанной системы
кординат $M_2$. Вводятся радиус-вектора $\mathbf{r}''_0 = \left(x''_0, y''_0, z''_0\right)$,
$\mathbf{r}''_1 = \left(x''_1, y''_1, z''_1\right)$, по ним же определяются единичные вектора
$\mathbf{s}''_1$, $\mathbf{s}''_2$ и $\mathbf{s}''_3$:
\begin{equation}
    \mathbf{s}''_1 = \frac{\mathbf{r}''_0}{|\mathbf{r}''_0|},\
    \mathbf{s}''_2 = \frac{\mathbf{r}''_1}{|\mathbf{r}''_1|},\
    \mathbf{s}''_3 = \mathbf{s}''_1 \times \mathbf{s}''_2.
\end{equation}
В балтийской системе координат:
\begin{equation}
    \mathbf{s}_{02} = \frac{\mathbf{r}_0 - \mathbf{r}_2}{|\mathbf{r}_0 - \mathbf{r}_2|},\
    \mathbf{s}_{12} = \frac{\mathbf{r}_1 - \mathbf{r}_2}{|\mathbf{r}_1 - \mathbf{r}_2|},\
    \mathbf{n}_2 = \mathbf{s}_{02} \times \mathbf{s}_{12}.
\end{equation}
Отсюда, матрица поворота определяется следующим образом:
\begin{equation}
    \mathbf{R_2} =
    \left(
    \begin{matrix}
        \mathbf{s}''_1 \\
        \mathbf{s}''_2 \\
        \mathbf{s}''_3 \\
    \end{matrix}
    \right) \times
    \left(
    \begin{matrix}
        \mathbf{s}_{02} \\
        \mathbf{s}_{12} \\
        \mathbf{n}_2 \\
    \end{matrix}
    \right)^{-1}
\end{equation}

Углы курса $\psi_1$, $\psi_2$, крена $\mu_1$, $\mu_2$ и тангажа $\vartheta_1$, $\vartheta_2$ находятся
из матриц $\mathbf{R}_1$ и $\mathbf{R}_2$ в соответствии с~\cite{antennas}.

% и $\mathbf{n}_1$,
% которые определяются следующим образом:
% \begin{align*}
%     \mathbf{r}'_0 =&~\left(x'_0, y'_0, z'_0\right) / |\left(x'_0, y'_0, z'_0\right)|, \\
%     \mathbf{r}'_2 =&~\left(x'_2, y'_2, z'_2\right) / |\left(x'_0, y'_0, z'_0\right)|, \\
%     \mathbf{n}_1 =&~\mathbf{r}'_0 \times \mathbf{r}'_2
% \end{align*}

\newpage
\begin{thebibliography}{9}
    \bibitem{antennas}
    \textit{Виноградов А.Д., Минин Л.А., Морозов Е.Ю., Ушаков С.Н.}
    Детерминированный подход к решению задачи определения координат и угловой
    ориентации бортовой пеленгаторной антенны по результатам радиопеленгования
    радиоориентиров // Информационно-измерительные и управляющие системы, 2019, №1.
\end{thebibliography}

\end{document}